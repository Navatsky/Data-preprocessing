
    




    
\documentclass[11pt]{article}

    
    \usepackage[breakable]{tcolorbox}
    \tcbset{nobeforeafter} % prevents tcolorboxes being placing in paragraphs
    \usepackage{float}
    \floatplacement{figure}{H} % forces figures to be placed at the correct location
    
    \usepackage[T1]{fontenc}
    % Nicer default font (+ math font) than Computer Modern for most use cases
    \usepackage{mathpazo}

    % Basic figure setup, for now with no caption control since it's done
    % automatically by Pandoc (which extracts ![](path) syntax from Markdown).
    \usepackage{graphicx}
    % We will generate all images so they have a width \maxwidth. This means
    % that they will get their normal width if they fit onto the page, but
    % are scaled down if they would overflow the margins.
    \makeatletter
    \def\maxwidth{\ifdim\Gin@nat@width>\linewidth\linewidth
    \else\Gin@nat@width\fi}
    \makeatother
    \let\Oldincludegraphics\includegraphics
    % Set max figure width to be 80% of text width, for now hardcoded.
    \renewcommand{\includegraphics}[1]{\Oldincludegraphics[width=.8\maxwidth]{#1}}
    % Ensure that by default, figures have no caption (until we provide a
    % proper Figure object with a Caption API and a way to capture that
    % in the conversion process - todo).
    \usepackage{caption}
    \DeclareCaptionLabelFormat{nolabel}{}
    \captionsetup{labelformat=nolabel}

    \usepackage{adjustbox} % Used to constrain images to a maximum size 
    \usepackage{xcolor} % Allow colors to be defined
    \usepackage{enumerate} % Needed for markdown enumerations to work
    \usepackage{geometry} % Used to adjust the document margins
    \usepackage{amsmath} % Equations
    \usepackage{amssymb} % Equations
    \usepackage{textcomp} % defines textquotesingle
    % Hack from http://tex.stackexchange.com/a/47451/13684:
    \AtBeginDocument{%
        \def\PYZsq{\textquotesingle}% Upright quotes in Pygmentized code
    }
    \usepackage{upquote} % Upright quotes for verbatim code
    \usepackage{eurosym} % defines \euro
    \usepackage[mathletters]{ucs} % Extended unicode (utf-8) support
    \usepackage[utf8x]{inputenc} % Allow utf-8 characters in the tex document
    \usepackage{fancyvrb} % verbatim replacement that allows latex
    \usepackage{grffile} % extends the file name processing of package graphics 
                         % to support a larger range 
    % The hyperref package gives us a pdf with properly built
    % internal navigation ('pdf bookmarks' for the table of contents,
    % internal cross-reference links, web links for URLs, etc.)
    \usepackage{hyperref}
    \usepackage{longtable} % longtable support required by pandoc >1.10
    \usepackage{booktabs}  % table support for pandoc > 1.12.2
    \usepackage[inline]{enumitem} % IRkernel/repr support (it uses the enumerate* environment)
    \usepackage[normalem]{ulem} % ulem is needed to support strikethroughs (\sout)
                                % normalem makes italics be italics, not underlines
    \usepackage{mathrsfs}
    

    
    % Colors for the hyperref package
    \definecolor{urlcolor}{rgb}{0,.145,.698}
    \definecolor{linkcolor}{rgb}{.71,0.21,0.01}
    \definecolor{citecolor}{rgb}{.12,.54,.11}

    % ANSI colors
    \definecolor{ansi-black}{HTML}{3E424D}
    \definecolor{ansi-black-intense}{HTML}{282C36}
    \definecolor{ansi-red}{HTML}{E75C58}
    \definecolor{ansi-red-intense}{HTML}{B22B31}
    \definecolor{ansi-green}{HTML}{00A250}
    \definecolor{ansi-green-intense}{HTML}{007427}
    \definecolor{ansi-yellow}{HTML}{DDB62B}
    \definecolor{ansi-yellow-intense}{HTML}{B27D12}
    \definecolor{ansi-blue}{HTML}{208FFB}
    \definecolor{ansi-blue-intense}{HTML}{0065CA}
    \definecolor{ansi-magenta}{HTML}{D160C4}
    \definecolor{ansi-magenta-intense}{HTML}{A03196}
    \definecolor{ansi-cyan}{HTML}{60C6C8}
    \definecolor{ansi-cyan-intense}{HTML}{258F8F}
    \definecolor{ansi-white}{HTML}{C5C1B4}
    \definecolor{ansi-white-intense}{HTML}{A1A6B2}
    \definecolor{ansi-default-inverse-fg}{HTML}{FFFFFF}
    \definecolor{ansi-default-inverse-bg}{HTML}{000000}

    % commands and environments needed by pandoc snippets
    % extracted from the output of `pandoc -s`
    \providecommand{\tightlist}{%
      \setlength{\itemsep}{0pt}\setlength{\parskip}{0pt}}
    \DefineVerbatimEnvironment{Highlighting}{Verbatim}{commandchars=\\\{\}}
    % Add ',fontsize=\small' for more characters per line
    \newenvironment{Shaded}{}{}
    \newcommand{\KeywordTok}[1]{\textcolor[rgb]{0.00,0.44,0.13}{\textbf{{#1}}}}
    \newcommand{\DataTypeTok}[1]{\textcolor[rgb]{0.56,0.13,0.00}{{#1}}}
    \newcommand{\DecValTok}[1]{\textcolor[rgb]{0.25,0.63,0.44}{{#1}}}
    \newcommand{\BaseNTok}[1]{\textcolor[rgb]{0.25,0.63,0.44}{{#1}}}
    \newcommand{\FloatTok}[1]{\textcolor[rgb]{0.25,0.63,0.44}{{#1}}}
    \newcommand{\CharTok}[1]{\textcolor[rgb]{0.25,0.44,0.63}{{#1}}}
    \newcommand{\StringTok}[1]{\textcolor[rgb]{0.25,0.44,0.63}{{#1}}}
    \newcommand{\CommentTok}[1]{\textcolor[rgb]{0.38,0.63,0.69}{\textit{{#1}}}}
    \newcommand{\OtherTok}[1]{\textcolor[rgb]{0.00,0.44,0.13}{{#1}}}
    \newcommand{\AlertTok}[1]{\textcolor[rgb]{1.00,0.00,0.00}{\textbf{{#1}}}}
    \newcommand{\FunctionTok}[1]{\textcolor[rgb]{0.02,0.16,0.49}{{#1}}}
    \newcommand{\RegionMarkerTok}[1]{{#1}}
    \newcommand{\ErrorTok}[1]{\textcolor[rgb]{1.00,0.00,0.00}{\textbf{{#1}}}}
    \newcommand{\NormalTok}[1]{{#1}}
    
    % Additional commands for more recent versions of Pandoc
    \newcommand{\ConstantTok}[1]{\textcolor[rgb]{0.53,0.00,0.00}{{#1}}}
    \newcommand{\SpecialCharTok}[1]{\textcolor[rgb]{0.25,0.44,0.63}{{#1}}}
    \newcommand{\VerbatimStringTok}[1]{\textcolor[rgb]{0.25,0.44,0.63}{{#1}}}
    \newcommand{\SpecialStringTok}[1]{\textcolor[rgb]{0.73,0.40,0.53}{{#1}}}
    \newcommand{\ImportTok}[1]{{#1}}
    \newcommand{\DocumentationTok}[1]{\textcolor[rgb]{0.73,0.13,0.13}{\textit{{#1}}}}
    \newcommand{\AnnotationTok}[1]{\textcolor[rgb]{0.38,0.63,0.69}{\textbf{\textit{{#1}}}}}
    \newcommand{\CommentVarTok}[1]{\textcolor[rgb]{0.38,0.63,0.69}{\textbf{\textit{{#1}}}}}
    \newcommand{\VariableTok}[1]{\textcolor[rgb]{0.10,0.09,0.49}{{#1}}}
    \newcommand{\ControlFlowTok}[1]{\textcolor[rgb]{0.00,0.44,0.13}{\textbf{{#1}}}}
    \newcommand{\OperatorTok}[1]{\textcolor[rgb]{0.40,0.40,0.40}{{#1}}}
    \newcommand{\BuiltInTok}[1]{{#1}}
    \newcommand{\ExtensionTok}[1]{{#1}}
    \newcommand{\PreprocessorTok}[1]{\textcolor[rgb]{0.74,0.48,0.00}{{#1}}}
    \newcommand{\AttributeTok}[1]{\textcolor[rgb]{0.49,0.56,0.16}{{#1}}}
    \newcommand{\InformationTok}[1]{\textcolor[rgb]{0.38,0.63,0.69}{\textbf{\textit{{#1}}}}}
    \newcommand{\WarningTok}[1]{\textcolor[rgb]{0.38,0.63,0.69}{\textbf{\textit{{#1}}}}}
    
    
    % Define a nice break command that doesn't care if a line doesn't already
    % exist.
    \def\br{\hspace*{\fill} \\* }
    % Math Jax compatibility definitions
    \def\gt{>}
    \def\lt{<}
    \let\Oldtex\TeX
    \let\Oldlatex\LaTeX
    \renewcommand{\TeX}{\textrm{\Oldtex}}
    \renewcommand{\LaTeX}{\textrm{\Oldlatex}}
    % Document parameters
    % Document title
    \title{a9bf9961-b792-4376-b522-c0aed1b67d60}
    
    
    
    
    
% Pygments definitions
\makeatletter
\def\PY@reset{\let\PY@it=\relax \let\PY@bf=\relax%
    \let\PY@ul=\relax \let\PY@tc=\relax%
    \let\PY@bc=\relax \let\PY@ff=\relax}
\def\PY@tok#1{\csname PY@tok@#1\endcsname}
\def\PY@toks#1+{\ifx\relax#1\empty\else%
    \PY@tok{#1}\expandafter\PY@toks\fi}
\def\PY@do#1{\PY@bc{\PY@tc{\PY@ul{%
    \PY@it{\PY@bf{\PY@ff{#1}}}}}}}
\def\PY#1#2{\PY@reset\PY@toks#1+\relax+\PY@do{#2}}

\expandafter\def\csname PY@tok@w\endcsname{\def\PY@tc##1{\textcolor[rgb]{0.73,0.73,0.73}{##1}}}
\expandafter\def\csname PY@tok@c\endcsname{\let\PY@it=\textit\def\PY@tc##1{\textcolor[rgb]{0.25,0.50,0.50}{##1}}}
\expandafter\def\csname PY@tok@cp\endcsname{\def\PY@tc##1{\textcolor[rgb]{0.74,0.48,0.00}{##1}}}
\expandafter\def\csname PY@tok@k\endcsname{\let\PY@bf=\textbf\def\PY@tc##1{\textcolor[rgb]{0.00,0.50,0.00}{##1}}}
\expandafter\def\csname PY@tok@kp\endcsname{\def\PY@tc##1{\textcolor[rgb]{0.00,0.50,0.00}{##1}}}
\expandafter\def\csname PY@tok@kt\endcsname{\def\PY@tc##1{\textcolor[rgb]{0.69,0.00,0.25}{##1}}}
\expandafter\def\csname PY@tok@o\endcsname{\def\PY@tc##1{\textcolor[rgb]{0.40,0.40,0.40}{##1}}}
\expandafter\def\csname PY@tok@ow\endcsname{\let\PY@bf=\textbf\def\PY@tc##1{\textcolor[rgb]{0.67,0.13,1.00}{##1}}}
\expandafter\def\csname PY@tok@nb\endcsname{\def\PY@tc##1{\textcolor[rgb]{0.00,0.50,0.00}{##1}}}
\expandafter\def\csname PY@tok@nf\endcsname{\def\PY@tc##1{\textcolor[rgb]{0.00,0.00,1.00}{##1}}}
\expandafter\def\csname PY@tok@nc\endcsname{\let\PY@bf=\textbf\def\PY@tc##1{\textcolor[rgb]{0.00,0.00,1.00}{##1}}}
\expandafter\def\csname PY@tok@nn\endcsname{\let\PY@bf=\textbf\def\PY@tc##1{\textcolor[rgb]{0.00,0.00,1.00}{##1}}}
\expandafter\def\csname PY@tok@ne\endcsname{\let\PY@bf=\textbf\def\PY@tc##1{\textcolor[rgb]{0.82,0.25,0.23}{##1}}}
\expandafter\def\csname PY@tok@nv\endcsname{\def\PY@tc##1{\textcolor[rgb]{0.10,0.09,0.49}{##1}}}
\expandafter\def\csname PY@tok@no\endcsname{\def\PY@tc##1{\textcolor[rgb]{0.53,0.00,0.00}{##1}}}
\expandafter\def\csname PY@tok@nl\endcsname{\def\PY@tc##1{\textcolor[rgb]{0.63,0.63,0.00}{##1}}}
\expandafter\def\csname PY@tok@ni\endcsname{\let\PY@bf=\textbf\def\PY@tc##1{\textcolor[rgb]{0.60,0.60,0.60}{##1}}}
\expandafter\def\csname PY@tok@na\endcsname{\def\PY@tc##1{\textcolor[rgb]{0.49,0.56,0.16}{##1}}}
\expandafter\def\csname PY@tok@nt\endcsname{\let\PY@bf=\textbf\def\PY@tc##1{\textcolor[rgb]{0.00,0.50,0.00}{##1}}}
\expandafter\def\csname PY@tok@nd\endcsname{\def\PY@tc##1{\textcolor[rgb]{0.67,0.13,1.00}{##1}}}
\expandafter\def\csname PY@tok@s\endcsname{\def\PY@tc##1{\textcolor[rgb]{0.73,0.13,0.13}{##1}}}
\expandafter\def\csname PY@tok@sd\endcsname{\let\PY@it=\textit\def\PY@tc##1{\textcolor[rgb]{0.73,0.13,0.13}{##1}}}
\expandafter\def\csname PY@tok@si\endcsname{\let\PY@bf=\textbf\def\PY@tc##1{\textcolor[rgb]{0.73,0.40,0.53}{##1}}}
\expandafter\def\csname PY@tok@se\endcsname{\let\PY@bf=\textbf\def\PY@tc##1{\textcolor[rgb]{0.73,0.40,0.13}{##1}}}
\expandafter\def\csname PY@tok@sr\endcsname{\def\PY@tc##1{\textcolor[rgb]{0.73,0.40,0.53}{##1}}}
\expandafter\def\csname PY@tok@ss\endcsname{\def\PY@tc##1{\textcolor[rgb]{0.10,0.09,0.49}{##1}}}
\expandafter\def\csname PY@tok@sx\endcsname{\def\PY@tc##1{\textcolor[rgb]{0.00,0.50,0.00}{##1}}}
\expandafter\def\csname PY@tok@m\endcsname{\def\PY@tc##1{\textcolor[rgb]{0.40,0.40,0.40}{##1}}}
\expandafter\def\csname PY@tok@gh\endcsname{\let\PY@bf=\textbf\def\PY@tc##1{\textcolor[rgb]{0.00,0.00,0.50}{##1}}}
\expandafter\def\csname PY@tok@gu\endcsname{\let\PY@bf=\textbf\def\PY@tc##1{\textcolor[rgb]{0.50,0.00,0.50}{##1}}}
\expandafter\def\csname PY@tok@gd\endcsname{\def\PY@tc##1{\textcolor[rgb]{0.63,0.00,0.00}{##1}}}
\expandafter\def\csname PY@tok@gi\endcsname{\def\PY@tc##1{\textcolor[rgb]{0.00,0.63,0.00}{##1}}}
\expandafter\def\csname PY@tok@gr\endcsname{\def\PY@tc##1{\textcolor[rgb]{1.00,0.00,0.00}{##1}}}
\expandafter\def\csname PY@tok@ge\endcsname{\let\PY@it=\textit}
\expandafter\def\csname PY@tok@gs\endcsname{\let\PY@bf=\textbf}
\expandafter\def\csname PY@tok@gp\endcsname{\let\PY@bf=\textbf\def\PY@tc##1{\textcolor[rgb]{0.00,0.00,0.50}{##1}}}
\expandafter\def\csname PY@tok@go\endcsname{\def\PY@tc##1{\textcolor[rgb]{0.53,0.53,0.53}{##1}}}
\expandafter\def\csname PY@tok@gt\endcsname{\def\PY@tc##1{\textcolor[rgb]{0.00,0.27,0.87}{##1}}}
\expandafter\def\csname PY@tok@err\endcsname{\def\PY@bc##1{\setlength{\fboxsep}{0pt}\fcolorbox[rgb]{1.00,0.00,0.00}{1,1,1}{\strut ##1}}}
\expandafter\def\csname PY@tok@kc\endcsname{\let\PY@bf=\textbf\def\PY@tc##1{\textcolor[rgb]{0.00,0.50,0.00}{##1}}}
\expandafter\def\csname PY@tok@kd\endcsname{\let\PY@bf=\textbf\def\PY@tc##1{\textcolor[rgb]{0.00,0.50,0.00}{##1}}}
\expandafter\def\csname PY@tok@kn\endcsname{\let\PY@bf=\textbf\def\PY@tc##1{\textcolor[rgb]{0.00,0.50,0.00}{##1}}}
\expandafter\def\csname PY@tok@kr\endcsname{\let\PY@bf=\textbf\def\PY@tc##1{\textcolor[rgb]{0.00,0.50,0.00}{##1}}}
\expandafter\def\csname PY@tok@bp\endcsname{\def\PY@tc##1{\textcolor[rgb]{0.00,0.50,0.00}{##1}}}
\expandafter\def\csname PY@tok@fm\endcsname{\def\PY@tc##1{\textcolor[rgb]{0.00,0.00,1.00}{##1}}}
\expandafter\def\csname PY@tok@vc\endcsname{\def\PY@tc##1{\textcolor[rgb]{0.10,0.09,0.49}{##1}}}
\expandafter\def\csname PY@tok@vg\endcsname{\def\PY@tc##1{\textcolor[rgb]{0.10,0.09,0.49}{##1}}}
\expandafter\def\csname PY@tok@vi\endcsname{\def\PY@tc##1{\textcolor[rgb]{0.10,0.09,0.49}{##1}}}
\expandafter\def\csname PY@tok@vm\endcsname{\def\PY@tc##1{\textcolor[rgb]{0.10,0.09,0.49}{##1}}}
\expandafter\def\csname PY@tok@sa\endcsname{\def\PY@tc##1{\textcolor[rgb]{0.73,0.13,0.13}{##1}}}
\expandafter\def\csname PY@tok@sb\endcsname{\def\PY@tc##1{\textcolor[rgb]{0.73,0.13,0.13}{##1}}}
\expandafter\def\csname PY@tok@sc\endcsname{\def\PY@tc##1{\textcolor[rgb]{0.73,0.13,0.13}{##1}}}
\expandafter\def\csname PY@tok@dl\endcsname{\def\PY@tc##1{\textcolor[rgb]{0.73,0.13,0.13}{##1}}}
\expandafter\def\csname PY@tok@s2\endcsname{\def\PY@tc##1{\textcolor[rgb]{0.73,0.13,0.13}{##1}}}
\expandafter\def\csname PY@tok@sh\endcsname{\def\PY@tc##1{\textcolor[rgb]{0.73,0.13,0.13}{##1}}}
\expandafter\def\csname PY@tok@s1\endcsname{\def\PY@tc##1{\textcolor[rgb]{0.73,0.13,0.13}{##1}}}
\expandafter\def\csname PY@tok@mb\endcsname{\def\PY@tc##1{\textcolor[rgb]{0.40,0.40,0.40}{##1}}}
\expandafter\def\csname PY@tok@mf\endcsname{\def\PY@tc##1{\textcolor[rgb]{0.40,0.40,0.40}{##1}}}
\expandafter\def\csname PY@tok@mh\endcsname{\def\PY@tc##1{\textcolor[rgb]{0.40,0.40,0.40}{##1}}}
\expandafter\def\csname PY@tok@mi\endcsname{\def\PY@tc##1{\textcolor[rgb]{0.40,0.40,0.40}{##1}}}
\expandafter\def\csname PY@tok@il\endcsname{\def\PY@tc##1{\textcolor[rgb]{0.40,0.40,0.40}{##1}}}
\expandafter\def\csname PY@tok@mo\endcsname{\def\PY@tc##1{\textcolor[rgb]{0.40,0.40,0.40}{##1}}}
\expandafter\def\csname PY@tok@ch\endcsname{\let\PY@it=\textit\def\PY@tc##1{\textcolor[rgb]{0.25,0.50,0.50}{##1}}}
\expandafter\def\csname PY@tok@cm\endcsname{\let\PY@it=\textit\def\PY@tc##1{\textcolor[rgb]{0.25,0.50,0.50}{##1}}}
\expandafter\def\csname PY@tok@cpf\endcsname{\let\PY@it=\textit\def\PY@tc##1{\textcolor[rgb]{0.25,0.50,0.50}{##1}}}
\expandafter\def\csname PY@tok@c1\endcsname{\let\PY@it=\textit\def\PY@tc##1{\textcolor[rgb]{0.25,0.50,0.50}{##1}}}
\expandafter\def\csname PY@tok@cs\endcsname{\let\PY@it=\textit\def\PY@tc##1{\textcolor[rgb]{0.25,0.50,0.50}{##1}}}

\def\PYZbs{\char`\\}
\def\PYZus{\char`\_}
\def\PYZob{\char`\{}
\def\PYZcb{\char`\}}
\def\PYZca{\char`\^}
\def\PYZam{\char`\&}
\def\PYZlt{\char`\<}
\def\PYZgt{\char`\>}
\def\PYZsh{\char`\#}
\def\PYZpc{\char`\%}
\def\PYZdl{\char`\$}
\def\PYZhy{\char`\-}
\def\PYZsq{\char`\'}
\def\PYZdq{\char`\"}
\def\PYZti{\char`\~}
% for compatibility with earlier versions
\def\PYZat{@}
\def\PYZlb{[}
\def\PYZrb{]}
\makeatother


    % For linebreaks inside Verbatim environment from package fancyvrb. 
    \makeatletter
        \newbox\Wrappedcontinuationbox 
        \newbox\Wrappedvisiblespacebox 
        \newcommand*\Wrappedvisiblespace {\textcolor{red}{\textvisiblespace}} 
        \newcommand*\Wrappedcontinuationsymbol {\textcolor{red}{\llap{\tiny$\m@th\hookrightarrow$}}} 
        \newcommand*\Wrappedcontinuationindent {3ex } 
        \newcommand*\Wrappedafterbreak {\kern\Wrappedcontinuationindent\copy\Wrappedcontinuationbox} 
        % Take advantage of the already applied Pygments mark-up to insert 
        % potential linebreaks for TeX processing. 
        %        {, <, #, %, $, ' and ": go to next line. 
        %        _, }, ^, &, >, - and ~: stay at end of broken line. 
        % Use of \textquotesingle for straight quote. 
        \newcommand*\Wrappedbreaksatspecials {% 
            \def\PYGZus{\discretionary{\char`\_}{\Wrappedafterbreak}{\char`\_}}% 
            \def\PYGZob{\discretionary{}{\Wrappedafterbreak\char`\{}{\char`\{}}% 
            \def\PYGZcb{\discretionary{\char`\}}{\Wrappedafterbreak}{\char`\}}}% 
            \def\PYGZca{\discretionary{\char`\^}{\Wrappedafterbreak}{\char`\^}}% 
            \def\PYGZam{\discretionary{\char`\&}{\Wrappedafterbreak}{\char`\&}}% 
            \def\PYGZlt{\discretionary{}{\Wrappedafterbreak\char`\<}{\char`\<}}% 
            \def\PYGZgt{\discretionary{\char`\>}{\Wrappedafterbreak}{\char`\>}}% 
            \def\PYGZsh{\discretionary{}{\Wrappedafterbreak\char`\#}{\char`\#}}% 
            \def\PYGZpc{\discretionary{}{\Wrappedafterbreak\char`\%}{\char`\%}}% 
            \def\PYGZdl{\discretionary{}{\Wrappedafterbreak\char`\$}{\char`\$}}% 
            \def\PYGZhy{\discretionary{\char`\-}{\Wrappedafterbreak}{\char`\-}}% 
            \def\PYGZsq{\discretionary{}{\Wrappedafterbreak\textquotesingle}{\textquotesingle}}% 
            \def\PYGZdq{\discretionary{}{\Wrappedafterbreak\char`\"}{\char`\"}}% 
            \def\PYGZti{\discretionary{\char`\~}{\Wrappedafterbreak}{\char`\~}}% 
        } 
        % Some characters . , ; ? ! / are not pygmentized. 
        % This macro makes them "active" and they will insert potential linebreaks 
        \newcommand*\Wrappedbreaksatpunct {% 
            \lccode`\~`\.\lowercase{\def~}{\discretionary{\hbox{\char`\.}}{\Wrappedafterbreak}{\hbox{\char`\.}}}% 
            \lccode`\~`\,\lowercase{\def~}{\discretionary{\hbox{\char`\,}}{\Wrappedafterbreak}{\hbox{\char`\,}}}% 
            \lccode`\~`\;\lowercase{\def~}{\discretionary{\hbox{\char`\;}}{\Wrappedafterbreak}{\hbox{\char`\;}}}% 
            \lccode`\~`\:\lowercase{\def~}{\discretionary{\hbox{\char`\:}}{\Wrappedafterbreak}{\hbox{\char`\:}}}% 
            \lccode`\~`\?\lowercase{\def~}{\discretionary{\hbox{\char`\?}}{\Wrappedafterbreak}{\hbox{\char`\?}}}% 
            \lccode`\~`\!\lowercase{\def~}{\discretionary{\hbox{\char`\!}}{\Wrappedafterbreak}{\hbox{\char`\!}}}% 
            \lccode`\~`\/\lowercase{\def~}{\discretionary{\hbox{\char`\/}}{\Wrappedafterbreak}{\hbox{\char`\/}}}% 
            \catcode`\.\active
            \catcode`\,\active 
            \catcode`\;\active
            \catcode`\:\active
            \catcode`\?\active
            \catcode`\!\active
            \catcode`\/\active 
            \lccode`\~`\~ 	
        }
    \makeatother

    \let\OriginalVerbatim=\Verbatim
    \makeatletter
    \renewcommand{\Verbatim}[1][1]{%
        %\parskip\z@skip
        \sbox\Wrappedcontinuationbox {\Wrappedcontinuationsymbol}%
        \sbox\Wrappedvisiblespacebox {\FV@SetupFont\Wrappedvisiblespace}%
        \def\FancyVerbFormatLine ##1{\hsize\linewidth
            \vtop{\raggedright\hyphenpenalty\z@\exhyphenpenalty\z@
                \doublehyphendemerits\z@\finalhyphendemerits\z@
                \strut ##1\strut}%
        }%
        % If the linebreak is at a space, the latter will be displayed as visible
        % space at end of first line, and a continuation symbol starts next line.
        % Stretch/shrink are however usually zero for typewriter font.
        \def\FV@Space {%
            \nobreak\hskip\z@ plus\fontdimen3\font minus\fontdimen4\font
            \discretionary{\copy\Wrappedvisiblespacebox}{\Wrappedafterbreak}
            {\kern\fontdimen2\font}%
        }%
        
        % Allow breaks at special characters using \PYG... macros.
        \Wrappedbreaksatspecials
        % Breaks at punctuation characters . , ; ? ! and / need catcode=\active 	
        \OriginalVerbatim[#1,codes*=\Wrappedbreaksatpunct]%
    }
    \makeatother

    % Exact colors from NB
    \definecolor{incolor}{HTML}{303F9F}
    \definecolor{outcolor}{HTML}{D84315}
    \definecolor{cellborder}{HTML}{CFCFCF}
    \definecolor{cellbackground}{HTML}{F7F7F7}
    
    % prompt
    \newcommand{\prompt}[4]{
        \llap{{\color{#2}[#3]: #4}}\vspace{-1.25em}
    }
    

    
    % Prevent overflowing lines due to hard-to-break entities
    \sloppy 
    % Setup hyperref package
    \hypersetup{
      breaklinks=true,  % so long urls are correctly broken across lines
      colorlinks=true,
      urlcolor=urlcolor,
      linkcolor=linkcolor,
      citecolor=citecolor,
      }
    % Slightly bigger margins than the latex defaults
    
    \geometry{verbose,tmargin=1in,bmargin=1in,lmargin=1in,rmargin=1in}
    
    

    \begin{document}
    
    
    \maketitle
    
    

    
    Привет еще раз. Спасибо, что доделал работу. Мои комментарии к
исправленным замечаниям будут выделены зеленым цветом. Если же
потребуется доработка некоторых пунктов проекта, то я отмечу это красным
цветом.

\begin{center}\rule{0.5\linewidth}{0.5pt}\end{center}

    \hypertarget{ux438ux441ux441ux43bux435ux434ux43eux432ux430ux43dux438ux435-ux43dux430ux434ux451ux436ux43dux43eux441ux442ux438-ux437ux430ux451ux43cux449ux438ux43aux43eux432}{%
\section{Исследование надёжности
заёмщиков}\label{ux438ux441ux441ux43bux435ux434ux43eux432ux430ux43dux438ux435-ux43dux430ux434ux451ux436ux43dux43eux441ux442ux438-ux437ux430ux451ux43cux449ux438ux43aux43eux432}}

    Комментарий наставника

Привет! Спасибо, что прислал задание:) Поздравляю с первым сданным
проектом. Мои комментарии обозначены пометкой \textbf{Комментарий
наставника}. Далее в файле ты сможешь найти их в похожих ячейках (если
рамки комментария зелёные - всё сделано правильно; оранжевые - есть
замечания, но не критично; красные - нужно переделать). Не удаляй эти
комментарии и постарайся учесть их в ходе выполнения данного проекта.

    Заказчик --- кредитный отдел банка. Нужно разобраться, влияет ли
семейное положение и количество детей клиента на факт погашения кредита
в срок. Входные данные от банка --- статистика о платёжеспособности
клиентов.

Результаты исследования будут учтены при построении модели
\textbf{кредитного скоринга} --- специальной системы, которая оценивает
способность потенциального заёмщика вернуть кредит банку.

Описание данных

children --- количество детей в семье

days\_employed --- общий трудовой стаж в днях

dob\_years --- возраст клиента в годах

education --- уровень образования клиента

education\_id --- идентификатор уровня образования

family\_status --- семейное положение

family\_status\_id --- идентификатор семейного положения

gender --- пол клиента

income\_type --- тип занятости

debt --- имел ли задолженность по возврату кредитов

total\_income --- ежемесячный доход

purpose --- цель получения кредита

    Комментарий наставника

Правильно, что есть краткое вступление в работу, описание того, что надо
делать. В работе необходимо приводить краткий план того, что надо
сделать(если этот план имеется), а также информацию о входных данных:
какие столбцы есть в таблице, их названия и какую информацию они несут
(см. пример). Также название работы лучше отображать в отдельной ячейке
и делать крупный шрифт (заголовок). Так работа выглядит презентабельно.

\hypertarget{ux43fux440ux438ux43cux435ux440}{%
\subsection{Пример:}\label{ux43fux440ux438ux43cux435ux440}}

Описание данных: - children --- количество детей в семье -
days\_employed --- общий трудовой стаж в днях - dob\_years --- возраст
клиента в годах - education --- уровень образования клиента -
education\_id --- идентификатор уровня образования - \ldots{}

    DONE

    \hypertarget{ux448ux430ux433-1.-ux43eux442ux43aux440ux43eux439ux442ux435-ux444ux430ux439ux43b-ux441-ux434ux430ux43dux43dux44bux43cux438-ux438-ux438ux437ux443ux447ux438ux442ux435-ux43eux431ux449ux443ux44e-ux438ux43dux444ux43eux440ux43cux430ux446ux438ux44e.}{%
\subsubsection{Шаг 1. Откройте файл с данными и изучите общую
информацию.}\label{ux448ux430ux433-1.-ux43eux442ux43aux440ux43eux439ux442ux435-ux444ux430ux439ux43b-ux441-ux434ux430ux43dux43dux44bux43cux438-ux438-ux438ux437ux443ux447ux438ux442ux435-ux43eux431ux449ux443ux44e-ux438ux43dux444ux43eux440ux43cux430ux446ux438ux44e.}}

    \begin{tcolorbox}[breakable, size=fbox, boxrule=1pt, pad at break*=1mm,colback=cellbackground, colframe=cellborder]
\prompt{In}{incolor}{97}{\hspace{4pt}}
\begin{Verbatim}[commandchars=\\\{\}]
\PY{k+kn}{import} \PY{n+nn}{pandas} \PY{k}{as} \PY{n+nn}{pd}
\end{Verbatim}
\end{tcolorbox}

    \begin{tcolorbox}[breakable, size=fbox, boxrule=1pt, pad at break*=1mm,colback=cellbackground, colframe=cellborder]
\prompt{In}{incolor}{98}{\hspace{4pt}}
\begin{Verbatim}[commandchars=\\\{\}]
\PY{n}{bank\PYZus{}clients} \PY{o}{=} \PY{n}{pd}\PY{o}{.}\PY{n}{read\PYZus{}csv}\PY{p}{(}\PY{l+s+s1}{\PYZsq{}}\PY{l+s+s1}{/datasets/data.csv}\PY{l+s+s1}{\PYZsq{}}\PY{p}{)}
\end{Verbatim}
\end{tcolorbox}

    До начала всей работы проверим дубликаты и удалим

    \begin{tcolorbox}[breakable, size=fbox, boxrule=1pt, pad at break*=1mm,colback=cellbackground, colframe=cellborder]
\prompt{In}{incolor}{99}{\hspace{4pt}}
\begin{Verbatim}[commandchars=\\\{\}]
\PY{n}{x} \PY{o}{=} \PY{n}{bank\PYZus{}clients}\PY{o}{.}\PY{n}{duplicated}\PY{p}{(}\PY{p}{)}\PY{o}{.}\PY{n}{sum}\PY{p}{(}\PY{p}{)}
\PY{n+nb}{print}\PY{p}{(}\PY{l+s+s1}{\PYZsq{}}\PY{l+s+s1}{На данный момент дубликатов}\PY{l+s+s1}{\PYZsq{}}\PY{p}{,} \PY{n}{x}\PY{p}{)}
\PY{n}{bank\PYZus{}clients} \PY{o}{=} \PY{n}{bank\PYZus{}clients}\PY{o}{.}\PY{n}{drop\PYZus{}duplicates}\PY{p}{(}\PY{p}{)}\PY{o}{.}\PY{n}{reset\PYZus{}index}\PY{p}{(}\PY{n}{drop}\PY{o}{=}\PY{k+kc}{True}\PY{p}{)}
\end{Verbatim}
\end{tcolorbox}

    \begin{Verbatim}[commandchars=\\\{\}]
На данный момент дубликатов 54
\end{Verbatim}

    Изучим таблицу Изучим информацию по таблице

    \begin{tcolorbox}[breakable, size=fbox, boxrule=1pt, pad at break*=1mm,colback=cellbackground, colframe=cellborder]
\prompt{In}{incolor}{100}{\hspace{4pt}}
\begin{Verbatim}[commandchars=\\\{\}]
\PY{n+nb}{print}\PY{p}{(}\PY{n}{bank\PYZus{}clients}\PY{o}{.}\PY{n}{info}\PY{p}{(}\PY{p}{)}\PY{p}{)}
\PY{n+nb}{print}\PY{p}{(}\PY{l+s+s1}{\PYZsq{}}\PY{l+s+s1}{\PYZsq{}}\PY{p}{)}
\PY{n+nb}{print} \PY{p}{(}\PY{l+s+s1}{\PYZsq{}}\PY{l+s+s1}{Пропущено в столбцах }\PY{l+s+s1}{\PYZdq{}}\PY{l+s+s1}{days\PYZus{}employed}\PY{l+s+s1}{\PYZdq{}}\PY{l+s+s1}{ и }\PY{l+s+s1}{\PYZdq{}}\PY{l+s+s1}{total\PYZus{}income}\PY{l+s+s1}{\PYZdq{}}\PY{l+s+s1}{\PYZsq{}}\PY{p}{,} \PY{l+m+mi}{21525}\PY{o}{\PYZhy{}}\PY{l+m+mi}{19351}\PY{p}{)}
\PY{n+nb}{print}\PY{p}{(}\PY{l+s+s1}{\PYZsq{}}\PY{l+s+s1}{\PYZsq{}}\PY{p}{)}
\PY{n}{bank\PYZus{}clients}\PY{o}{.}\PY{n}{head}\PY{p}{(}\PY{l+m+mi}{11}\PY{p}{)}
\end{Verbatim}
\end{tcolorbox}

    \begin{Verbatim}[commandchars=\\\{\}]
<class 'pandas.core.frame.DataFrame'>
RangeIndex: 21471 entries, 0 to 21470
Data columns (total 12 columns):
children            21471 non-null int64
days\_employed       19351 non-null float64
dob\_years           21471 non-null int64
education           21471 non-null object
education\_id        21471 non-null int64
family\_status       21471 non-null object
family\_status\_id    21471 non-null int64
gender              21471 non-null object
income\_type         21471 non-null object
debt                21471 non-null int64
total\_income        19351 non-null float64
purpose             21471 non-null object
dtypes: float64(2), int64(5), object(5)
memory usage: 2.0+ MB
None

Пропущено в столбцах "days\_employed" и "total\_income" 2174

\end{Verbatim}

            \begin{tcolorbox}[breakable, boxrule=.5pt, size=fbox, pad at break*=1mm, opacityfill=0]
\prompt{Out}{outcolor}{100}{\hspace{3.5pt}}
\begin{Verbatim}[commandchars=\\\{\}]
    children  days\_employed  dob\_years education  education\_id  \textbackslash{}
0          1   -8437.673028         42    высшее             0
1          1   -4024.803754         36   среднее             1
2          0   -5623.422610         33   Среднее             1
3          3   -4124.747207         32   среднее             1
4          0  340266.072047         53   среднее             1
5          0    -926.185831         27    высшее             0
6          0   -2879.202052         43    высшее             0
7          0    -152.779569         50   СРЕДНЕЕ             1
8          2   -6929.865299         35    ВЫСШЕЕ             0
9          0   -2188.756445         41   среднее             1
10         2   -4171.483647         36    высшее             0

       family\_status  family\_status\_id gender income\_type  debt  \textbackslash{}
0    женат / замужем                 0      F   сотрудник     0
1    женат / замужем                 0      F   сотрудник     0
2    женат / замужем                 0      M   сотрудник     0
3    женат / замужем                 0      M   сотрудник     0
4   гражданский брак                 1      F   пенсионер     0
5   гражданский брак                 1      M   компаньон     0
6    женат / замужем                 0      F   компаньон     0
7    женат / замужем                 0      M   сотрудник     0
8   гражданский брак                 1      F   сотрудник     0
9    женат / замужем                 0      M   сотрудник     0
10   женат / замужем                 0      M   компаньон     0

     total\_income                     purpose
0   253875.639453               покупка жилья
1   112080.014102     приобретение автомобиля
2   145885.952297               покупка жилья
3   267628.550329  дополнительное образование
4   158616.077870             сыграть свадьбу
5   255763.565419               покупка жилья
6   240525.971920           операции с жильем
7   135823.934197                 образование
8    95856.832424       на проведение свадьбы
9   144425.938277     покупка жилья для семьи
10  113943.491460        покупка недвижимости
\end{Verbatim}
\end{tcolorbox}
        
    \hypertarget{ux432ux44bux432ux43eux434}{%
\subsubsection{Вывод}\label{ux432ux44bux432ux43eux434}}

    21,5 тычяча строк!!! Вручную просмотреть такую таблицу будет сложновато.
1. В столбцах days\_employed и total\_income есть нулевые значения 2.
Почему-то days\_employed - число с плавающей точкой 3. family\_status и
family\_status\_id, education и education\_id нужно посмотреть в чем
принципиальная разница и есть ли необходимость в этих парах, либо они
дублирующиеся 4. gender - стоит посмотреть уникальные значение. Больше
2х - будет чутка странно, в Беларуси не поймут(хотя этот пункт под
вопросом) 5. Прверить total\_income, у всех ли он положительный. Если
нет, сразу можно отправлять домой. Хотя, если будут люди с нулевым или
отрицательным 6. total\_income и days\_employed с одинаковым количеством
пропусков. Совпадение? Не думаю!

Задача поставленная нам - узнать ``влияет ли семейное положение и
количество детей клиента на факт погашения кредита в срок'' Поэтому есть
шанс что total\_income нам не пригодится)))

    Комментарий наставника

У меня имеется несколько комментариев по данному шагу: - желательно
выводить около 10 строчек таблицы. Меньше не рекомендуется, можно не
увидеть структуру данных; - считывание данных и импорт необходимых
библиотек лучше проводить в разных ячейках; - следует давать переменным
осознанные имена. Таблица про клиентов - название надо бы выбрать
соответствующее; - Подумай также о возможных причинах появления
пропусков, а также о том, являются ли они случайными или нет.

Первый взгляд на таблицу выполнен.

    DONE

    \begin{tcolorbox}[breakable, size=fbox, boxrule=1pt, pad at break*=1mm,colback=cellbackground, colframe=cellborder]
\prompt{In}{incolor}{ }{\hspace{4pt}}
\begin{Verbatim}[commandchars=\\\{\}]

\end{Verbatim}
\end{tcolorbox}

    Но все дело в том, что у нас нет ни имен, ни фамилий ни даже ID людей,
поэтому при удалении дубликатов мы можем удалить часть необходимых
данных:

    \hypertarget{ux448ux430ux433-2.-ux43fux440ux435ux434ux43eux431ux440ux430ux431ux43eux442ux43aux430-ux434ux430ux43dux43dux44bux445}{%
\subsubsection{Шаг 2. Предобработка
данных}\label{ux448ux430ux433-2.-ux43fux440ux435ux434ux43eux431ux440ux430ux431ux43eux442ux43aux430-ux434ux430ux43dux43dux44bux445}}

    \hypertarget{ux43eux431ux440ux430ux431ux43eux442ux43aux430-ux43fux440ux43eux43fux443ux441ux43aux43eux432}{%
\subsubsection{Обработка
пропусков}\label{ux43eux431ux440ux430ux431ux43eux442ux43aux430-ux43fux440ux43eux43fux443ux441ux43aux43eux432}}

    Комментарий наставника

Комментарии по работе делай в отдельных markdown ячейках.

    DONE

    Начнемс. Точнее проолжим, ведь я уже это прошел и не сохранил, просто у
кого-то не оттуда руки, поэтому буду переделывать! Смотрим
days\_employed

    \begin{tcolorbox}[breakable, size=fbox, boxrule=1pt, pad at break*=1mm,colback=cellbackground, colframe=cellborder]
\prompt{In}{incolor}{101}{\hspace{4pt}}
\begin{Verbatim}[commandchars=\\\{\}]
\PY{n+nb}{print}\PY{p}{(}\PY{n}{bank\PYZus{}clients}\PY{p}{[}\PY{n}{bank\PYZus{}clients}\PY{p}{[}\PY{l+s+s1}{\PYZsq{}}\PY{l+s+s1}{days\PYZus{}employed}\PY{l+s+s1}{\PYZsq{}}\PY{p}{]}\PY{o}{.}\PY{n}{isnull}\PY{p}{(}\PY{p}{)}\PY{p}{]}\PY{o}{.}\PY{n}{count}\PY{p}{(}\PY{p}{)}\PY{p}{)} 
\end{Verbatim}
\end{tcolorbox}

    \begin{Verbatim}[commandchars=\\\{\}]
children            2120
days\_employed          0
dob\_years           2120
education           2120
education\_id        2120
family\_status       2120
family\_status\_id    2120
gender              2120
income\_type         2120
debt                2120
total\_income           0
purpose             2120
dtype: int64
\end{Verbatim}

    Получается, что во всех пропущенных строках рабочих дней пропущены и
заработки Из этого следует вывод, что это неслучайные пропуски. Есть
вероятность что они работают, но ``вчерную'' Далее я проверил
вероятность того, что все 2174 человека являются 18-летними и не могут
иметь стаж

    \begin{tcolorbox}[breakable, size=fbox, boxrule=1pt, pad at break*=1mm,colback=cellbackground, colframe=cellborder]
\prompt{In}{incolor}{102}{\hspace{4pt}}
\begin{Verbatim}[commandchars=\\\{\}]
\PY{n}{unemployed} \PY{o}{=} \PY{n}{bank\PYZus{}clients}\PY{p}{[}\PY{n}{bank\PYZus{}clients}\PY{p}{[}\PY{l+s+s1}{\PYZsq{}}\PY{l+s+s1}{days\PYZus{}employed}\PY{l+s+s1}{\PYZsq{}}\PY{p}{]}\PY{o}{.}\PY{n}{isnull}\PY{p}{(}\PY{p}{)}\PY{p}{]}
\PY{n+nb}{print}\PY{p}{(}\PY{n}{unemployed}\PY{o}{.}\PY{n}{groupby}\PY{p}{(}\PY{l+s+s1}{\PYZsq{}}\PY{l+s+s1}{dob\PYZus{}years}\PY{l+s+s1}{\PYZsq{}}\PY{p}{)}\PY{p}{[}\PY{l+s+s1}{\PYZsq{}}\PY{l+s+s1}{dob\PYZus{}years}\PY{l+s+s1}{\PYZsq{}}\PY{p}{]}\PY{o}{.}\PY{n}{count}\PY{p}{(}\PY{p}{)}\PY{p}{)}
\end{Verbatim}
\end{tcolorbox}

    \begin{Verbatim}[commandchars=\\\{\}]
dob\_years
0     10
19     1
20     5
21    18
22    17
23    35
24    21
25    23
26    35
27    36
28    57
29    49
30    56
31    64
32    36
33    51
34    67
35    63
36    62
37    52
38    53
39    50
40    64
41    58
42    64
43    49
44    42
45    50
46    46
47    56
48    45
49    50
50    50
51    50
52    53
53    44
54    52
55    48
56    51
57    52
58    51
59    33
60    36
61    37
62    35
63    29
64    34
65    20
66    19
67    16
68     9
69     5
70     3
71     5
72     2
73     1
Name: dob\_years, dtype: int64
\end{Verbatim}

    10 человек из 21,5к младше 18 лет - меньше 0,05\%. Расслабились и
отпустили. Хотя есть шанс, что на общей выборке их будет больше Все ок.
данные идентичны. Проверяем детей

    \begin{tcolorbox}[breakable, size=fbox, boxrule=1pt, pad at break*=1mm,colback=cellbackground, colframe=cellborder]
\prompt{In}{incolor}{103}{\hspace{4pt}}
\begin{Verbatim}[commandchars=\\\{\}]
\PY{n+nb}{print}\PY{p}{(}\PY{n}{bank\PYZus{}clients}\PY{p}{[}\PY{l+s+s1}{\PYZsq{}}\PY{l+s+s1}{children}\PY{l+s+s1}{\PYZsq{}}\PY{p}{]}\PY{o}{.}\PY{n}{value\PYZus{}counts}\PY{p}{(}\PY{p}{)}\PY{p}{)}
\end{Verbatim}
\end{tcolorbox}

    \begin{Verbatim}[commandchars=\\\{\}]
 0     14107
 1      4809
 2      2052
 3       330
 20       76
-1        47
 4        41
 5         9
Name: children, dtype: int64
\end{Verbatim}

    ``-1'' ребенок настораживает. на мой взгляд тут возможны 3 варианта: -
Ошибка при заполнении (тогда просто переводим в 1) - ``смерть'', как бы
он был, а теперь нет (маловероятно) - Беременность (нужно проверить.
Если да, тоже переводить в 1)

Теоретически можно сразу перевести, но интересно найти закономерность

    \begin{tcolorbox}[breakable, size=fbox, boxrule=1pt, pad at break*=1mm,colback=cellbackground, colframe=cellborder]
\prompt{In}{incolor}{104}{\hspace{4pt}}
\begin{Verbatim}[commandchars=\\\{\}]
\PY{n+nb}{print}\PY{p}{(}\PY{n}{bank\PYZus{}clients}\PY{p}{[}\PY{n}{bank\PYZus{}clients}\PY{p}{[}\PY{l+s+s1}{\PYZsq{}}\PY{l+s+s1}{children}\PY{l+s+s1}{\PYZsq{}}\PY{p}{]} \PY{o}{==} \PY{o}{\PYZhy{}}\PY{l+m+mi}{1}\PY{p}{]}\PY{p}{)}
\end{Verbatim}
\end{tcolorbox}

    \begin{Verbatim}[commandchars=\\\{\}]
       children  days\_employed  dob\_years            education  education\_id  \textbackslash{}
291          -1   -4417.703588         46              среднее             1
705          -1    -902.084528         50              среднее             1
742          -1   -3174.456205         57              среднее             1
800          -1  349987.852217         54              среднее             1
941          -1            NaN         57              Среднее             1
1363         -1   -1195.264956         55              СРЕДНЕЕ             1
1929         -1   -1461.303336         38              среднее             1
2073         -1   -2539.761232         42              среднее             1
3813         -1   -3045.290443         26              Среднее             1
4199         -1    -901.101738         41              среднее             1
4400         -1  398001.302888         64              СРЕДНЕЕ             1
4540         -1   -1811.899756         32              среднее             1
5269         -1   -1143.485347         46              среднее             1
6009         -1   -1361.258696         46               высшее             0
6381         -1  370215.476226         48              среднее             1
7186         -1   -5928.202068         34              СРЕДНЕЕ             1
7274         -1    -526.318451         51              среднее             1
7611         -1            NaN         35              среднее             1
7681         -1   -3237.360455         53              среднее             1
8224         -1   -1803.441074         54               высшее             0
8246         -1   -3113.998449         31              среднее             1
8556         -1    -195.479496         31              среднее             1
9557         -1   -2896.629686         34               высшее             0
9566         -1   -2710.419901         28               высшее             0
10168        -1   -1743.604011         37              среднее             1
10356        -1  345774.125957         63               ВЫСШЕЕ             0
10906        -1  340499.039342         50              среднее             1
11096        -1   -3438.463024         59              среднее             1
11255        -1   -1048.782203         30               высшее             0
11604        -1    -370.827130         27              среднее             1
12290        -1   -8493.101252         61  неоконченное высшее             2
13766        -1            NaN         42              среднее             1
14335        -1    -268.337037         23              среднее             1
15122        -1   -9851.184337         44              среднее             1
15408        -1   -3614.220232         40               высшее             0
16102        -1    -457.861760         33              среднее             1
16237        -1   -2802.218127         40               высшее             0
17064        -1   -2809.693200         34              среднее             1
17397        -1    -895.379738         37              среднее             1
17625        -1   -4571.957475         41              среднее             1
18185        -1   -3575.215641         33              среднее             1
19059        -1    -617.246968         38              среднее             1
19323        -1    -372.034749         43              СРЕДНЕЕ             1
19373        -1  350340.760224         28              среднее             1
20344        -1  355157.107212         69              среднее             1
20667        -1    -661.822321         32               высшее             0
21088        -1   -1422.668059         44              среднее             1

               family\_status  family\_status\_id gender  income\_type  debt  \textbackslash{}
291         гражданский брак                 1      F    сотрудник     0
705          женат / замужем                 0      F  госслужащий     0
742          женат / замужем                 0      F    сотрудник     0
800    Не женат / не замужем                 4      F    пенсионер     0
941          женат / замужем                 0      F    пенсионер     0
1363         женат / замужем                 0      F    компаньон     0
1929   Не женат / не замужем                 4      M    сотрудник     0
2073               в разводе                 3      F    компаньон     0
3813        гражданский брак                 1      F  госслужащий     0
4199         женат / замужем                 0      F  госслужащий     0
4400         женат / замужем                 0      F    пенсионер     0
4540         женат / замужем                 0      F    сотрудник     0
5269          вдовец / вдова                 2      F    сотрудник     0
6009         женат / замужем                 0      F    сотрудник     0
6381          вдовец / вдова                 2      F    пенсионер     0
7186         женат / замужем                 0      M    сотрудник     0
7274         женат / замужем                 0      F    сотрудник     0
7611         женат / замужем                 0      M    сотрудник     0
7681          вдовец / вдова                 2      F    сотрудник     0
8224         женат / замужем                 0      F    компаньон     0
8246        гражданский брак                 1      F    сотрудник     0
8556         женат / замужем                 0      F    сотрудник     0
9557         женат / замужем                 0      M    сотрудник     0
9566         женат / замужем                 0      M  госслужащий     0
10168        женат / замужем                 0      M    сотрудник     0
10356         вдовец / вдова                 2      F    пенсионер     0
10906        женат / замужем                 0      M    пенсионер     0
11096        женат / замужем                 0      F    сотрудник     0
11255       гражданский брак                 1      F    компаньон     0
11604        женат / замужем                 0      F    компаньон     0
12290        женат / замужем                 0      M    сотрудник     0
13766  Не женат / не замужем                 4      M    компаньон     0
14335              в разводе                 3      F    компаньон     0
15122              в разводе                 3      F    сотрудник     0
15408       гражданский брак                 1      F    сотрудник     0
16102        женат / замужем                 0      F    сотрудник     1
16237        женат / замужем                 0      M    сотрудник     0
17064        женат / замужем                 0      F    сотрудник     0
17397        женат / замужем                 0      F    компаньон     0
17625        женат / замужем                 0      F    сотрудник     0
18185        женат / замужем                 0      F    сотрудник     0
19059  Не женат / не замужем                 4      M    сотрудник     0
19323        женат / замужем                 0      M    сотрудник     0
19373              в разводе                 3      F    пенсионер     0
20344  Не женат / не замужем                 4      F    пенсионер     0
20667        женат / замужем                 0      F    сотрудник     0
21088        женат / замужем                 0      F    компаньон     0

        total\_income                                 purpose
291    102816.346412                  профильное образование
705    137882.899271                 приобретение автомобиля
742     64268.044444              дополнительное образование
800     86293.724153              дополнительное образование
941              NaN            на покупку своего автомобиля
1363    69550.699692                  профильное образование
1929   109121.569013                           покупка жилья
2073   162638.609373                           покупка жилья
3813   131892.785435                   на проведение свадьбы
4199   226375.766751         операции со своей недвижимостью
4400   163264.062064                    покупка недвижимости
4540   160544.718446                             образование
5269   278708.018820                    покупка недвижимости
6009   143008.454914  строительство собственной недвижимости
6381    36052.447435                             образование
7186   184315.121979                 приобретение автомобиля
7274   146928.769439   операции с коммерческой недвижимостью
7611             NaN                             образование
7681   159676.174958        строительство жилой недвижимости
8224   138809.082930                    покупка недвижимости
8246    54412.056005                         свой автомобиль
8556   145577.881522                            ремонт жилью
9557   126754.656529                операции с недвижимостью
9566   303137.161001   получение дополнительного образования
10168  310367.509001                   получение образования
10356  170762.751325        строительство жилой недвижимости
10906  170762.773162                    покупка своего жилья
11096  219874.012345                       операции с жильем
11255  321603.700207                   на покупку автомобиля
11604  164591.260338                 покупка жилья для сдачи
12290  315006.182056                            недвижимость
13766            NaN                              автомобиль
14335   92257.579312              покупка жилой недвижимости
15122  110990.810581                         свой автомобиль
15408   98127.537462                         сыграть свадьбу
16102  149641.079451                              автомобиль
16237  111984.472021                           покупка жилья
17064  182543.890127      на покупку подержанного автомобиля
17397  214814.018780                                   жилье
17625  122105.415823         операции со своей недвижимостью
18185  128362.023879                      высшее образование
19059  122205.497527        строительство жилой недвижимости
19323  155588.766795        сделка с подержанным автомобилем
19373   52872.993654                              автомобили
20344  116521.045858                операции с недвижимостью
20667  137405.384917              покупка жилой недвижимости
21088  169562.091999         операции со своей недвижимостью
\end{Verbatim}

    Есть как М, так и Ж, значит это не беременные. Отправляем всех -1 детей
к однодетным

    \begin{tcolorbox}[breakable, size=fbox, boxrule=1pt, pad at break*=1mm,colback=cellbackground, colframe=cellborder]
\prompt{In}{incolor}{105}{\hspace{4pt}}
\begin{Verbatim}[commandchars=\\\{\}]
\PY{n}{bank\PYZus{}clients}\PY{o}{.}\PY{n}{loc}\PY{p}{[}\PY{n}{bank\PYZus{}clients}\PY{p}{[}\PY{l+s+s1}{\PYZsq{}}\PY{l+s+s1}{children}\PY{l+s+s1}{\PYZsq{}}\PY{p}{]} \PY{o}{==} \PY{o}{\PYZhy{}}\PY{l+m+mi}{1}\PY{p}{,} \PY{l+s+s1}{\PYZsq{}}\PY{l+s+s1}{children}\PY{l+s+s1}{\PYZsq{}}\PY{p}{]} \PY{o}{=} \PY{l+m+mi}{1}
\PY{n+nb}{print}\PY{p}{(}\PY{n}{bank\PYZus{}clients}\PY{p}{[}\PY{l+s+s1}{\PYZsq{}}\PY{l+s+s1}{children}\PY{l+s+s1}{\PYZsq{}}\PY{p}{]}\PY{o}{.}\PY{n}{value\PYZus{}counts}\PY{p}{(}\PY{p}{)}\PY{p}{)}
\end{Verbatim}
\end{tcolorbox}

    \begin{Verbatim}[commandchars=\\\{\}]
0     14107
1      4856
2      2052
3       330
20       76
4        41
5         9
Name: children, dtype: int64
\end{Verbatim}

    Вот теперь это похоже на правду. А что за 76 человек с 20 детьми?

    \begin{tcolorbox}[breakable, size=fbox, boxrule=1pt, pad at break*=1mm,colback=cellbackground, colframe=cellborder]
\prompt{In}{incolor}{106}{\hspace{4pt}}
\begin{Verbatim}[commandchars=\\\{\}]
\PY{n}{children\PYZus{}20} \PY{o}{=} \PY{n}{bank\PYZus{}clients}\PY{p}{[}\PY{n}{bank\PYZus{}clients}\PY{p}{[}\PY{l+s+s1}{\PYZsq{}}\PY{l+s+s1}{children}\PY{l+s+s1}{\PYZsq{}}\PY{p}{]} \PY{o}{==} \PY{l+m+mi}{20}\PY{p}{]}
\PY{n+nb}{print}\PY{p}{(}\PY{n}{children\PYZus{}20}\PY{o}{.}\PY{n}{head}\PY{p}{(}\PY{l+m+mi}{15}\PY{p}{)}\PY{p}{)}
\PY{n+nb}{print}\PY{p}{(}\PY{l+s+s1}{\PYZsq{}}\PY{l+s+s1}{\PYZsq{}}\PY{p}{)}
\PY{n+nb}{print}\PY{p}{(}\PY{l+s+s1}{\PYZsq{}}\PY{l+s+s1}{\PYZsq{}}\PY{p}{)}
\PY{n+nb}{print}\PY{p}{(}\PY{n}{bank\PYZus{}clients}\PY{o}{.}\PY{n}{info}\PY{p}{(}\PY{p}{)}\PY{p}{)}
\end{Verbatim}
\end{tcolorbox}

    \begin{Verbatim}[commandchars=\\\{\}]
      children  days\_employed  dob\_years education  education\_id  \textbackslash{}
606         20    -880.221113         21   среднее             1
720         20    -855.595512         44   среднее             1
1074        20   -3310.411598         56   среднее             1
2510        20   -2714.161249         59    высшее             0
2940        20   -2161.591519          0   среднее             1
3301        20            NaN         35   среднее             1
3395        20            NaN         56    высшее             0
3670        20    -913.161503         23   среднее             1
3696        20   -2907.910616         40   среднее             1
3734        20    -805.044438         26    высшее             0
3876        20   -8190.644409         45   среднее             1
5017        20    -231.783475         37   среднее             1
5312        20   -2047.754733         24   среднее             1
5346        20  361744.836360         64   среднее             1
5359        20  355898.021316         69   среднее             1

              family\_status  family\_status\_id gender  income\_type  debt  \textbackslash{}
606         женат / замужем                 0      M    компаньон     0
720         женат / замужем                 0      F    компаньон     0
1074        женат / замужем                 0      F    сотрудник     1
2510         вдовец / вдова                 2      F    сотрудник     0
2940        женат / замужем                 0      F    сотрудник     0
3301  Не женат / не замужем                 4      F  госслужащий     0
3395        женат / замужем                 0      F    компаньон     0
3670  Не женат / не замужем                 4      F    сотрудник     0
3696       гражданский брак                 1      M    сотрудник     0
3734  Не женат / не замужем                 4      M    сотрудник     0
3876       гражданский брак                 1      F    сотрудник     0
5017       гражданский брак                 1      M    компаньон     0
5312        женат / замужем                 0      F    сотрудник     0
5346         вдовец / вдова                 2      F    пенсионер     0
5359        женат / замужем                 0      M    пенсионер     0

       total\_income                                purpose
606   145334.865002                          покупка жилья
720   112998.738649                   покупка недвижимости
1074  229518.537004                  получение образования
2510  264474.835577  операции с коммерческой недвижимостью
2940  199739.941398                  на покупку автомобиля
3301            NaN                 профильное образование
3395            NaN                     высшее образование
3670  101255.492076     на покупку подержанного автомобиля
3696  115380.694664     на покупку подержанного автомобиля
3734  137200.646181                           ремонт жилью
3876  312270.145176       сделка с подержанным автомобилем
5017  260203.344629                        сыграть свадьбу
5312  100415.236833      покупка коммерческой недвижимости
5346   81926.818467                      операции с жильем
5359  138579.082863                             автомобили


<class 'pandas.core.frame.DataFrame'>
RangeIndex: 21471 entries, 0 to 21470
Data columns (total 12 columns):
children            21471 non-null int64
days\_employed       19351 non-null float64
dob\_years           21471 non-null int64
education           21471 non-null object
education\_id        21471 non-null int64
family\_status       21471 non-null object
family\_status\_id    21471 non-null int64
gender              21471 non-null object
income\_type         21471 non-null object
debt                21471 non-null int64
total\_income        19351 non-null float64
purpose             21471 non-null object
dtypes: float64(2), int64(5), object(5)
memory usage: 2.0+ MB
None
\end{Verbatim}

    Теоретически, если бы все они брали кредиты на недвижимость и были
старше 40 лет (это фантазия), можно было бы предположить, Что случилось
невероятное и они многодетные от слова совсем.

    \begin{tcolorbox}[breakable, size=fbox, boxrule=1pt, pad at break*=1mm,colback=cellbackground, colframe=cellborder]
\prompt{In}{incolor}{107}{\hspace{4pt}}
\begin{Verbatim}[commandchars=\\\{\}]
\PY{n+nb}{print}\PY{p}{(}\PY{n}{children\PYZus{}20}\PY{o}{.}\PY{n}{groupby}\PY{p}{(}\PY{l+s+s1}{\PYZsq{}}\PY{l+s+s1}{dob\PYZus{}years}\PY{l+s+s1}{\PYZsq{}}\PY{p}{)}\PY{p}{[}\PY{l+s+s1}{\PYZsq{}}\PY{l+s+s1}{dob\PYZus{}years}\PY{l+s+s1}{\PYZsq{}}\PY{p}{]}\PY{o}{.}\PY{n}{count}\PY{p}{(}\PY{p}{)}\PY{p}{)}
\end{Verbatim}
\end{tcolorbox}

    \begin{Verbatim}[commandchars=\\\{\}]
dob\_years
0     1
21    1
23    1
24    1
25    1
26    1
27    2
29    2
30    3
31    2
32    2
33    2
34    3
35    2
36    2
37    4
38    1
39    1
40    4
41    2
42    3
43    2
44    2
45    3
46    3
48    1
49    3
50    3
51    1
52    1
53    1
54    1
55    1
56    5
57    1
59    2
60    1
61    1
62    1
64    1
69    1
Name: dob\_years, dtype: int64
\end{Verbatim}

    ну или 76 человек - только мужчины, которые жизнь положили на увеличение
потомков

    \begin{tcolorbox}[breakable, size=fbox, boxrule=1pt, pad at break*=1mm,colback=cellbackground, colframe=cellborder]
\prompt{In}{incolor}{108}{\hspace{4pt}}
\begin{Verbatim}[commandchars=\\\{\}]
\PY{n+nb}{print}\PY{p}{(}\PY{n}{children\PYZus{}20}\PY{o}{.}\PY{n}{groupby}\PY{p}{(}\PY{l+s+s1}{\PYZsq{}}\PY{l+s+s1}{gender}\PY{l+s+s1}{\PYZsq{}}\PY{p}{)}\PY{p}{[}\PY{l+s+s1}{\PYZsq{}}\PY{l+s+s1}{gender}\PY{l+s+s1}{\PYZsq{}}\PY{p}{]}\PY{o}{.}\PY{n}{count}\PY{p}{(}\PY{p}{)}\PY{p}{)}
\end{Verbatim}
\end{tcolorbox}

    \begin{Verbatim}[commandchars=\\\{\}]
gender
F    47
M    29
Name: gender, dtype: int64
\end{Verbatim}

    Ни то ни другое истиной не является и т.к. в выборке с увеличением детей
количество людей снижается, смею предположить, что ``20'' является
ошибкой, если не преднамеренным исправлением Закидываем всех с 20 детьми
к 2м детям, только нужно проверить, не сильно ли изменит это выборку с
долгами

    \begin{tcolorbox}[breakable, size=fbox, boxrule=1pt, pad at break*=1mm,colback=cellbackground, colframe=cellborder]
\prompt{In}{incolor}{109}{\hspace{4pt}}
\begin{Verbatim}[commandchars=\\\{\}]
\PY{n+nb}{print}\PY{p}{(}\PY{n}{bank\PYZus{}clients}\PY{o}{.}\PY{n}{groupby}\PY{p}{(}\PY{l+s+s1}{\PYZsq{}}\PY{l+s+s1}{children}\PY{l+s+s1}{\PYZsq{}}\PY{p}{)}\PY{p}{[}\PY{l+s+s1}{\PYZsq{}}\PY{l+s+s1}{debt}\PY{l+s+s1}{\PYZsq{}}\PY{p}{]}\PY{o}{.}\PY{n}{value\PYZus{}counts}\PY{p}{(}\PY{p}{)}\PY{p}{)}
\end{Verbatim}
\end{tcolorbox}

    \begin{Verbatim}[commandchars=\\\{\}]
children  debt
0         0       13044
          1        1063
1         0        4411
          1         445
2         0        1858
          1         194
3         0         303
          1          27
4         0          37
          1           4
5         0           9
20        0          68
          1           8
Name: debt, dtype: int64
\end{Verbatim}

    Повальных долгов нет, и 76 человек из всей выборки - меньше процента.
Отправляем их к двудетным

    \begin{tcolorbox}[breakable, size=fbox, boxrule=1pt, pad at break*=1mm,colback=cellbackground, colframe=cellborder]
\prompt{In}{incolor}{110}{\hspace{4pt}}
\begin{Verbatim}[commandchars=\\\{\}]
\PY{n}{bank\PYZus{}clients}\PY{o}{.}\PY{n}{loc}\PY{p}{[}\PY{n}{bank\PYZus{}clients}\PY{p}{[}\PY{l+s+s1}{\PYZsq{}}\PY{l+s+s1}{children}\PY{l+s+s1}{\PYZsq{}}\PY{p}{]} \PY{o}{==} \PY{l+m+mi}{20}\PY{p}{,} \PY{l+s+s1}{\PYZsq{}}\PY{l+s+s1}{children}\PY{l+s+s1}{\PYZsq{}}\PY{p}{]} \PY{o}{=} \PY{l+m+mi}{2}

\PY{n+nb}{print}\PY{p}{(}\PY{n}{bank\PYZus{}clients}\PY{p}{[}\PY{l+s+s1}{\PYZsq{}}\PY{l+s+s1}{children}\PY{l+s+s1}{\PYZsq{}}\PY{p}{]}\PY{o}{.}\PY{n}{value\PYZus{}counts}\PY{p}{(}\PY{p}{)}\PY{p}{)}
\end{Verbatim}
\end{tcolorbox}

    \begin{Verbatim}[commandchars=\\\{\}]
0    14107
1     4856
2     2128
3      330
4       41
5        9
Name: children, dtype: int64
\end{Verbatim}

    Сказали что так делать нельзя

Огонек! Уберем пропуски в days\_employed и total\_income. Т.к. нули в
этм случае равняются пропускам, заполним их нулями

bank\_clients{[}`days\_employed'{]} =
bank\_clients{[}`days\_employed'{]}.fillna(0)

bank\_clients{[}`total\_income'{]} =
bank\_clients{[}`total\_income'{]}.fillna(0)

print(bank\_clients.info())

    Найдем все уникальные ``профессии''

    \begin{tcolorbox}[breakable, size=fbox, boxrule=1pt, pad at break*=1mm,colback=cellbackground, colframe=cellborder]
\prompt{In}{incolor}{111}{\hspace{4pt}}
\begin{Verbatim}[commandchars=\\\{\}]
\PY{n+nb}{print}\PY{p}{(}\PY{n}{bank\PYZus{}clients}\PY{p}{[}\PY{l+s+s1}{\PYZsq{}}\PY{l+s+s1}{income\PYZus{}type}\PY{l+s+s1}{\PYZsq{}}\PY{p}{]}\PY{o}{.}\PY{n}{unique}\PY{p}{(}\PY{p}{)}\PY{p}{)}
\end{Verbatim}
\end{tcolorbox}

    \begin{Verbatim}[commandchars=\\\{\}]
['сотрудник' 'пенсионер' 'компаньон' 'госслужащий' 'безработный'
 'предприниматель' 'студент' 'в декрете']
\end{Verbatim}

    Тут без повторов, и на том спасибо. Давайте узнаем сколько кто не
получает зарплату по нашим данным

    \begin{tcolorbox}[breakable, size=fbox, boxrule=1pt, pad at break*=1mm,colback=cellbackground, colframe=cellborder]
\prompt{In}{incolor}{112}{\hspace{4pt}}
\begin{Verbatim}[commandchars=\\\{\}]
\PY{n}{unemployed\PYZus{}categories} \PY{o}{=} \PY{n}{bank\PYZus{}clients}\PY{p}{[}\PY{n}{bank\PYZus{}clients}\PY{p}{[}\PY{l+s+s1}{\PYZsq{}}\PY{l+s+s1}{total\PYZus{}income}\PY{l+s+s1}{\PYZsq{}}\PY{p}{]}\PY{o}{.}\PY{n}{isnull}\PY{p}{(}\PY{p}{)}\PY{p}{]}
\PY{n+nb}{print}\PY{p}{(}\PY{n}{unemployed\PYZus{}categories}\PY{p}{[}\PY{l+s+s1}{\PYZsq{}}\PY{l+s+s1}{income\PYZus{}type}\PY{l+s+s1}{\PYZsq{}}\PY{p}{]}\PY{o}{.}\PY{n}{unique}\PY{p}{(}\PY{p}{)}\PY{p}{)}
\end{Verbatim}
\end{tcolorbox}

    \begin{Verbatim}[commandchars=\\\{\}]
['пенсионер' 'госслужащий' 'компаньон' 'сотрудник' 'предприниматель']
\end{Verbatim}

    Типов всего 5, глянем, кого больше

    \begin{tcolorbox}[breakable, size=fbox, boxrule=1pt, pad at break*=1mm,colback=cellbackground, colframe=cellborder]
\prompt{In}{incolor}{113}{\hspace{4pt}}
\begin{Verbatim}[commandchars=\\\{\}]
\PY{n+nb}{print}\PY{p}{(}\PY{n}{unemployed\PYZus{}categories}\PY{o}{.}\PY{n}{groupby}\PY{p}{(}\PY{l+s+s1}{\PYZsq{}}\PY{l+s+s1}{income\PYZus{}type}\PY{l+s+s1}{\PYZsq{}}\PY{p}{)}\PY{p}{[}\PY{l+s+s1}{\PYZsq{}}\PY{l+s+s1}{income\PYZus{}type}\PY{l+s+s1}{\PYZsq{}}\PY{p}{]}\PY{o}{.}\PY{n}{count}\PY{p}{(}\PY{p}{)}\PY{p}{)}
\end{Verbatim}
\end{tcolorbox}

    \begin{Verbatim}[commandchars=\\\{\}]
income\_type
госслужащий         145
компаньон           503
пенсионер           394
предприниматель       1
сотрудник          1077
Name: income\_type, dtype: int64
\end{Verbatim}

    Начнем избавляться от пропусков с самой многочисленной группы -
``сотрудники'' Посчитаем средний заработок всех ``сотрудников'', что
есть у нас в таблице и заполним пропуски медианой Учтем их

    \begin{tcolorbox}[breakable, size=fbox, boxrule=1pt, pad at break*=1mm,colback=cellbackground, colframe=cellborder]
\prompt{In}{incolor}{114}{\hspace{4pt}}
\begin{Verbatim}[commandchars=\\\{\}]
\PY{n}{employe\PYZus{}m} \PY{o}{=} \PY{n}{bank\PYZus{}clients}\PY{p}{[}\PY{p}{(}\PY{n}{bank\PYZus{}clients}\PY{p}{[}\PY{l+s+s1}{\PYZsq{}}\PY{l+s+s1}{income\PYZus{}type}\PY{l+s+s1}{\PYZsq{}}\PY{p}{]} \PY{o}{==} \PY{l+s+s1}{\PYZsq{}}\PY{l+s+s1}{сотрудник}\PY{l+s+s1}{\PYZsq{}}\PY{p}{)} 
                         \PY{o}{\PYZam{}} \PY{p}{(}\PY{n}{bank\PYZus{}clients}\PY{p}{[}\PY{l+s+s1}{\PYZsq{}}\PY{l+s+s1}{gender}\PY{l+s+s1}{\PYZsq{}}\PY{p}{]} \PY{o}{==} \PY{l+s+s1}{\PYZsq{}}\PY{l+s+s1}{M}\PY{l+s+s1}{\PYZsq{}}\PY{p}{)}\PY{p}{]}\PY{p}{[}\PY{l+s+s1}{\PYZsq{}}\PY{l+s+s1}{total\PYZus{}income}\PY{l+s+s1}{\PYZsq{}}\PY{p}{]}\PY{o}{.}\PY{n}{dropna}\PY{p}{(}\PY{p}{)}\PY{o}{.}\PY{n}{median}\PY{p}{(}\PY{p}{)}

\PY{n+nb}{print}\PY{p}{(}\PY{l+s+s1}{\PYZsq{}}\PY{l+s+s1}{Медиана для }\PY{l+s+s1}{\PYZdq{}}\PY{l+s+s1}{сотрудник мужчина}\PY{l+s+s1}{\PYZdq{}}\PY{l+s+s1}{ }\PY{l+s+si}{\PYZob{}:.1f\PYZcb{}}\PY{l+s+s1}{\PYZsq{}}\PY{o}{.}\PY{n}{format}\PY{p}{(}\PY{n}{employe\PYZus{}m}\PY{p}{)}\PY{p}{)}

\PY{n}{employe\PYZus{}f} \PY{o}{=} \PY{n}{bank\PYZus{}clients}\PY{p}{[}\PY{p}{(}\PY{n}{bank\PYZus{}clients}\PY{p}{[}\PY{l+s+s1}{\PYZsq{}}\PY{l+s+s1}{income\PYZus{}type}\PY{l+s+s1}{\PYZsq{}}\PY{p}{]} \PY{o}{==} \PY{l+s+s1}{\PYZsq{}}\PY{l+s+s1}{сотрудник}\PY{l+s+s1}{\PYZsq{}}\PY{p}{)} 
                         \PY{o}{\PYZam{}} \PY{p}{(}\PY{n}{bank\PYZus{}clients}\PY{p}{[}\PY{l+s+s1}{\PYZsq{}}\PY{l+s+s1}{gender}\PY{l+s+s1}{\PYZsq{}}\PY{p}{]} \PY{o}{==} \PY{l+s+s1}{\PYZsq{}}\PY{l+s+s1}{F}\PY{l+s+s1}{\PYZsq{}}\PY{p}{)}\PY{p}{]}\PY{p}{[}\PY{l+s+s1}{\PYZsq{}}\PY{l+s+s1}{total\PYZus{}income}\PY{l+s+s1}{\PYZsq{}}\PY{p}{]}\PY{o}{.}\PY{n}{dropna}\PY{p}{(}\PY{p}{)}\PY{o}{.}\PY{n}{median}\PY{p}{(}\PY{p}{)}

\PY{n+nb}{print}\PY{p}{(}\PY{l+s+s1}{\PYZsq{}}\PY{l+s+s1}{Медиана для }\PY{l+s+s1}{\PYZdq{}}\PY{l+s+s1}{сотрудник женщина}\PY{l+s+s1}{\PYZdq{}}\PY{l+s+s1}{ }\PY{l+s+si}{\PYZob{}:.1f\PYZcb{}}\PY{l+s+s1}{\PYZsq{}}\PY{o}{.}\PY{n}{format}\PY{p}{(}\PY{n}{employe\PYZus{}f}\PY{p}{)}\PY{p}{)}
\end{Verbatim}
\end{tcolorbox}

    \begin{Verbatim}[commandchars=\\\{\}]
Медиана для "сотрудник мужчина" 162161.2
Медиана для "сотрудник женщина" 130615.6
\end{Verbatim}

    По такой же аналогии просчитаем все остальные категории

    \begin{tcolorbox}[breakable, size=fbox, boxrule=1pt, pad at break*=1mm,colback=cellbackground, colframe=cellborder]
\prompt{In}{incolor}{115}{\hspace{4pt}}
\begin{Verbatim}[commandchars=\\\{\}]
\PY{n}{civil\PYZus{}servant\PYZus{}m} \PY{o}{=} \PY{n}{bank\PYZus{}clients}\PY{p}{[}\PY{p}{(}\PY{n}{bank\PYZus{}clients}\PY{p}{[}\PY{l+s+s1}{\PYZsq{}}\PY{l+s+s1}{income\PYZus{}type}\PY{l+s+s1}{\PYZsq{}}\PY{p}{]} \PY{o}{==} \PY{l+s+s1}{\PYZsq{}}\PY{l+s+s1}{госслужащий}\PY{l+s+s1}{\PYZsq{}}\PY{p}{)} 
                         \PY{o}{\PYZam{}} \PY{p}{(}\PY{n}{bank\PYZus{}clients}\PY{p}{[}\PY{l+s+s1}{\PYZsq{}}\PY{l+s+s1}{gender}\PY{l+s+s1}{\PYZsq{}}\PY{p}{]} \PY{o}{==} \PY{l+s+s1}{\PYZsq{}}\PY{l+s+s1}{M}\PY{l+s+s1}{\PYZsq{}}\PY{p}{)}\PY{p}{]}\PY{p}{[}\PY{l+s+s1}{\PYZsq{}}\PY{l+s+s1}{total\PYZus{}income}\PY{l+s+s1}{\PYZsq{}}\PY{p}{]}\PY{o}{.}\PY{n}{dropna}\PY{p}{(}\PY{p}{)}\PY{o}{.}\PY{n}{median}\PY{p}{(}\PY{p}{)}
\PY{n+nb}{print}\PY{p}{(}\PY{l+s+s1}{\PYZsq{}}\PY{l+s+s1}{Медиана для }\PY{l+s+s1}{\PYZdq{}}\PY{l+s+s1}{госслужащий мужчина}\PY{l+s+s1}{\PYZdq{}}\PY{l+s+s1}{ }\PY{l+s+si}{\PYZob{}:.1f\PYZcb{}}\PY{l+s+s1}{\PYZsq{}}\PY{o}{.}\PY{n}{format}\PY{p}{(}\PY{n}{civil\PYZus{}servant\PYZus{}m}\PY{p}{)}\PY{p}{)}
\PY{n}{civil\PYZus{}servant\PYZus{}f} \PY{o}{=} \PY{n}{bank\PYZus{}clients}\PY{p}{[}\PY{p}{(}\PY{n}{bank\PYZus{}clients}\PY{p}{[}\PY{l+s+s1}{\PYZsq{}}\PY{l+s+s1}{income\PYZus{}type}\PY{l+s+s1}{\PYZsq{}}\PY{p}{]} \PY{o}{==} \PY{l+s+s1}{\PYZsq{}}\PY{l+s+s1}{госслужащий}\PY{l+s+s1}{\PYZsq{}}\PY{p}{)} 
                         \PY{o}{\PYZam{}} \PY{p}{(}\PY{n}{bank\PYZus{}clients}\PY{p}{[}\PY{l+s+s1}{\PYZsq{}}\PY{l+s+s1}{gender}\PY{l+s+s1}{\PYZsq{}}\PY{p}{]} \PY{o}{==} \PY{l+s+s1}{\PYZsq{}}\PY{l+s+s1}{F}\PY{l+s+s1}{\PYZsq{}}\PY{p}{)}\PY{p}{]}\PY{p}{[}\PY{l+s+s1}{\PYZsq{}}\PY{l+s+s1}{total\PYZus{}income}\PY{l+s+s1}{\PYZsq{}}\PY{p}{]}\PY{o}{.}\PY{n}{dropna}\PY{p}{(}\PY{p}{)}\PY{o}{.}\PY{n}{median}\PY{p}{(}\PY{p}{)}
\PY{n+nb}{print}\PY{p}{(}\PY{l+s+s1}{\PYZsq{}}\PY{l+s+s1}{Медиана для }\PY{l+s+s1}{\PYZdq{}}\PY{l+s+s1}{госслужащий женщина}\PY{l+s+s1}{\PYZdq{}}\PY{l+s+s1}{ }\PY{l+s+si}{\PYZob{}:.1f\PYZcb{}}\PY{l+s+s1}{\PYZsq{}}\PY{o}{.}\PY{n}{format}\PY{p}{(}\PY{n}{civil\PYZus{}servant\PYZus{}f}\PY{p}{)}\PY{p}{)}

\PY{n}{companion\PYZus{}m} \PY{o}{=} \PY{n}{bank\PYZus{}clients}\PY{p}{[}\PY{p}{(}\PY{n}{bank\PYZus{}clients}\PY{p}{[}\PY{l+s+s1}{\PYZsq{}}\PY{l+s+s1}{income\PYZus{}type}\PY{l+s+s1}{\PYZsq{}}\PY{p}{]} \PY{o}{==} \PY{l+s+s1}{\PYZsq{}}\PY{l+s+s1}{компаньон}\PY{l+s+s1}{\PYZsq{}}\PY{p}{)} 
                         \PY{o}{\PYZam{}} \PY{p}{(}\PY{n}{bank\PYZus{}clients}\PY{p}{[}\PY{l+s+s1}{\PYZsq{}}\PY{l+s+s1}{gender}\PY{l+s+s1}{\PYZsq{}}\PY{p}{]} \PY{o}{==} \PY{l+s+s1}{\PYZsq{}}\PY{l+s+s1}{M}\PY{l+s+s1}{\PYZsq{}}\PY{p}{)}\PY{p}{]}\PY{p}{[}\PY{l+s+s1}{\PYZsq{}}\PY{l+s+s1}{total\PYZus{}income}\PY{l+s+s1}{\PYZsq{}}\PY{p}{]}\PY{o}{.}\PY{n}{dropna}\PY{p}{(}\PY{p}{)}\PY{o}{.}\PY{n}{median}\PY{p}{(}\PY{p}{)}
\PY{n+nb}{print}\PY{p}{(}\PY{l+s+s1}{\PYZsq{}}\PY{l+s+s1}{Медиана для }\PY{l+s+s1}{\PYZdq{}}\PY{l+s+s1}{компаньон мужчина}\PY{l+s+s1}{\PYZdq{}}\PY{l+s+s1}{ }\PY{l+s+si}{\PYZob{}:.1f\PYZcb{}}\PY{l+s+s1}{\PYZsq{}}\PY{o}{.}\PY{n}{format}\PY{p}{(}\PY{n}{companion\PYZus{}m}\PY{p}{)}\PY{p}{)}
\PY{n}{companion\PYZus{}f} \PY{o}{=} \PY{n}{bank\PYZus{}clients}\PY{p}{[}\PY{p}{(}\PY{n}{bank\PYZus{}clients}\PY{p}{[}\PY{l+s+s1}{\PYZsq{}}\PY{l+s+s1}{income\PYZus{}type}\PY{l+s+s1}{\PYZsq{}}\PY{p}{]} \PY{o}{==} \PY{l+s+s1}{\PYZsq{}}\PY{l+s+s1}{компаньон}\PY{l+s+s1}{\PYZsq{}}\PY{p}{)} 
                         \PY{o}{\PYZam{}} \PY{p}{(}\PY{n}{bank\PYZus{}clients}\PY{p}{[}\PY{l+s+s1}{\PYZsq{}}\PY{l+s+s1}{gender}\PY{l+s+s1}{\PYZsq{}}\PY{p}{]} \PY{o}{==} \PY{l+s+s1}{\PYZsq{}}\PY{l+s+s1}{F}\PY{l+s+s1}{\PYZsq{}}\PY{p}{)}\PY{p}{]}\PY{p}{[}\PY{l+s+s1}{\PYZsq{}}\PY{l+s+s1}{total\PYZus{}income}\PY{l+s+s1}{\PYZsq{}}\PY{p}{]}\PY{o}{.}\PY{n}{dropna}\PY{p}{(}\PY{p}{)}\PY{o}{.}\PY{n}{median}\PY{p}{(}\PY{p}{)}
\PY{n+nb}{print}\PY{p}{(}\PY{l+s+s1}{\PYZsq{}}\PY{l+s+s1}{Медиана для }\PY{l+s+s1}{\PYZdq{}}\PY{l+s+s1}{компаньон женщина}\PY{l+s+s1}{\PYZdq{}}\PY{l+s+s1}{ }\PY{l+s+si}{\PYZob{}:.1f\PYZcb{}}\PY{l+s+s1}{\PYZsq{}}\PY{o}{.}\PY{n}{format}\PY{p}{(}\PY{n}{companion\PYZus{}f}\PY{p}{)}\PY{p}{)}


\PY{n}{rest\PYZus{}m} \PY{o}{=} \PY{n}{bank\PYZus{}clients}\PY{p}{[}\PY{p}{(}\PY{n}{bank\PYZus{}clients}\PY{p}{[}\PY{l+s+s1}{\PYZsq{}}\PY{l+s+s1}{income\PYZus{}type}\PY{l+s+s1}{\PYZsq{}}\PY{p}{]} \PY{o}{==} \PY{l+s+s1}{\PYZsq{}}\PY{l+s+s1}{пенсионер}\PY{l+s+s1}{\PYZsq{}}\PY{p}{)} 
                         \PY{o}{\PYZam{}} \PY{p}{(}\PY{n}{bank\PYZus{}clients}\PY{p}{[}\PY{l+s+s1}{\PYZsq{}}\PY{l+s+s1}{gender}\PY{l+s+s1}{\PYZsq{}}\PY{p}{]} \PY{o}{==} \PY{l+s+s1}{\PYZsq{}}\PY{l+s+s1}{M}\PY{l+s+s1}{\PYZsq{}}\PY{p}{)}\PY{p}{]}\PY{p}{[}\PY{l+s+s1}{\PYZsq{}}\PY{l+s+s1}{total\PYZus{}income}\PY{l+s+s1}{\PYZsq{}}\PY{p}{]}\PY{o}{.}\PY{n}{dropna}\PY{p}{(}\PY{p}{)}\PY{o}{.}\PY{n}{median}\PY{p}{(}\PY{p}{)}
\PY{n+nb}{print}\PY{p}{(}\PY{l+s+s1}{\PYZsq{}}\PY{l+s+s1}{Медиана для }\PY{l+s+s1}{\PYZdq{}}\PY{l+s+s1}{пенсионер мужчина}\PY{l+s+s1}{\PYZdq{}}\PY{l+s+s1}{ }\PY{l+s+si}{\PYZob{}:.1f\PYZcb{}}\PY{l+s+s1}{\PYZsq{}}\PY{o}{.}\PY{n}{format}\PY{p}{(}\PY{n}{rest\PYZus{}m}\PY{p}{)}\PY{p}{)}
\PY{n}{rest\PYZus{}f} \PY{o}{=} \PY{n}{bank\PYZus{}clients}\PY{p}{[}\PY{p}{(}\PY{n}{bank\PYZus{}clients}\PY{p}{[}\PY{l+s+s1}{\PYZsq{}}\PY{l+s+s1}{income\PYZus{}type}\PY{l+s+s1}{\PYZsq{}}\PY{p}{]} \PY{o}{==} \PY{l+s+s1}{\PYZsq{}}\PY{l+s+s1}{пенсионер}\PY{l+s+s1}{\PYZsq{}}\PY{p}{)} 
                         \PY{o}{\PYZam{}} \PY{p}{(}\PY{n}{bank\PYZus{}clients}\PY{p}{[}\PY{l+s+s1}{\PYZsq{}}\PY{l+s+s1}{gender}\PY{l+s+s1}{\PYZsq{}}\PY{p}{]} \PY{o}{==} \PY{l+s+s1}{\PYZsq{}}\PY{l+s+s1}{F}\PY{l+s+s1}{\PYZsq{}}\PY{p}{)}\PY{p}{]}\PY{p}{[}\PY{l+s+s1}{\PYZsq{}}\PY{l+s+s1}{total\PYZus{}income}\PY{l+s+s1}{\PYZsq{}}\PY{p}{]}\PY{o}{.}\PY{n}{dropna}\PY{p}{(}\PY{p}{)}\PY{o}{.}\PY{n}{median}\PY{p}{(}\PY{p}{)}
\PY{n+nb}{print}\PY{p}{(}\PY{l+s+s1}{\PYZsq{}}\PY{l+s+s1}{Медиана для }\PY{l+s+s1}{\PYZdq{}}\PY{l+s+s1}{пенсионер женщина}\PY{l+s+s1}{\PYZdq{}}\PY{l+s+s1}{ }\PY{l+s+si}{\PYZob{}:.1f\PYZcb{}}\PY{l+s+s1}{\PYZsq{}}\PY{o}{.}\PY{n}{format}\PY{p}{(}\PY{n}{rest\PYZus{}f}\PY{p}{)}\PY{p}{)}

\PY{n}{businessman} \PY{o}{=} \PY{n}{bank\PYZus{}clients}\PY{p}{[}\PY{p}{(}\PY{n}{bank\PYZus{}clients}\PY{p}{[}\PY{l+s+s1}{\PYZsq{}}\PY{l+s+s1}{income\PYZus{}type}\PY{l+s+s1}{\PYZsq{}}\PY{p}{]} \PY{o}{==} \PY{l+s+s1}{\PYZsq{}}\PY{l+s+s1}{предприниматель}\PY{l+s+s1}{\PYZsq{}}\PY{p}{)}\PY{p}{]}\PY{p}{[}\PY{l+s+s1}{\PYZsq{}}\PY{l+s+s1}{total\PYZus{}income}\PY{l+s+s1}{\PYZsq{}}\PY{p}{]}\PY{o}{.}\PY{n}{dropna}\PY{p}{(}\PY{p}{)}\PY{o}{.}\PY{n}{median}\PY{p}{(}\PY{p}{)}

\PY{n+nb}{print}\PY{p}{(}\PY{l+s+s1}{\PYZsq{}}\PY{l+s+s1}{Медиана для }\PY{l+s+s1}{\PYZdq{}}\PY{l+s+s1}{предприниматель}\PY{l+s+s1}{\PYZdq{}}\PY{l+s+s1}{ }\PY{l+s+si}{\PYZob{}:.1f\PYZcb{}}\PY{l+s+s1}{\PYZsq{}}\PY{o}{.}\PY{n}{format}\PY{p}{(}\PY{n}{businessman}\PY{p}{)}\PY{p}{)}
\end{Verbatim}
\end{tcolorbox}

    \begin{Verbatim}[commandchars=\\\{\}]
Медиана для "госслужащий мужчина" 185964.9
Медиана для "госслужащий женщина" 136982.5
Медиана для "компаньон мужчина" 196818.8
Медиана для "компаньон женщина" 160820.8
Медиана для "пенсионер мужчина" 130739.8
Медиана для "пенсионер женщина" 115807.8
Медиана для "предприниматель" 499163.1
\end{Verbatim}

    Мы узнали данные для заполнения, теперь будем заполнять пропуски Сначала
заменим все пропуски на 0, а после в зависимости от пола и занятости
заполним все 0

    \begin{tcolorbox}[breakable, size=fbox, boxrule=1pt, pad at break*=1mm,colback=cellbackground, colframe=cellborder]
\prompt{In}{incolor}{116}{\hspace{4pt}}
\begin{Verbatim}[commandchars=\\\{\}]
\PY{n}{bank\PYZus{}clients}\PY{p}{[}\PY{l+s+s1}{\PYZsq{}}\PY{l+s+s1}{total\PYZus{}income}\PY{l+s+s1}{\PYZsq{}}\PY{p}{]} \PY{o}{=} \PY{n}{bank\PYZus{}clients}\PY{p}{[}\PY{l+s+s1}{\PYZsq{}}\PY{l+s+s1}{total\PYZus{}income}\PY{l+s+s1}{\PYZsq{}}\PY{p}{]}\PY{o}{.}\PY{n}{fillna}\PY{p}{(}\PY{l+m+mi}{0}\PY{p}{)}
\PY{n}{bank\PYZus{}clients}\PY{o}{.}\PY{n}{loc}\PY{p}{[}\PY{p}{:}\PY{p}{,} \PY{l+s+s1}{\PYZsq{}}\PY{l+s+s1}{total\PYZus{}income}\PY{l+s+s1}{\PYZsq{}}\PY{p}{]} \PY{o}{=} \PY{n}{bank\PYZus{}clients}\PY{o}{.}\PY{n}{loc}\PY{p}{[}\PY{p}{:}\PY{p}{,} \PY{l+s+s1}{\PYZsq{}}\PY{l+s+s1}{total\PYZus{}income}\PY{l+s+s1}{\PYZsq{}}\PY{p}{]}\PY{o}{.}\PY{n}{astype}\PY{p}{(}\PY{l+s+s1}{\PYZsq{}}\PY{l+s+s1}{int}\PY{l+s+s1}{\PYZsq{}}\PY{p}{)}
\end{Verbatim}
\end{tcolorbox}

    \begin{tcolorbox}[breakable, size=fbox, boxrule=1pt, pad at break*=1mm,colback=cellbackground, colframe=cellborder]
\prompt{In}{incolor}{117}{\hspace{4pt}}
\begin{Verbatim}[commandchars=\\\{\}]
\PY{n}{x} \PY{o}{=} \PY{n}{bank\PYZus{}clients}\PY{p}{[}\PY{n}{bank\PYZus{}clients}\PY{p}{[}\PY{l+s+s1}{\PYZsq{}}\PY{l+s+s1}{total\PYZus{}income}\PY{l+s+s1}{\PYZsq{}}\PY{p}{]} \PY{o}{==} \PY{l+m+mi}{0}\PY{p}{]}\PY{o}{.}\PY{n}{count}\PY{p}{(}\PY{p}{)}
\PY{n}{x}
\end{Verbatim}
\end{tcolorbox}

            \begin{tcolorbox}[breakable, boxrule=.5pt, size=fbox, pad at break*=1mm, opacityfill=0]
\prompt{Out}{outcolor}{117}{\hspace{3.5pt}}
\begin{Verbatim}[commandchars=\\\{\}]
children            2120
days\_employed          0
dob\_years           2120
education           2120
education\_id        2120
family\_status       2120
family\_status\_id    2120
gender              2120
income\_type         2120
debt                2120
total\_income        2120
purpose             2120
dtype: int64
\end{Verbatim}
\end{tcolorbox}
        
    \begin{tcolorbox}[breakable, size=fbox, boxrule=1pt, pad at break*=1mm,colback=cellbackground, colframe=cellborder]
\prompt{In}{incolor}{118}{\hspace{4pt}}
\begin{Verbatim}[commandchars=\\\{\}]
\PY{k}{def} \PY{n+nf}{zero\PYZus{}incom\PYZus{}delete}\PY{p}{(}\PY{n}{row}\PY{p}{)}\PY{p}{:}
    \PY{n}{income\PYZus{}type} \PY{o}{=} \PY{n}{row}\PY{p}{[}\PY{l+s+s1}{\PYZsq{}}\PY{l+s+s1}{income\PYZus{}type}\PY{l+s+s1}{\PYZsq{}}\PY{p}{]} 
    \PY{n}{gender} \PY{o}{=} \PY{n}{row}\PY{p}{[}\PY{l+s+s1}{\PYZsq{}}\PY{l+s+s1}{gender}\PY{l+s+s1}{\PYZsq{}}\PY{p}{]} 
    \PY{n}{total\PYZus{}income} \PY{o}{=} \PY{n}{row}\PY{p}{[}\PY{l+s+s1}{\PYZsq{}}\PY{l+s+s1}{total\PYZus{}income}\PY{l+s+s1}{\PYZsq{}}\PY{p}{]} 
    \PY{k}{if} \PY{n}{total\PYZus{}income} \PY{o}{==} \PY{l+m+mi}{0}\PY{p}{:}
        \PY{k}{if} \PY{n}{income\PYZus{}type} \PY{o}{==} \PY{l+s+s1}{\PYZsq{}}\PY{l+s+s1}{пенсионер}\PY{l+s+s1}{\PYZsq{}}\PY{p}{:}
            \PY{k}{if} \PY{n}{gender} \PY{o}{==} \PY{l+s+s1}{\PYZsq{}}\PY{l+s+s1}{m}\PY{l+s+s1}{\PYZsq{}}\PY{p}{:}
                \PY{k}{return} \PY{n}{rest\PYZus{}m}
            \PY{k}{else}\PY{p}{:}
                \PY{k}{return} \PY{n}{rest\PYZus{}f}
        \PY{k}{if} \PY{n}{income\PYZus{}type} \PY{o}{==} \PY{l+s+s1}{\PYZsq{}}\PY{l+s+s1}{госслужащий}\PY{l+s+s1}{\PYZsq{}}\PY{p}{:}
            \PY{k}{if} \PY{n}{gender} \PY{o}{==} \PY{l+s+s1}{\PYZsq{}}\PY{l+s+s1}{m}\PY{l+s+s1}{\PYZsq{}}\PY{p}{:}
                \PY{k}{return} \PY{n}{civil\PYZus{}servant\PYZus{}m}
            \PY{k}{else}\PY{p}{:}
                \PY{k}{return} \PY{n}{civil\PYZus{}servant\PYZus{}f}
        \PY{k}{if} \PY{n}{income\PYZus{}type} \PY{o}{==} \PY{l+s+s1}{\PYZsq{}}\PY{l+s+s1}{компаньон}\PY{l+s+s1}{\PYZsq{}}\PY{p}{:}
            \PY{k}{if} \PY{n}{gender} \PY{o}{==} \PY{l+s+s1}{\PYZsq{}}\PY{l+s+s1}{m}\PY{l+s+s1}{\PYZsq{}}\PY{p}{:}
                \PY{k}{return} \PY{n}{companion\PYZus{}m}
            \PY{k}{else}\PY{p}{:}
                \PY{k}{return} \PY{n}{companion\PYZus{}f}
        \PY{k}{if} \PY{n}{income\PYZus{}type} \PY{o}{==} \PY{l+s+s1}{\PYZsq{}}\PY{l+s+s1}{сотрудник}\PY{l+s+s1}{\PYZsq{}}\PY{p}{:}
            \PY{k}{if} \PY{n}{gender} \PY{o}{==} \PY{l+s+s1}{\PYZsq{}}\PY{l+s+s1}{m}\PY{l+s+s1}{\PYZsq{}}\PY{p}{:}
                \PY{k}{return} \PY{n}{employe\PYZus{}m}
            \PY{k}{else}\PY{p}{:}
                \PY{k}{return} \PY{n}{employe\PYZus{}f}
        \PY{k}{if} \PY{n}{income\PYZus{}type} \PY{o}{==} \PY{l+s+s1}{\PYZsq{}}\PY{l+s+s1}{предприниматель}\PY{l+s+s1}{\PYZsq{}}\PY{p}{:}
                \PY{k}{return} \PY{n}{businessman}
    \PY{k}{return} \PY{n}{total\PYZus{}income}

\PY{n}{bank\PYZus{}clients}\PY{p}{[}\PY{l+s+s1}{\PYZsq{}}\PY{l+s+s1}{total\PYZus{}income}\PY{l+s+s1}{\PYZsq{}}\PY{p}{]} \PY{o}{=} \PY{n}{bank\PYZus{}clients}\PY{o}{.}\PY{n}{apply}\PY{p}{(}\PY{n}{zero\PYZus{}incom\PYZus{}delete}\PY{p}{,} \PY{n}{axis}\PY{o}{=}\PY{l+m+mi}{1}\PY{p}{)}

\PY{n}{bank\PYZus{}clients}\PY{o}{.}\PY{n}{head}\PY{p}{(}\PY{l+m+mi}{30}\PY{p}{)}
\end{Verbatim}
\end{tcolorbox}

            \begin{tcolorbox}[breakable, boxrule=.5pt, size=fbox, pad at break*=1mm, opacityfill=0]
\prompt{Out}{outcolor}{118}{\hspace{3.5pt}}
\begin{Verbatim}[commandchars=\\\{\}]
    children  days\_employed  dob\_years            education  education\_id  \textbackslash{}
0          1   -8437.673028         42               высшее             0
1          1   -4024.803754         36              среднее             1
2          0   -5623.422610         33              Среднее             1
3          3   -4124.747207         32              среднее             1
4          0  340266.072047         53              среднее             1
5          0    -926.185831         27               высшее             0
6          0   -2879.202052         43               высшее             0
7          0    -152.779569         50              СРЕДНЕЕ             1
8          2   -6929.865299         35               ВЫСШЕЕ             0
9          0   -2188.756445         41              среднее             1
10         2   -4171.483647         36               высшее             0
11         0    -792.701887         40              среднее             1
12         0            NaN         65              среднее             1
13         0   -1846.641941         54  неоконченное высшее             2
14         0   -1844.956182         56               высшее             0
15         1    -972.364419         26              среднее             1
16         0   -1719.934226         35              среднее             1
17         0   -2369.999720         33               высшее             0
18         0  400281.136913         53              среднее             1
19         0  -10038.818549         48              СРЕДНЕЕ             1
20         1   -1311.604166         36              среднее             1
21         1    -253.685166         33              среднее             1
22         1   -1766.644138         24              среднее             1
23         0    -272.981385         21               высшее             0
24         1  338551.952911         57              среднее             1
25         0  363548.489348         67              среднее             1
26         0            NaN         41              среднее             1
27         0    -529.191635         28               высшее             0
28         1    -717.274324         26               высшее             0
29         0            NaN         63              среднее             1

            family\_status  family\_status\_id gender  income\_type  debt  \textbackslash{}
0         женат / замужем                 0      F    сотрудник     0
1         женат / замужем                 0      F    сотрудник     0
2         женат / замужем                 0      M    сотрудник     0
3         женат / замужем                 0      M    сотрудник     0
4        гражданский брак                 1      F    пенсионер     0
5        гражданский брак                 1      M    компаньон     0
6         женат / замужем                 0      F    компаньон     0
7         женат / замужем                 0      M    сотрудник     0
8        гражданский брак                 1      F    сотрудник     0
9         женат / замужем                 0      M    сотрудник     0
10        женат / замужем                 0      M    компаньон     0
11        женат / замужем                 0      F    сотрудник     0
12       гражданский брак                 1      M    пенсионер     0
13        женат / замужем                 0      F    сотрудник     0
14       гражданский брак                 1      F    компаньон     1
15        женат / замужем                 0      F    сотрудник     0
16        женат / замужем                 0      F    сотрудник     0
17       гражданский брак                 1      M    сотрудник     0
18         вдовец / вдова                 2      F    пенсионер     0
19              в разводе                 3      F    сотрудник     0
20        женат / замужем                 0      M    сотрудник     0
21       гражданский брак                 1      F    сотрудник     0
22       гражданский брак                 1      F    сотрудник     0
23       гражданский брак                 1      M    сотрудник     0
24  Не женат / не замужем                 4      F    пенсионер     0
25        женат / замужем                 0      M    пенсионер     0
26        женат / замужем                 0      M  госслужащий     0
27        женат / замужем                 0      M    сотрудник     0
28        женат / замужем                 0      F    сотрудник     0
29  Не женат / не замужем                 4      F    пенсионер     0

     total\_income                                 purpose
0   253875.000000                           покупка жилья
1   112080.000000                 приобретение автомобиля
2   145885.000000                           покупка жилья
3   267628.000000              дополнительное образование
4   158616.000000                         сыграть свадьбу
5   255763.000000                           покупка жилья
6   240525.000000                       операции с жильем
7   135823.000000                             образование
8    95856.000000                   на проведение свадьбы
9   144425.000000                 покупка жилья для семьи
10  113943.000000                    покупка недвижимости
11   77069.000000       покупка коммерческой недвижимости
12  115807.789733                         сыграть свадьбу
13  130458.000000                 приобретение автомобиля
14  165127.000000              покупка жилой недвижимости
15  116820.000000  строительство собственной недвижимости
16  289202.000000                            недвижимость
17   90410.000000              строительство недвижимости
18   56823.000000      на покупку подержанного автомобиля
19  242831.000000            на покупку своего автомобиля
20  209552.000000                            недвижимость
21  131812.000000                 приобретение автомобиля
22  149681.000000      на покупку подержанного автомобиля
23  128265.000000                         сыграть свадьбу
24  290547.000000   операции с коммерческой недвижимостью
25   55112.000000                    покупка недвижимости
26  136982.489425                             образование
27  308848.000000  строительство собственной недвижимости
28  187863.000000  строительство собственной недвижимости
29  115807.789733        строительство жилой недвижимости
\end{Verbatim}
\end{tcolorbox}
        
    АЕ! Кто молодец? Я! Заполнили total\_income, притом не какой-то общей
медианой, а медианой по полу и занятости!

    Что с days\_employed? зачем там плавающая точка, это же дни Никто не
запоминает с количество отработаннх дней с точностью до месяца, за малым
исключением, Поэтому смело можно избавляться от данных после запятой.

    Молодец, обработка пропусков теперь выполнена правильно. Так данные мы
можем восстановить наиболее качественно.

\begin{center}\rule{0.5\linewidth}{0.5pt}\end{center}

    

    \begin{tcolorbox}[breakable, size=fbox, boxrule=1pt, pad at break*=1mm,colback=cellbackground, colframe=cellborder]
\prompt{In}{incolor}{119}{\hspace{4pt}}
\begin{Verbatim}[commandchars=\\\{\}]
\PY{n}{bank\PYZus{}clients}\PY{p}{[}\PY{l+s+s1}{\PYZsq{}}\PY{l+s+s1}{days\PYZus{}employed}\PY{l+s+s1}{\PYZsq{}}\PY{p}{]} \PY{o}{=} \PY{n}{bank\PYZus{}clients}\PY{p}{[}\PY{l+s+s1}{\PYZsq{}}\PY{l+s+s1}{days\PYZus{}employed}\PY{l+s+s1}{\PYZsq{}}\PY{p}{]}\PY{o}{.}\PY{n}{fillna}\PY{p}{(}\PY{l+m+mi}{0}\PY{p}{)}
\PY{n+nb}{print}\PY{p}{(}\PY{n}{bank\PYZus{}clients}\PY{p}{[}\PY{n}{bank\PYZus{}clients}\PY{p}{[}\PY{l+s+s1}{\PYZsq{}}\PY{l+s+s1}{days\PYZus{}employed}\PY{l+s+s1}{\PYZsq{}}\PY{p}{]} \PY{o}{\PYZlt{}} \PY{l+m+mi}{1}\PY{p}{]}\PY{o}{.}\PY{n}{info}\PY{p}{(}\PY{p}{)}\PY{p}{)}
\end{Verbatim}
\end{tcolorbox}

    \begin{Verbatim}[commandchars=\\\{\}]
<class 'pandas.core.frame.DataFrame'>
Int64Index: 18026 entries, 0 to 21470
Data columns (total 12 columns):
children            18026 non-null int64
days\_employed       18026 non-null float64
dob\_years           18026 non-null int64
education           18026 non-null object
education\_id        18026 non-null int64
family\_status       18026 non-null object
family\_status\_id    18026 non-null int64
gender              18026 non-null object
income\_type         18026 non-null object
debt                18026 non-null int64
total\_income        18026 non-null float64
purpose             18026 non-null object
dtypes: float64(2), int64(5), object(5)
memory usage: 1.8+ MB
None
\end{Verbatim}

    18к неположительных часов рабочих. У нас спрашивают зависимость от иной
переменной, но ``странных'' данных слишком много, Поэтому предположим,
что знак минус, как и с детьми это дефис

    \begin{tcolorbox}[breakable, size=fbox, boxrule=1pt, pad at break*=1mm,colback=cellbackground, colframe=cellborder]
\prompt{In}{incolor}{120}{\hspace{4pt}}
\begin{Verbatim}[commandchars=\\\{\}]
\PY{n}{bank\PYZus{}clients}\PY{o}{.}\PY{n}{loc}\PY{p}{[}\PY{n}{bank\PYZus{}clients}\PY{o}{.}\PY{n}{loc}\PY{p}{[}\PY{p}{:}\PY{p}{,} \PY{l+s+s1}{\PYZsq{}}\PY{l+s+s1}{days\PYZus{}employed}\PY{l+s+s1}{\PYZsq{}}\PY{p}{]} \PY{o}{\PYZlt{}} \PY{l+m+mi}{0}\PY{p}{,} 
         \PY{l+s+s1}{\PYZsq{}}\PY{l+s+s1}{days\PYZus{}employed}\PY{l+s+s1}{\PYZsq{}}\PY{p}{]} \PY{o}{=} \PY{o}{\PYZhy{}}\PY{n}{bank\PYZus{}clients}\PY{o}{.}\PY{n}{loc}\PY{p}{[}\PY{n}{bank\PYZus{}clients}\PY{o}{.}\PY{n}{loc}\PY{p}{[}\PY{p}{:}\PY{p}{,} \PY{l+s+s1}{\PYZsq{}}\PY{l+s+s1}{days\PYZus{}employed}\PY{l+s+s1}{\PYZsq{}}\PY{p}{]} \PY{o}{\PYZlt{}} \PY{l+m+mi}{0}\PY{p}{,} \PY{l+s+s1}{\PYZsq{}}\PY{l+s+s1}{days\PYZus{}employed}\PY{l+s+s1}{\PYZsq{}}\PY{p}{]}

\PY{n+nb}{print}\PY{p}{(}\PY{n}{bank\PYZus{}clients}\PY{p}{[}\PY{l+s+s1}{\PYZsq{}}\PY{l+s+s1}{days\PYZus{}employed}\PY{l+s+s1}{\PYZsq{}}\PY{p}{]}\PY{o}{.}\PY{n}{value\PYZus{}counts}\PY{p}{(}\PY{p}{)}\PY{p}{)}
\end{Verbatim}
\end{tcolorbox}

    \begin{Verbatim}[commandchars=\\\{\}]
0.000000       2120
1645.463049       1
6620.396473       1
1238.560080       1
3047.519891       1
               {\ldots}
2849.351119       1
5619.328204       1
448.829898        1
1687.038672       1
206.107342        1
Name: days\_employed, Length: 19352, dtype: int64
\end{Verbatim}

    Находим людей, у которых стаж длиннее жизни)))

    \begin{tcolorbox}[breakable, size=fbox, boxrule=1pt, pad at break*=1mm,colback=cellbackground, colframe=cellborder]
\prompt{In}{incolor}{121}{\hspace{4pt}}
\begin{Verbatim}[commandchars=\\\{\}]
\PY{n}{days\PYZus{}before\PYZus{}work} \PY{o}{=} \PY{l+m+mi}{18}\PY{o}{*}\PY{l+m+mf}{365.25}
\PY{n}{bank\PYZus{}clients}\PY{p}{[}\PY{l+s+s1}{\PYZsq{}}\PY{l+s+s1}{check}\PY{l+s+s1}{\PYZsq{}}\PY{p}{]} \PY{o}{=} \PY{n}{bank\PYZus{}clients}\PY{p}{[}\PY{l+s+s1}{\PYZsq{}}\PY{l+s+s1}{dob\PYZus{}years}\PY{l+s+s1}{\PYZsq{}}\PY{p}{]}\PY{o}{*}\PY{l+m+mi}{365} \PY{o}{\PYZhy{}} \PY{n}{days\PYZus{}before\PYZus{}work} \PY{o}{\PYZhy{}} \PY{n}{bank\PYZus{}clients}\PY{p}{[}\PY{l+s+s1}{\PYZsq{}}\PY{l+s+s1}{days\PYZus{}employed}\PY{l+s+s1}{\PYZsq{}}\PY{p}{]}
\PY{n}{bank\PYZus{}clients}\PY{o}{.}\PY{n}{loc}\PY{p}{[}\PY{p}{:}\PY{p}{,} \PY{l+s+s1}{\PYZsq{}}\PY{l+s+s1}{check}\PY{l+s+s1}{\PYZsq{}}\PY{p}{]} \PY{o}{=} \PY{n}{bank\PYZus{}clients}\PY{o}{.}\PY{n}{loc}\PY{p}{[}\PY{p}{:}\PY{p}{,} \PY{l+s+s1}{\PYZsq{}}\PY{l+s+s1}{check}\PY{l+s+s1}{\PYZsq{}}\PY{p}{]}\PY{o}{.}\PY{n}{astype}\PY{p}{(}\PY{l+s+s1}{\PYZsq{}}\PY{l+s+s1}{int}\PY{l+s+s1}{\PYZsq{}}\PY{p}{)}

\PY{n+nb}{print}\PY{p}{(}\PY{n}{bank\PYZus{}clients}\PY{p}{[}\PY{n}{bank\PYZus{}clients}\PY{p}{[}\PY{l+s+s1}{\PYZsq{}}\PY{l+s+s1}{check}\PY{l+s+s1}{\PYZsq{}}\PY{p}{]} \PY{o}{\PYZlt{}} \PY{l+m+mi}{0}\PY{p}{]}\PY{o}{.}\PY{n}{groupby}\PY{p}{(}\PY{l+s+s1}{\PYZsq{}}\PY{l+s+s1}{debt}\PY{l+s+s1}{\PYZsq{}}\PY{p}{)}\PY{p}{[}\PY{l+s+s1}{\PYZsq{}}\PY{l+s+s1}{debt}\PY{l+s+s1}{\PYZsq{}}\PY{p}{]}\PY{o}{.}\PY{n}{count}\PY{p}{(}\PY{p}{)}\PY{p}{)}
\end{Verbatim}
\end{tcolorbox}

    \begin{Verbatim}[commandchars=\\\{\}]
debt
0    3725
1     218
Name: debt, dtype: int64
\end{Verbatim}

    Подавляющее большинство не имеют долгов, а это значит что увеличение
стажа не яалялось преднамеренным для согласования кредита Давайте
попробуем заполнить пропущенные данные с помощью медиан

    \begin{tcolorbox}[breakable, size=fbox, boxrule=1pt, pad at break*=1mm,colback=cellbackground, colframe=cellborder]
\prompt{In}{incolor}{122}{\hspace{4pt}}
\begin{Verbatim}[commandchars=\\\{\}]
\PY{n}{days\PYZus{}employed\PYZus{}employe\PYZus{}m} \PY{o}{=} \PY{n}{bank\PYZus{}clients}\PY{p}{[}\PY{p}{(}\PY{n}{bank\PYZus{}clients}\PY{p}{[}\PY{l+s+s1}{\PYZsq{}}\PY{l+s+s1}{income\PYZus{}type}\PY{l+s+s1}{\PYZsq{}}\PY{p}{]} \PY{o}{==} \PY{l+s+s1}{\PYZsq{}}\PY{l+s+s1}{сотрудник}\PY{l+s+s1}{\PYZsq{}}\PY{p}{)} 
                         \PY{o}{\PYZam{}} \PY{p}{(}\PY{n}{bank\PYZus{}clients}\PY{p}{[}\PY{l+s+s1}{\PYZsq{}}\PY{l+s+s1}{gender}\PY{l+s+s1}{\PYZsq{}}\PY{p}{]} \PY{o}{==} \PY{l+s+s1}{\PYZsq{}}\PY{l+s+s1}{M}\PY{l+s+s1}{\PYZsq{}}\PY{p}{)} \PY{o}{\PYZam{}} \PY{p}{(}\PY{n}{bank\PYZus{}clients}\PY{p}{[}\PY{l+s+s1}{\PYZsq{}}\PY{l+s+s1}{days\PYZus{}employed}\PY{l+s+s1}{\PYZsq{}}\PY{p}{]} \PY{o}{!=} \PY{l+m+mi}{0}\PY{p}{)}\PY{p}{]}\PY{p}{[}\PY{l+s+s1}{\PYZsq{}}\PY{l+s+s1}{days\PYZus{}employed}\PY{l+s+s1}{\PYZsq{}}\PY{p}{]}\PY{o}{.}\PY{n}{median}\PY{p}{(}\PY{p}{)}
\PY{n+nb}{print}\PY{p}{(}\PY{l+s+s1}{\PYZsq{}}\PY{l+s+s1}{Медиана для }\PY{l+s+s1}{\PYZdq{}}\PY{l+s+s1}{сотрудник мужчина}\PY{l+s+s1}{\PYZdq{}}\PY{l+s+s1}{ }\PY{l+s+si}{\PYZob{}:.1f\PYZcb{}}\PY{l+s+s1}{\PYZsq{}}\PY{o}{.}\PY{n}{format}\PY{p}{(}\PY{n}{days\PYZus{}employed\PYZus{}employe\PYZus{}m}\PY{p}{)}\PY{p}{)}                                         

\PY{n}{days\PYZus{}employed\PYZus{}employe\PYZus{}f} \PY{o}{=} \PY{n}{bank\PYZus{}clients}\PY{p}{[}\PY{p}{(}\PY{n}{bank\PYZus{}clients}\PY{p}{[}\PY{l+s+s1}{\PYZsq{}}\PY{l+s+s1}{income\PYZus{}type}\PY{l+s+s1}{\PYZsq{}}\PY{p}{]} \PY{o}{==} \PY{l+s+s1}{\PYZsq{}}\PY{l+s+s1}{сотрудник}\PY{l+s+s1}{\PYZsq{}}\PY{p}{)} 
                         \PY{o}{\PYZam{}} \PY{p}{(}\PY{n}{bank\PYZus{}clients}\PY{p}{[}\PY{l+s+s1}{\PYZsq{}}\PY{l+s+s1}{gender}\PY{l+s+s1}{\PYZsq{}}\PY{p}{]} \PY{o}{==} \PY{l+s+s1}{\PYZsq{}}\PY{l+s+s1}{F}\PY{l+s+s1}{\PYZsq{}}\PY{p}{)} \PY{o}{\PYZam{}} \PY{p}{(}\PY{n}{bank\PYZus{}clients}\PY{p}{[}\PY{l+s+s1}{\PYZsq{}}\PY{l+s+s1}{days\PYZus{}employed}\PY{l+s+s1}{\PYZsq{}}\PY{p}{]} \PY{o}{!=} \PY{l+m+mi}{0}\PY{p}{)}\PY{p}{]}\PY{p}{[}\PY{l+s+s1}{\PYZsq{}}\PY{l+s+s1}{days\PYZus{}employed}\PY{l+s+s1}{\PYZsq{}}\PY{p}{]}\PY{o}{.}\PY{n}{median}\PY{p}{(}\PY{p}{)}

\PY{n+nb}{print}\PY{p}{(}\PY{l+s+s1}{\PYZsq{}}\PY{l+s+s1}{Медиана для }\PY{l+s+s1}{\PYZdq{}}\PY{l+s+s1}{сотрудник женщина}\PY{l+s+s1}{\PYZdq{}}\PY{l+s+s1}{ }\PY{l+s+si}{\PYZob{}:.1f\PYZcb{}}\PY{l+s+s1}{\PYZsq{}}\PY{o}{.}\PY{n}{format}\PY{p}{(}\PY{n}{days\PYZus{}employed\PYZus{}employe\PYZus{}f}\PY{p}{)}\PY{p}{)}

\PY{n}{days\PYZus{}employed\PYZus{}civil\PYZus{}servant\PYZus{}m} \PY{o}{=} \PY{n}{bank\PYZus{}clients}\PY{p}{[}\PY{p}{(}\PY{n}{bank\PYZus{}clients}\PY{p}{[}\PY{l+s+s1}{\PYZsq{}}\PY{l+s+s1}{income\PYZus{}type}\PY{l+s+s1}{\PYZsq{}}\PY{p}{]} \PY{o}{==} \PY{l+s+s1}{\PYZsq{}}\PY{l+s+s1}{госслужащий}\PY{l+s+s1}{\PYZsq{}}\PY{p}{)} 
                         \PY{o}{\PYZam{}} \PY{p}{(}\PY{n}{bank\PYZus{}clients}\PY{p}{[}\PY{l+s+s1}{\PYZsq{}}\PY{l+s+s1}{gender}\PY{l+s+s1}{\PYZsq{}}\PY{p}{]} \PY{o}{==} \PY{l+s+s1}{\PYZsq{}}\PY{l+s+s1}{M}\PY{l+s+s1}{\PYZsq{}}\PY{p}{)} \PY{o}{\PYZam{}} \PY{p}{(}\PY{n}{bank\PYZus{}clients}\PY{p}{[}\PY{l+s+s1}{\PYZsq{}}\PY{l+s+s1}{days\PYZus{}employed}\PY{l+s+s1}{\PYZsq{}}\PY{p}{]} \PY{o}{!=} \PY{l+m+mi}{0}\PY{p}{)}\PY{p}{]}\PY{p}{[}\PY{l+s+s1}{\PYZsq{}}\PY{l+s+s1}{days\PYZus{}employed}\PY{l+s+s1}{\PYZsq{}}\PY{p}{]}\PY{o}{.}\PY{n}{median}\PY{p}{(}\PY{p}{)}
\PY{n+nb}{print}\PY{p}{(}\PY{l+s+s1}{\PYZsq{}}\PY{l+s+s1}{Медиана для }\PY{l+s+s1}{\PYZdq{}}\PY{l+s+s1}{госслужащий мужчина}\PY{l+s+s1}{\PYZdq{}}\PY{l+s+s1}{ }\PY{l+s+si}{\PYZob{}:.1f\PYZcb{}}\PY{l+s+s1}{\PYZsq{}}\PY{o}{.}\PY{n}{format}\PY{p}{(}\PY{n}{days\PYZus{}employed\PYZus{}civil\PYZus{}servant\PYZus{}m}\PY{p}{)}\PY{p}{)}
\PY{n}{days\PYZus{}employed\PYZus{}civil\PYZus{}servant\PYZus{}f} \PY{o}{=} \PY{n}{bank\PYZus{}clients}\PY{p}{[}\PY{p}{(}\PY{n}{bank\PYZus{}clients}\PY{p}{[}\PY{l+s+s1}{\PYZsq{}}\PY{l+s+s1}{income\PYZus{}type}\PY{l+s+s1}{\PYZsq{}}\PY{p}{]} \PY{o}{==} \PY{l+s+s1}{\PYZsq{}}\PY{l+s+s1}{госслужащий}\PY{l+s+s1}{\PYZsq{}}\PY{p}{)} 
                         \PY{o}{\PYZam{}} \PY{p}{(}\PY{n}{bank\PYZus{}clients}\PY{p}{[}\PY{l+s+s1}{\PYZsq{}}\PY{l+s+s1}{gender}\PY{l+s+s1}{\PYZsq{}}\PY{p}{]} \PY{o}{==} \PY{l+s+s1}{\PYZsq{}}\PY{l+s+s1}{F}\PY{l+s+s1}{\PYZsq{}}\PY{p}{)} \PY{o}{\PYZam{}} \PY{p}{(}\PY{n}{bank\PYZus{}clients}\PY{p}{[}\PY{l+s+s1}{\PYZsq{}}\PY{l+s+s1}{days\PYZus{}employed}\PY{l+s+s1}{\PYZsq{}}\PY{p}{]} \PY{o}{!=} \PY{l+m+mi}{0}\PY{p}{)}\PY{p}{]}\PY{p}{[}\PY{l+s+s1}{\PYZsq{}}\PY{l+s+s1}{days\PYZus{}employed}\PY{l+s+s1}{\PYZsq{}}\PY{p}{]}\PY{o}{.}\PY{n}{median}\PY{p}{(}\PY{p}{)}
\PY{n+nb}{print}\PY{p}{(}\PY{l+s+s1}{\PYZsq{}}\PY{l+s+s1}{Медиана для }\PY{l+s+s1}{\PYZdq{}}\PY{l+s+s1}{госслужащий женщина}\PY{l+s+s1}{\PYZdq{}}\PY{l+s+s1}{ }\PY{l+s+si}{\PYZob{}:.1f\PYZcb{}}\PY{l+s+s1}{\PYZsq{}}\PY{o}{.}\PY{n}{format}\PY{p}{(}\PY{n}{days\PYZus{}employed\PYZus{}civil\PYZus{}servant\PYZus{}f}\PY{p}{)}\PY{p}{)}

\PY{n}{days\PYZus{}employed\PYZus{}companion\PYZus{}m} \PY{o}{=} \PY{n}{bank\PYZus{}clients}\PY{p}{[}\PY{p}{(}\PY{n}{bank\PYZus{}clients}\PY{p}{[}\PY{l+s+s1}{\PYZsq{}}\PY{l+s+s1}{income\PYZus{}type}\PY{l+s+s1}{\PYZsq{}}\PY{p}{]} \PY{o}{==} \PY{l+s+s1}{\PYZsq{}}\PY{l+s+s1}{компаньон}\PY{l+s+s1}{\PYZsq{}}\PY{p}{)} 
                         \PY{o}{\PYZam{}} \PY{p}{(}\PY{n}{bank\PYZus{}clients}\PY{p}{[}\PY{l+s+s1}{\PYZsq{}}\PY{l+s+s1}{gender}\PY{l+s+s1}{\PYZsq{}}\PY{p}{]} \PY{o}{==} \PY{l+s+s1}{\PYZsq{}}\PY{l+s+s1}{M}\PY{l+s+s1}{\PYZsq{}}\PY{p}{)} \PY{o}{\PYZam{}} \PY{p}{(}\PY{n}{bank\PYZus{}clients}\PY{p}{[}\PY{l+s+s1}{\PYZsq{}}\PY{l+s+s1}{days\PYZus{}employed}\PY{l+s+s1}{\PYZsq{}}\PY{p}{]} \PY{o}{!=} \PY{l+m+mi}{0}\PY{p}{)}\PY{p}{]}\PY{p}{[}\PY{l+s+s1}{\PYZsq{}}\PY{l+s+s1}{days\PYZus{}employed}\PY{l+s+s1}{\PYZsq{}}\PY{p}{]}\PY{o}{.}\PY{n}{median}\PY{p}{(}\PY{p}{)}
\PY{n+nb}{print}\PY{p}{(}\PY{l+s+s1}{\PYZsq{}}\PY{l+s+s1}{Медиана для }\PY{l+s+s1}{\PYZdq{}}\PY{l+s+s1}{компаньон мужчина}\PY{l+s+s1}{\PYZdq{}}\PY{l+s+s1}{ }\PY{l+s+si}{\PYZob{}:.1f\PYZcb{}}\PY{l+s+s1}{\PYZsq{}}\PY{o}{.}\PY{n}{format}\PY{p}{(}\PY{n}{days\PYZus{}employed\PYZus{}companion\PYZus{}m}\PY{p}{)}\PY{p}{)}
\PY{n}{days\PYZus{}employed\PYZus{}companion\PYZus{}f} \PY{o}{=} \PY{n}{bank\PYZus{}clients}\PY{p}{[}\PY{p}{(}\PY{n}{bank\PYZus{}clients}\PY{p}{[}\PY{l+s+s1}{\PYZsq{}}\PY{l+s+s1}{income\PYZus{}type}\PY{l+s+s1}{\PYZsq{}}\PY{p}{]} \PY{o}{==} \PY{l+s+s1}{\PYZsq{}}\PY{l+s+s1}{компаньон}\PY{l+s+s1}{\PYZsq{}}\PY{p}{)} 
                         \PY{o}{\PYZam{}} \PY{p}{(}\PY{n}{bank\PYZus{}clients}\PY{p}{[}\PY{l+s+s1}{\PYZsq{}}\PY{l+s+s1}{gender}\PY{l+s+s1}{\PYZsq{}}\PY{p}{]} \PY{o}{==} \PY{l+s+s1}{\PYZsq{}}\PY{l+s+s1}{F}\PY{l+s+s1}{\PYZsq{}}\PY{p}{)} \PY{o}{\PYZam{}} \PY{p}{(}\PY{n}{bank\PYZus{}clients}\PY{p}{[}\PY{l+s+s1}{\PYZsq{}}\PY{l+s+s1}{days\PYZus{}employed}\PY{l+s+s1}{\PYZsq{}}\PY{p}{]} \PY{o}{!=} \PY{l+m+mi}{0}\PY{p}{)}\PY{p}{]}\PY{p}{[}\PY{l+s+s1}{\PYZsq{}}\PY{l+s+s1}{days\PYZus{}employed}\PY{l+s+s1}{\PYZsq{}}\PY{p}{]}\PY{o}{.}\PY{n}{median}\PY{p}{(}\PY{p}{)}
\PY{n+nb}{print}\PY{p}{(}\PY{l+s+s1}{\PYZsq{}}\PY{l+s+s1}{Медиана для }\PY{l+s+s1}{\PYZdq{}}\PY{l+s+s1}{компаньон женщина}\PY{l+s+s1}{\PYZdq{}}\PY{l+s+s1}{ }\PY{l+s+si}{\PYZob{}:.1f\PYZcb{}}\PY{l+s+s1}{\PYZsq{}}\PY{o}{.}\PY{n}{format}\PY{p}{(}\PY{n}{days\PYZus{}employed\PYZus{}companion\PYZus{}f}\PY{p}{)}\PY{p}{)}


\PY{n}{days\PYZus{}employed\PYZus{}rest\PYZus{}m} \PY{o}{=} \PY{n}{bank\PYZus{}clients}\PY{p}{[}\PY{p}{(}\PY{n}{bank\PYZus{}clients}\PY{p}{[}\PY{l+s+s1}{\PYZsq{}}\PY{l+s+s1}{income\PYZus{}type}\PY{l+s+s1}{\PYZsq{}}\PY{p}{]} \PY{o}{==} \PY{l+s+s1}{\PYZsq{}}\PY{l+s+s1}{пенсионер}\PY{l+s+s1}{\PYZsq{}}\PY{p}{)} 
                         \PY{o}{\PYZam{}} \PY{p}{(}\PY{n}{bank\PYZus{}clients}\PY{p}{[}\PY{l+s+s1}{\PYZsq{}}\PY{l+s+s1}{gender}\PY{l+s+s1}{\PYZsq{}}\PY{p}{]} \PY{o}{==} \PY{l+s+s1}{\PYZsq{}}\PY{l+s+s1}{M}\PY{l+s+s1}{\PYZsq{}}\PY{p}{)} \PY{o}{\PYZam{}} \PY{p}{(}\PY{n}{bank\PYZus{}clients}\PY{p}{[}\PY{l+s+s1}{\PYZsq{}}\PY{l+s+s1}{days\PYZus{}employed}\PY{l+s+s1}{\PYZsq{}}\PY{p}{]} \PY{o}{!=} \PY{l+m+mi}{0}\PY{p}{)}\PY{p}{]}\PY{p}{[}\PY{l+s+s1}{\PYZsq{}}\PY{l+s+s1}{days\PYZus{}employed}\PY{l+s+s1}{\PYZsq{}}\PY{p}{]}\PY{o}{.}\PY{n}{median}\PY{p}{(}\PY{p}{)}
\PY{n+nb}{print}\PY{p}{(}\PY{l+s+s1}{\PYZsq{}}\PY{l+s+s1}{Медиана для }\PY{l+s+s1}{\PYZdq{}}\PY{l+s+s1}{пенсионер мужчина}\PY{l+s+s1}{\PYZdq{}}\PY{l+s+s1}{ }\PY{l+s+si}{\PYZob{}:.1f\PYZcb{}}\PY{l+s+s1}{\PYZsq{}}\PY{o}{.}\PY{n}{format}\PY{p}{(}\PY{n}{days\PYZus{}employed\PYZus{}rest\PYZus{}m}\PY{p}{)}\PY{p}{)}
\PY{n}{days\PYZus{}employed\PYZus{}rest\PYZus{}f} \PY{o}{=} \PY{n}{bank\PYZus{}clients}\PY{p}{[}\PY{p}{(}\PY{n}{bank\PYZus{}clients}\PY{p}{[}\PY{l+s+s1}{\PYZsq{}}\PY{l+s+s1}{income\PYZus{}type}\PY{l+s+s1}{\PYZsq{}}\PY{p}{]} \PY{o}{==} \PY{l+s+s1}{\PYZsq{}}\PY{l+s+s1}{пенсионер}\PY{l+s+s1}{\PYZsq{}}\PY{p}{)} 
                         \PY{o}{\PYZam{}} \PY{p}{(}\PY{n}{bank\PYZus{}clients}\PY{p}{[}\PY{l+s+s1}{\PYZsq{}}\PY{l+s+s1}{gender}\PY{l+s+s1}{\PYZsq{}}\PY{p}{]} \PY{o}{==} \PY{l+s+s1}{\PYZsq{}}\PY{l+s+s1}{F}\PY{l+s+s1}{\PYZsq{}}\PY{p}{)} \PY{o}{\PYZam{}} \PY{p}{(}\PY{n}{bank\PYZus{}clients}\PY{p}{[}\PY{l+s+s1}{\PYZsq{}}\PY{l+s+s1}{days\PYZus{}employed}\PY{l+s+s1}{\PYZsq{}}\PY{p}{]} \PY{o}{!=} \PY{l+m+mi}{0}\PY{p}{)}\PY{p}{]}\PY{p}{[}\PY{l+s+s1}{\PYZsq{}}\PY{l+s+s1}{days\PYZus{}employed}\PY{l+s+s1}{\PYZsq{}}\PY{p}{]}\PY{o}{.}\PY{n}{median}\PY{p}{(}\PY{p}{)}
\PY{n+nb}{print}\PY{p}{(}\PY{l+s+s1}{\PYZsq{}}\PY{l+s+s1}{Медиана для }\PY{l+s+s1}{\PYZdq{}}\PY{l+s+s1}{пенсионер женщина}\PY{l+s+s1}{\PYZdq{}}\PY{l+s+s1}{ }\PY{l+s+si}{\PYZob{}:.1f\PYZcb{}}\PY{l+s+s1}{\PYZsq{}}\PY{o}{.}\PY{n}{format}\PY{p}{(}\PY{n}{days\PYZus{}employed\PYZus{}rest\PYZus{}f}\PY{p}{)}\PY{p}{)}

\PY{n}{days\PYZus{}employed\PYZus{}businessman} \PY{o}{=} \PY{n}{bank\PYZus{}clients}\PY{p}{[}\PY{p}{(}\PY{n}{bank\PYZus{}clients}\PY{p}{[}\PY{l+s+s1}{\PYZsq{}}\PY{l+s+s1}{income\PYZus{}type}\PY{l+s+s1}{\PYZsq{}}\PY{p}{]} \PY{o}{==} \PY{l+s+s1}{\PYZsq{}}\PY{l+s+s1}{предприниматель}\PY{l+s+s1}{\PYZsq{}}\PY{p}{)} \PY{o}{\PYZam{}} \PY{p}{(}\PY{n}{bank\PYZus{}clients}\PY{p}{[}\PY{l+s+s1}{\PYZsq{}}\PY{l+s+s1}{days\PYZus{}employed}\PY{l+s+s1}{\PYZsq{}}\PY{p}{]} \PY{o}{!=} \PY{l+m+mi}{0}\PY{p}{)}\PY{p}{]}\PY{p}{[}\PY{l+s+s1}{\PYZsq{}}\PY{l+s+s1}{days\PYZus{}employed}\PY{l+s+s1}{\PYZsq{}}\PY{p}{]}\PY{o}{.}\PY{n}{median}\PY{p}{(}\PY{p}{)}

\PY{n+nb}{print}\PY{p}{(}\PY{l+s+s1}{\PYZsq{}}\PY{l+s+s1}{Медиана для }\PY{l+s+s1}{\PYZdq{}}\PY{l+s+s1}{предприниматель}\PY{l+s+s1}{\PYZdq{}}\PY{l+s+s1}{ }\PY{l+s+si}{\PYZob{}:.1f\PYZcb{}}\PY{l+s+s1}{\PYZsq{}}\PY{o}{.}\PY{n}{format}\PY{p}{(}\PY{n}{days\PYZus{}employed\PYZus{}businessman}\PY{p}{)}\PY{p}{)}
\end{Verbatim}
\end{tcolorbox}

    \begin{Verbatim}[commandchars=\\\{\}]
Медиана для "сотрудник мужчина" 1362.7
Медиана для "сотрудник женщина" 1718.5
Медиана для "госслужащий мужчина" 2659.3
Медиана для "госслужащий женщина" 2705.8
Медиана для "компаньон мужчина" 1472.9
Медиана для "компаньон женщина" 1584.3
Медиана для "пенсионер мужчина" 361781.5
Медиана для "пенсионер женщина" 366182.9
Медиана для "предприниматель" 520.8
\end{Verbatim}

    Нашли медианы рабочих дней. Пора подставлять

    \begin{tcolorbox}[breakable, size=fbox, boxrule=1pt, pad at break*=1mm,colback=cellbackground, colframe=cellborder]
\prompt{In}{incolor}{123}{\hspace{4pt}}
\begin{Verbatim}[commandchars=\\\{\}]
\PY{n}{bank\PYZus{}clients}\PY{o}{.}\PY{n}{head}\PY{p}{(}\PY{l+m+mi}{30}\PY{p}{)}
\end{Verbatim}
\end{tcolorbox}

            \begin{tcolorbox}[breakable, boxrule=.5pt, size=fbox, pad at break*=1mm, opacityfill=0]
\prompt{Out}{outcolor}{123}{\hspace{3.5pt}}
\begin{Verbatim}[commandchars=\\\{\}]
    children  days\_employed  dob\_years            education  education\_id  \textbackslash{}
0          1    8437.673028         42               высшее             0
1          1    4024.803754         36              среднее             1
2          0    5623.422610         33              Среднее             1
3          3    4124.747207         32              среднее             1
4          0  340266.072047         53              среднее             1
5          0     926.185831         27               высшее             0
6          0    2879.202052         43               высшее             0
7          0     152.779569         50              СРЕДНЕЕ             1
8          2    6929.865299         35               ВЫСШЕЕ             0
9          0    2188.756445         41              среднее             1
10         2    4171.483647         36               высшее             0
11         0     792.701887         40              среднее             1
12         0       0.000000         65              среднее             1
13         0    1846.641941         54  неоконченное высшее             2
14         0    1844.956182         56               высшее             0
15         1     972.364419         26              среднее             1
16         0    1719.934226         35              среднее             1
17         0    2369.999720         33               высшее             0
18         0  400281.136913         53              среднее             1
19         0   10038.818549         48              СРЕДНЕЕ             1
20         1    1311.604166         36              среднее             1
21         1     253.685166         33              среднее             1
22         1    1766.644138         24              среднее             1
23         0     272.981385         21               высшее             0
24         1  338551.952911         57              среднее             1
25         0  363548.489348         67              среднее             1
26         0       0.000000         41              среднее             1
27         0     529.191635         28               высшее             0
28         1     717.274324         26               высшее             0
29         0       0.000000         63              среднее             1

            family\_status  family\_status\_id gender  income\_type  debt  \textbackslash{}
0         женат / замужем                 0      F    сотрудник     0
1         женат / замужем                 0      F    сотрудник     0
2         женат / замужем                 0      M    сотрудник     0
3         женат / замужем                 0      M    сотрудник     0
4        гражданский брак                 1      F    пенсионер     0
5        гражданский брак                 1      M    компаньон     0
6         женат / замужем                 0      F    компаньон     0
7         женат / замужем                 0      M    сотрудник     0
8        гражданский брак                 1      F    сотрудник     0
9         женат / замужем                 0      M    сотрудник     0
10        женат / замужем                 0      M    компаньон     0
11        женат / замужем                 0      F    сотрудник     0
12       гражданский брак                 1      M    пенсионер     0
13        женат / замужем                 0      F    сотрудник     0
14       гражданский брак                 1      F    компаньон     1
15        женат / замужем                 0      F    сотрудник     0
16        женат / замужем                 0      F    сотрудник     0
17       гражданский брак                 1      M    сотрудник     0
18         вдовец / вдова                 2      F    пенсионер     0
19              в разводе                 3      F    сотрудник     0
20        женат / замужем                 0      M    сотрудник     0
21       гражданский брак                 1      F    сотрудник     0
22       гражданский брак                 1      F    сотрудник     0
23       гражданский брак                 1      M    сотрудник     0
24  Не женат / не замужем                 4      F    пенсионер     0
25        женат / замужем                 0      M    пенсионер     0
26        женат / замужем                 0      M  госслужащий     0
27        женат / замужем                 0      M    сотрудник     0
28        женат / замужем                 0      F    сотрудник     0
29  Не женат / не замужем                 4      F    пенсионер     0

     total\_income                                 purpose   check
0   253875.000000                           покупка жилья     317
1   112080.000000                 приобретение автомобиля    2540
2   145885.000000                           покупка жилья    -152
3   267628.000000              дополнительное образование     980
4   158616.000000                         сыграть свадьбу -327495
5   255763.000000                           покупка жилья    2354
6   240525.000000                       операции с жильем    6241
7   135823.000000                             образование   11522
8    95856.000000                   на проведение свадьбы    -729
9   144425.000000                 покупка жилья для семьи    6201
10  113943.000000                    покупка недвижимости    2394
11   77069.000000       покупка коммерческой недвижимости    7232
12  115807.789733                         сыграть свадьбу   17150
13  130458.000000                 приобретение автомобиля   11288
14  165127.000000              покупка жилой недвижимости   12020
15  116820.000000  строительство собственной недвижимости    1943
16  289202.000000                            недвижимость    4480
17   90410.000000              строительство недвижимости    3100
18   56823.000000      на покупку подержанного автомобиля -387510
19  242831.000000            на покупку своего автомобиля     906
20  209552.000000                            недвижимость    5253
21  131812.000000                 приобретение автомобиля    5216
22  149681.000000      на покупку подержанного автомобиля     418
23  128265.000000                         сыграть свадьбу     817
24  290547.000000   операции с коммерческой недвижимостью -324321
25   55112.000000                    покупка недвижимости -345667
26  136982.489425                             образование    8390
27  308848.000000  строительство собственной недвижимости    3116
28  187863.000000  строительство собственной недвижимости    2198
29  115807.789733        строительство жилой недвижимости   16420
\end{Verbatim}
\end{tcolorbox}
        
    \begin{tcolorbox}[breakable, size=fbox, boxrule=1pt, pad at break*=1mm,colback=cellbackground, colframe=cellborder]
\prompt{In}{incolor}{124}{\hspace{4pt}}
\begin{Verbatim}[commandchars=\\\{\}]
\PY{k}{def} \PY{n+nf}{zero\PYZus{}days\PYZus{}employed\PYZus{}delete}\PY{p}{(}\PY{n}{row}\PY{p}{)}\PY{p}{:}
    \PY{n}{income\PYZus{}type} \PY{o}{=} \PY{n}{row}\PY{p}{[}\PY{l+s+s1}{\PYZsq{}}\PY{l+s+s1}{income\PYZus{}type}\PY{l+s+s1}{\PYZsq{}}\PY{p}{]} 
    \PY{n}{gender} \PY{o}{=} \PY{n}{row}\PY{p}{[}\PY{l+s+s1}{\PYZsq{}}\PY{l+s+s1}{gender}\PY{l+s+s1}{\PYZsq{}}\PY{p}{]} 
    \PY{n}{days\PYZus{}employed} \PY{o}{=} \PY{n}{row}\PY{p}{[}\PY{l+s+s1}{\PYZsq{}}\PY{l+s+s1}{days\PYZus{}employed}\PY{l+s+s1}{\PYZsq{}}\PY{p}{]} 
    \PY{k}{if} \PY{n}{days\PYZus{}employed} \PY{o}{==} \PY{l+m+mi}{0}\PY{p}{:}
        \PY{k}{if} \PY{n}{income\PYZus{}type} \PY{o}{==} \PY{l+s+s1}{\PYZsq{}}\PY{l+s+s1}{пенсионер}\PY{l+s+s1}{\PYZsq{}}\PY{p}{:}
            \PY{k}{if} \PY{n}{gender} \PY{o}{==} \PY{l+s+s1}{\PYZsq{}}\PY{l+s+s1}{m}\PY{l+s+s1}{\PYZsq{}}\PY{p}{:}
                \PY{k}{return} \PY{n}{days\PYZus{}employed\PYZus{}rest\PYZus{}m}
            \PY{k}{else}\PY{p}{:}
                \PY{k}{return} \PY{n}{days\PYZus{}employed\PYZus{}rest\PYZus{}f}
        \PY{k}{if} \PY{n}{income\PYZus{}type} \PY{o}{==} \PY{l+s+s1}{\PYZsq{}}\PY{l+s+s1}{госслужащий}\PY{l+s+s1}{\PYZsq{}}\PY{p}{:}
            \PY{k}{if} \PY{n}{gender} \PY{o}{==} \PY{l+s+s1}{\PYZsq{}}\PY{l+s+s1}{m}\PY{l+s+s1}{\PYZsq{}}\PY{p}{:}
                \PY{k}{return} \PY{n}{days\PYZus{}employed\PYZus{}civil\PYZus{}servant\PYZus{}m}
            \PY{k}{else}\PY{p}{:}
                \PY{k}{return} \PY{n}{days\PYZus{}employed\PYZus{}civil\PYZus{}servant\PYZus{}f}
        \PY{k}{if} \PY{n}{income\PYZus{}type} \PY{o}{==} \PY{l+s+s1}{\PYZsq{}}\PY{l+s+s1}{компаньон}\PY{l+s+s1}{\PYZsq{}}\PY{p}{:}
            \PY{k}{if} \PY{n}{gender} \PY{o}{==} \PY{l+s+s1}{\PYZsq{}}\PY{l+s+s1}{m}\PY{l+s+s1}{\PYZsq{}}\PY{p}{:}
                \PY{k}{return} \PY{n}{days\PYZus{}employed\PYZus{}companion\PYZus{}m}
            \PY{k}{else}\PY{p}{:}
                \PY{k}{return} \PY{n}{days\PYZus{}employed\PYZus{}companion\PYZus{}f}
        \PY{k}{if} \PY{n}{income\PYZus{}type} \PY{o}{==} \PY{l+s+s1}{\PYZsq{}}\PY{l+s+s1}{сотрудник}\PY{l+s+s1}{\PYZsq{}}\PY{p}{:}
            \PY{k}{if} \PY{n}{gender} \PY{o}{==} \PY{l+s+s1}{\PYZsq{}}\PY{l+s+s1}{m}\PY{l+s+s1}{\PYZsq{}}\PY{p}{:}
                \PY{k}{return} \PY{n}{days\PYZus{}employed\PYZus{}employe\PYZus{}m}
            \PY{k}{else}\PY{p}{:}
                \PY{k}{return} \PY{n}{days\PYZus{}employed\PYZus{}employe\PYZus{}f}
        \PY{k}{if} \PY{n}{income\PYZus{}type} \PY{o}{==} \PY{l+s+s1}{\PYZsq{}}\PY{l+s+s1}{предприниматель}\PY{l+s+s1}{\PYZsq{}}\PY{p}{:}
            \PY{k}{return} \PY{n}{days\PYZus{}employed\PYZus{}businessman}
    \PY{k}{return} \PY{n}{days\PYZus{}employed}

\PY{n}{bank\PYZus{}clients}\PY{p}{[}\PY{l+s+s1}{\PYZsq{}}\PY{l+s+s1}{days\PYZus{}employed}\PY{l+s+s1}{\PYZsq{}}\PY{p}{]} \PY{o}{=} \PY{n}{bank\PYZus{}clients}\PY{o}{.}\PY{n}{apply}\PY{p}{(}\PY{n}{zero\PYZus{}days\PYZus{}employed\PYZus{}delete}\PY{p}{,} \PY{n}{axis}\PY{o}{=}\PY{l+m+mi}{1}\PY{p}{)}

\PY{n+nb}{print}\PY{p}{(}\PY{n}{bank\PYZus{}clients}\PY{p}{[}\PY{l+s+s1}{\PYZsq{}}\PY{l+s+s1}{days\PYZus{}employed}\PY{l+s+s1}{\PYZsq{}}\PY{p}{]}\PY{o}{.}\PY{n}{value\PYZus{}counts}\PY{p}{(}\PY{p}{)}\PY{p}{)}
\end{Verbatim}
\end{tcolorbox}

    \begin{Verbatim}[commandchars=\\\{\}]
1718.507316      1078
1584.335109       503
366182.933484     394
2705.835929       145
520.848083          2
                 {\ldots}
2849.351119         1
5619.328204         1
448.829898          1
1687.038672         1
582.538413          1
Name: days\_employed, Length: 19354, dtype: int64
\end{Verbatim}

    Вот и славненько. Людей без стажа или с отрицательным стажем у нас нет!

    \hypertarget{ux432ux44bux432ux43eux434}{%
\subsubsection{Вывод}\label{ux432ux44bux432ux43eux434}}

    Данные очень гразные Есть вероятность, и очень большая, что в данных
очень много пасхалок, половину из которых я не нашел Заполнил пустые
ячейки медианами соотетствующих категорий. Вызывает огромный вопрос 4000
человек, стаж которых больше их взрослой жизни, но есть вероятность, что
люди трудятся на 2х и более работах А если и нет, то убрать ``этих
людей'' мы не можем - они составляют почти 20\% Можно было бы по медиане
присвоить и им стаж, но думаю, это решение будет неверным

    Комментарий наставника

Данные у нас по людям, уже получившим кредиты. Сомнительно, что у них не
будет дохода. Лучше использовать заполнение пропусков в соответствии с
типом дохода. Тогда пенсионеры будут получать также, как и другие
пенсионеры и т.д. Все-таки в данных есть достаточное число групп,
заполнять пропуски в стаже, скажем, у студентов и пенсионеров одним и
тем же значением не очень то правильно. То же самое касается и столбца с
доходами: пропуски у предпринимателя и пенсионера заполняются одним
числом, что вызывает вопросы. Подумай, как лучше обработать пропуски.

    DONE

    \hypertarget{ux437ux430ux43cux435ux43dux430-ux442ux438ux43fux430-ux434ux430ux43dux43dux44bux445}{%
\subsubsection{Замена типа
данных}\label{ux437ux430ux43cux435ux43dux430-ux442ux438ux43fux430-ux434ux430ux43dux43dux44bux445}}

    Убираем знаки после запятой в days\_employed и total\_income

    \begin{tcolorbox}[breakable, size=fbox, boxrule=1pt, pad at break*=1mm,colback=cellbackground, colframe=cellborder]
\prompt{In}{incolor}{125}{\hspace{4pt}}
\begin{Verbatim}[commandchars=\\\{\}]
\PY{n}{bank\PYZus{}clients}\PY{o}{.}\PY{n}{loc}\PY{p}{[}\PY{p}{:}\PY{p}{,} \PY{l+s+s1}{\PYZsq{}}\PY{l+s+s1}{days\PYZus{}employed}\PY{l+s+s1}{\PYZsq{}}\PY{p}{]} \PY{o}{=} \PY{n}{bank\PYZus{}clients}\PY{o}{.}\PY{n}{loc}\PY{p}{[}\PY{p}{:}\PY{p}{,} \PY{l+s+s1}{\PYZsq{}}\PY{l+s+s1}{days\PYZus{}employed}\PY{l+s+s1}{\PYZsq{}}\PY{p}{]}\PY{o}{.}\PY{n}{astype}\PY{p}{(}\PY{l+s+s1}{\PYZsq{}}\PY{l+s+s1}{int}\PY{l+s+s1}{\PYZsq{}}\PY{p}{)}
\PY{n}{bank\PYZus{}clients}\PY{o}{.}\PY{n}{loc}\PY{p}{[}\PY{p}{:}\PY{p}{,} \PY{l+s+s1}{\PYZsq{}}\PY{l+s+s1}{total\PYZus{}income}\PY{l+s+s1}{\PYZsq{}}\PY{p}{]} \PY{o}{=} \PY{n}{bank\PYZus{}clients}\PY{o}{.}\PY{n}{loc}\PY{p}{[}\PY{p}{:}\PY{p}{,} \PY{l+s+s1}{\PYZsq{}}\PY{l+s+s1}{total\PYZus{}income}\PY{l+s+s1}{\PYZsq{}}\PY{p}{]}\PY{o}{.}\PY{n}{astype}\PY{p}{(}\PY{l+s+s1}{\PYZsq{}}\PY{l+s+s1}{int}\PY{l+s+s1}{\PYZsq{}}\PY{p}{)}
\PY{n}{bank\PYZus{}clients}\PY{o}{.}\PY{n}{head}\PY{p}{(}\PY{p}{)}
\end{Verbatim}
\end{tcolorbox}

            \begin{tcolorbox}[breakable, boxrule=.5pt, size=fbox, pad at break*=1mm, opacityfill=0]
\prompt{Out}{outcolor}{125}{\hspace{3.5pt}}
\begin{Verbatim}[commandchars=\\\{\}]
   children  days\_employed  dob\_years education  education\_id  \textbackslash{}
0         1           8437         42    высшее             0
1         1           4024         36   среднее             1
2         0           5623         33   Среднее             1
3         3           4124         32   среднее             1
4         0         340266         53   среднее             1

      family\_status  family\_status\_id gender income\_type  debt  total\_income  \textbackslash{}
0   женат / замужем                 0      F   сотрудник     0        253875
1   женат / замужем                 0      F   сотрудник     0        112080
2   женат / замужем                 0      M   сотрудник     0        145885
3   женат / замужем                 0      M   сотрудник     0        267628
4  гражданский брак                 1      F   пенсионер     0        158616

                      purpose   check
0               покупка жилья     317
1     приобретение автомобиля    2540
2               покупка жилья    -152
3  дополнительное образование     980
4             сыграть свадьбу -327495
\end{Verbatim}
\end{tcolorbox}
        
    \begin{tcolorbox}[breakable, size=fbox, boxrule=1pt, pad at break*=1mm,colback=cellbackground, colframe=cellborder]
\prompt{In}{incolor}{126}{\hspace{4pt}}
\begin{Verbatim}[commandchars=\\\{\}]
\PY{n}{x} \PY{o}{=} \PY{n}{bank\PYZus{}clients}\PY{o}{.}\PY{n}{duplicated}\PY{p}{(}\PY{p}{)}\PY{o}{.}\PY{n}{sum}\PY{p}{(}\PY{p}{)}
\PY{n+nb}{print}\PY{p}{(}\PY{l+s+s1}{\PYZsq{}}\PY{l+s+s1}{На данный момент дубликатов}\PY{l+s+s1}{\PYZsq{}}\PY{p}{,} \PY{n}{x}\PY{p}{)}
\end{Verbatim}
\end{tcolorbox}

    \begin{Verbatim}[commandchars=\\\{\}]
На данный момент дубликатов 0
\end{Verbatim}

    Вывод: После удаления точек после запятой данные не пострадади ни на 1
индекс

    \hypertarget{ux432ux44bux432ux43eux434}{%
\subsubsection{Вывод}\label{ux432ux44bux432ux43eux434}}

    Замена типа данных появляется периодически за все прохождение задания.
Это связано с некорректным подбором типа данных в начальной таблице и
последующим его изменением

    Комментарий наставника

Исходя из общей информации о таблице мы видим, что два столбца имеют
вещественный тип данных, который и надо заменить на целочисленный. Это
замена произведена верно. Также хорошо бы освоить метод to\_numeric()
для будущих работ. Только желательно соблюдать структуру проекта и
проводить все действия в свих разделах.

    При всем желании и уважении, исправить пропуски без замены

    \hypertarget{ux43eux431ux440ux430ux431ux43eux442ux43aux430-ux434ux443ux431ux43bux438ux43aux430ux442ux43eux432}{%
\subsubsection{Обработка
дубликатов}\label{ux43eux431ux440ux430ux431ux43eux442ux43aux430-ux434ux443ux431ux43bux438ux43aux430ux442ux43eux432}}

    \begin{tcolorbox}[breakable, size=fbox, boxrule=1pt, pad at break*=1mm,colback=cellbackground, colframe=cellborder]
\prompt{In}{incolor}{127}{\hspace{4pt}}
\begin{Verbatim}[commandchars=\\\{\}]
\PY{n+nb}{print}\PY{p}{(}\PY{n}{bank\PYZus{}clients}\PY{p}{[}\PY{l+s+s1}{\PYZsq{}}\PY{l+s+s1}{education}\PY{l+s+s1}{\PYZsq{}}\PY{p}{]}\PY{o}{.}\PY{n}{value\PYZus{}counts}\PY{p}{(}\PY{p}{)}\PY{p}{)}
\end{Verbatim}
\end{tcolorbox}

    \begin{Verbatim}[commandchars=\\\{\}]
среднее                13705
высшее                  4710
СРЕДНЕЕ                  772
Среднее                  711
неоконченное высшее      668
ВЫСШЕЕ                   273
Высшее                   268
начальное                250
Неоконченное высшее       47
НЕОКОНЧЕННОЕ ВЫСШЕЕ       29
НАЧАЛЬНОЕ                 17
Начальное                 15
ученая степень             4
УЧЕНАЯ СТЕПЕНЬ             1
Ученая степень             1
Name: education, dtype: int64
\end{Verbatim}

    Обнаружилась кривизна заполнения. исправляем

    \begin{tcolorbox}[breakable, size=fbox, boxrule=1pt, pad at break*=1mm,colback=cellbackground, colframe=cellborder]
\prompt{In}{incolor}{128}{\hspace{4pt}}
\begin{Verbatim}[commandchars=\\\{\}]
\PY{n}{bank\PYZus{}clients}\PY{p}{[}\PY{l+s+s1}{\PYZsq{}}\PY{l+s+s1}{education}\PY{l+s+s1}{\PYZsq{}}\PY{p}{]} \PY{o}{=} \PY{n}{bank\PYZus{}clients}\PY{p}{[}\PY{l+s+s1}{\PYZsq{}}\PY{l+s+s1}{education}\PY{l+s+s1}{\PYZsq{}}\PY{p}{]}\PY{o}{.}\PY{n}{str}\PY{o}{.}\PY{n}{lower}\PY{p}{(}\PY{p}{)}
\PY{n+nb}{print}\PY{p}{(}\PY{n}{bank\PYZus{}clients}\PY{p}{[}\PY{l+s+s1}{\PYZsq{}}\PY{l+s+s1}{education}\PY{l+s+s1}{\PYZsq{}}\PY{p}{]}\PY{o}{.}\PY{n}{value\PYZus{}counts}\PY{p}{(}\PY{p}{)}\PY{p}{)}
\end{Verbatim}
\end{tcolorbox}

    \begin{Verbatim}[commandchars=\\\{\}]
среднее                15188
высшее                  5251
неоконченное высшее      744
начальное                282
ученая степень             6
Name: education, dtype: int64
\end{Verbatim}

    Проверяем соответствие столбцов education и education\_id

    \begin{tcolorbox}[breakable, size=fbox, boxrule=1pt, pad at break*=1mm,colback=cellbackground, colframe=cellborder]
\prompt{In}{incolor}{129}{\hspace{4pt}}
\begin{Verbatim}[commandchars=\\\{\}]
\PY{n+nb}{print}\PY{p}{(}\PY{n}{bank\PYZus{}clients}\PY{o}{.}\PY{n}{info}\PY{p}{(}\PY{p}{)}\PY{p}{)}
\PY{n+nb}{print}\PY{p}{(}\PY{l+s+s1}{\PYZsq{}}\PY{l+s+s1}{\PYZsq{}}\PY{p}{)}
\PY{n+nb}{print}\PY{p}{(}\PY{n}{bank\PYZus{}clients}\PY{o}{.}\PY{n}{groupby}\PY{p}{(}\PY{l+s+s1}{\PYZsq{}}\PY{l+s+s1}{education}\PY{l+s+s1}{\PYZsq{}}\PY{p}{)}\PY{p}{[}\PY{l+s+s1}{\PYZsq{}}\PY{l+s+s1}{education\PYZus{}id}\PY{l+s+s1}{\PYZsq{}}\PY{p}{]}\PY{o}{.}\PY{n}{value\PYZus{}counts}\PY{p}{(}\PY{p}{)}\PY{p}{)}
\end{Verbatim}
\end{tcolorbox}

    \begin{Verbatim}[commandchars=\\\{\}]
<class 'pandas.core.frame.DataFrame'>
RangeIndex: 21471 entries, 0 to 21470
Data columns (total 13 columns):
children            21471 non-null int64
days\_employed       21471 non-null int64
dob\_years           21471 non-null int64
education           21471 non-null object
education\_id        21471 non-null int64
family\_status       21471 non-null object
family\_status\_id    21471 non-null int64
gender              21471 non-null object
income\_type         21471 non-null object
debt                21471 non-null int64
total\_income        21471 non-null int64
purpose             21471 non-null object
check               21471 non-null int64
dtypes: int64(8), object(5)
memory usage: 2.1+ MB
None

education            education\_id
высшее               0                5251
начальное            3                 282
неоконченное высшее  2                 744
среднее              1               15188
ученая степень       4                   6
Name: education\_id, dtype: int64
\end{Verbatim}

    Все ок. данные идентичны Так же проверим и family\_status и
family\_status\_id

    \begin{tcolorbox}[breakable, size=fbox, boxrule=1pt, pad at break*=1mm,colback=cellbackground, colframe=cellborder]
\prompt{In}{incolor}{130}{\hspace{4pt}}
\begin{Verbatim}[commandchars=\\\{\}]
\PY{n}{bank\PYZus{}clients}\PY{p}{[}\PY{l+s+s1}{\PYZsq{}}\PY{l+s+s1}{family\PYZus{}status}\PY{l+s+s1}{\PYZsq{}}\PY{p}{]} \PY{o}{=} \PY{n}{bank\PYZus{}clients}\PY{p}{[}\PY{l+s+s1}{\PYZsq{}}\PY{l+s+s1}{family\PYZus{}status}\PY{l+s+s1}{\PYZsq{}}\PY{p}{]}\PY{o}{.}\PY{n}{str}\PY{o}{.}\PY{n}{lower}\PY{p}{(}\PY{p}{)}
\PY{n+nb}{print}\PY{p}{(}\PY{n}{bank\PYZus{}clients}\PY{o}{.}\PY{n}{groupby}\PY{p}{(}\PY{l+s+s1}{\PYZsq{}}\PY{l+s+s1}{family\PYZus{}status}\PY{l+s+s1}{\PYZsq{}}\PY{p}{)}\PY{p}{[}\PY{l+s+s1}{\PYZsq{}}\PY{l+s+s1}{family\PYZus{}status\PYZus{}id}\PY{l+s+s1}{\PYZsq{}}\PY{p}{]}\PY{o}{.}\PY{n}{value\PYZus{}counts}\PY{p}{(}\PY{p}{)}\PY{p}{)}
\end{Verbatim}
\end{tcolorbox}

    \begin{Verbatim}[commandchars=\\\{\}]
family\_status          family\_status\_id
в разводе              3                    1195
вдовец / вдова         2                     959
гражданский брак       1                    4163
женат / замужем        0                   12344
не женат / не замужем  4                    2810
Name: family\_status\_id, dtype: int64
\end{Verbatim}

    Тут все ровно. только вот для банка нет разницы между ``Не женат / не
замужем'', ``в разводе'' и ``вдовец / вдова'' Я бы еще туда и
гражданский брак отправил, т.к. по законодательству, по крайней мере РБ,
Нет разницы, но, на всякий случай, оставим

    \begin{tcolorbox}[breakable, size=fbox, boxrule=1pt, pad at break*=1mm,colback=cellbackground, colframe=cellborder]
\prompt{In}{incolor}{131}{\hspace{4pt}}
\begin{Verbatim}[commandchars=\\\{\}]
\PY{n}{bank\PYZus{}clients}\PY{o}{.}\PY{n}{loc}\PY{p}{[}\PY{n}{bank\PYZus{}clients}\PY{o}{.}\PY{n}{loc}\PY{p}{[}\PY{p}{:}\PY{p}{,} \PY{l+s+s1}{\PYZsq{}}\PY{l+s+s1}{family\PYZus{}status\PYZus{}id}\PY{l+s+s1}{\PYZsq{}}\PY{p}{]} \PY{o}{\PYZgt{}} \PY{l+m+mi}{1}\PY{p}{,} \PY{l+s+s1}{\PYZsq{}}\PY{l+s+s1}{family\PYZus{}status\PYZus{}id}\PY{l+s+s1}{\PYZsq{}}\PY{p}{]} \PY{o}{=} \PY{l+m+mi}{4}
\PY{n}{bank\PYZus{}clients}\PY{o}{.}\PY{n}{loc}\PY{p}{[}\PY{n}{bank\PYZus{}clients}\PY{o}{.}\PY{n}{loc}\PY{p}{[}\PY{p}{:}\PY{p}{,} \PY{l+s+s1}{\PYZsq{}}\PY{l+s+s1}{family\PYZus{}status\PYZus{}id}\PY{l+s+s1}{\PYZsq{}}\PY{p}{]} \PY{o}{\PYZgt{}} \PY{l+m+mi}{1}\PY{p}{,} 
                 \PY{l+s+s1}{\PYZsq{}}\PY{l+s+s1}{family\PYZus{}status}\PY{l+s+s1}{\PYZsq{}}\PY{p}{]} \PY{o}{=} \PY{l+s+s1}{\PYZsq{}}\PY{l+s+s1}{не женат / не замужем}\PY{l+s+s1}{\PYZsq{}}
\PY{n+nb}{print}\PY{p}{(}\PY{n}{bank\PYZus{}clients}\PY{o}{.}\PY{n}{groupby}\PY{p}{(}\PY{l+s+s1}{\PYZsq{}}\PY{l+s+s1}{family\PYZus{}status}\PY{l+s+s1}{\PYZsq{}}\PY{p}{)}\PY{p}{[}\PY{l+s+s1}{\PYZsq{}}\PY{l+s+s1}{family\PYZus{}status\PYZus{}id}\PY{l+s+s1}{\PYZsq{}}\PY{p}{]}\PY{o}{.}\PY{n}{value\PYZus{}counts}\PY{p}{(}\PY{p}{)}\PY{p}{)}
\end{Verbatim}
\end{tcolorbox}

    \begin{Verbatim}[commandchars=\\\{\}]
family\_status          family\_status\_id
гражданский брак       1                    4163
женат / замужем        0                   12344
не женат / не замужем  4                    4964
Name: family\_status\_id, dtype: int64
\end{Verbatim}

    Настало время drop\_duplicates!!! Но дело в том, что если на этом этапе
удалять дубликаты, есть шанс, что у нас появятся лишние дубликаты, т.к.
часть таблицы я заполнил медианами. Проверим

    \begin{tcolorbox}[breakable, size=fbox, boxrule=1pt, pad at break*=1mm,colback=cellbackground, colframe=cellborder]
\prompt{In}{incolor}{132}{\hspace{4pt}}
\begin{Verbatim}[commandchars=\\\{\}]
\PY{n}{x} \PY{o}{=} \PY{n}{bank\PYZus{}clients}\PY{o}{.}\PY{n}{duplicated}\PY{p}{(}\PY{p}{)}\PY{o}{.}\PY{n}{sum}\PY{p}{(}\PY{p}{)}
\PY{n+nb}{print}\PY{p}{(}\PY{l+s+s1}{\PYZsq{}}\PY{l+s+s1}{На данный момент дубликатов}\PY{l+s+s1}{\PYZsq{}}\PY{p}{,} \PY{n}{x}\PY{p}{)}
\end{Verbatim}
\end{tcolorbox}

    \begin{Verbatim}[commandchars=\\\{\}]
На данный момент дубликатов 21
\end{Verbatim}

    \hypertarget{ux432ux44bux432ux43eux434}{%
\subsubsection{Вывод}\label{ux432ux44bux432ux43eux434}}

    Можно было объеденить и гражданский брак с разводом, но это не совсем
верно Можно сократить количество вариантов до 3х уже при вводе данных
Время покажет

После изменения семейного статуса количество дубликатов возросло дна 21.
Однако, если данные были внесены с разными семейными статусами, из этого
следует, что данные изначально были не идентичны. Поэтому я удалил
дублирующиеся строки в самом начале работы

Проверю дубликаты после категоризации, ради интереса

    Комментарий наставника

Это исправление ошибок в данных. Дубликаты - одинаковые строки в данных.
Они вызывают смещение финальных результатов, поэтому их и надо удалять.
Конечно, в данных нет уникального идентификатора, однако надо уметь
удалять дубликаты.

    DONE

    Метод обработки дубликатов освоен.

\begin{center}\rule{0.5\linewidth}{0.5pt}\end{center}

    \hypertarget{ux43bux435ux43cux43cux430ux442ux438ux437ux430ux446ux438ux44f}{%
\subsubsection{Лемматизация}\label{ux43bux435ux43cux43cux430ux442ux438ux437ux430ux446ux438ux44f}}

    Вероятнее всего лемматизация нужна для purpose, там очень много
вариантов

    \begin{tcolorbox}[breakable, size=fbox, boxrule=1pt, pad at break*=1mm,colback=cellbackground, colframe=cellborder]
\prompt{In}{incolor}{133}{\hspace{4pt}}
\begin{Verbatim}[commandchars=\\\{\}]
\PY{n+nb}{print}\PY{p}{(}\PY{n}{bank\PYZus{}clients}\PY{p}{[}\PY{l+s+s1}{\PYZsq{}}\PY{l+s+s1}{purpose}\PY{l+s+s1}{\PYZsq{}}\PY{p}{]}\PY{o}{.}\PY{n}{unique}\PY{p}{(}\PY{p}{)}\PY{p}{)}
\end{Verbatim}
\end{tcolorbox}

    \begin{Verbatim}[commandchars=\\\{\}]
['покупка жилья' 'приобретение автомобиля' 'дополнительное образование'
 'сыграть свадьбу' 'операции с жильем' 'образование'
 'на проведение свадьбы' 'покупка жилья для семьи' 'покупка недвижимости'
 'покупка коммерческой недвижимости' 'покупка жилой недвижимости'
 'строительство собственной недвижимости' 'недвижимость'
 'строительство недвижимости' 'на покупку подержанного автомобиля'
 'на покупку своего автомобиля' 'операции с коммерческой недвижимостью'
 'строительство жилой недвижимости' 'жилье'
 'операции со своей недвижимостью' 'автомобили' 'заняться образованием'
 'сделка с подержанным автомобилем' 'получение образования' 'автомобиль'
 'свадьба' 'получение дополнительного образования' 'покупка своего жилья'
 'операции с недвижимостью' 'получение высшего образования'
 'свой автомобиль' 'сделка с автомобилем' 'профильное образование'
 'высшее образование' 'покупка жилья для сдачи' 'на покупку автомобиля'
 'ремонт жилью' 'заняться высшим образованием']
\end{Verbatim}

    Ну и, собственно, сама лемматизация:

    \begin{tcolorbox}[breakable, size=fbox, boxrule=1pt, pad at break*=1mm,colback=cellbackground, colframe=cellborder]
\prompt{In}{incolor}{134}{\hspace{4pt}}
\begin{Verbatim}[commandchars=\\\{\}]
\PY{k+kn}{from} \PY{n+nn}{pymystem3} \PY{k+kn}{import} \PY{n}{Mystem}
\PY{n}{m} \PY{o}{=} \PY{n}{Mystem}\PY{p}{(}\PY{p}{)}

\PY{n}{text} \PY{o}{=} \PY{n}{bank\PYZus{}clients}\PY{p}{[}\PY{l+s+s1}{\PYZsq{}}\PY{l+s+s1}{purpose}\PY{l+s+s1}{\PYZsq{}}\PY{p}{]}\PY{o}{.}\PY{n}{unique}\PY{p}{(}\PY{p}{)}
\PY{n}{lemmas} \PY{o}{=} \PY{n}{m}\PY{o}{.}\PY{n}{lemmatize}\PY{p}{(}\PY{l+s+s1}{\PYZsq{}}\PY{l+s+s1}{ }\PY{l+s+s1}{\PYZsq{}}\PY{o}{.}\PY{n}{join}\PY{p}{(}\PY{n}{text}\PY{p}{)}\PY{p}{)}
\PY{k+kn}{from} \PY{n+nn}{collections} \PY{k+kn}{import} \PY{n}{Counter}
\PY{n+nb}{print}\PY{p}{(}\PY{n}{Counter}\PY{p}{(}\PY{n}{lemmas}\PY{p}{)}\PY{p}{)}
\end{Verbatim}
\end{tcolorbox}

    \begin{Verbatim}[commandchars=\\\{\}]
Counter(\{' ': 96, 'покупка': 10, 'недвижимость': 10, 'автомобиль': 9,
'образование': 9, 'жилье': 7, 'с': 5, 'операция': 4, 'на': 4, 'свой': 4,
'свадьба': 3, 'строительство': 3, 'получение': 3, 'высокий': 3,
'дополнительный': 2, 'для': 2, 'коммерческий': 2, 'жилой': 2, 'подержать': 2,
'заниматься': 2, 'сделка': 2, 'приобретение': 1, 'сыграть': 1, 'проведение': 1,
'семья': 1, 'собственный': 1, 'со': 1, 'профильный': 1, 'сдача': 1, 'ремонт': 1,
'\textbackslash{}n': 1\})
\end{Verbatim}

    \hypertarget{ux432ux44bux432ux43eux434}{%
\subsubsection{Вывод}\label{ux432ux44bux432ux43eux434}}

    Есть у меня чуечка, что не тем я занялся. А если серьезно, то на
``покупка'' и ``недвижимость'' в топе. Но этот топ только по
разношерстным упоминаниям. После сортировки списка можно прийти к
меньшему количеству лемм

: 3, `получение': 3, `высокий': 3, `дополнительный': 2, `для': 2,
`коммерческий': 2, `жилой': 2, `подержать': 2, `заниматься': 2,
`сделка': 2, `приобретение': 1, `сыграть': 1, `проведение': 1, `семья':
1, `собственный': 1, `со': 1, `профильный': 1, `сдача': 1, `ремонт': 1,
`\n': 1\})

\begin{enumerate}
\def\labelenumi{\arabic{enumi}.}
\tightlist
\item
  `покупка' (есть шанс что люди брали просто на покупку, после проверки,
  вероятно, этот вариант уберется)
\item
  `недвижимость', `жилье', `строительство'
\item
  `автомобиль'
\item
  `образование'
\item
  `операция' (снова, нужно проверить)
\item
  `свадьба' (это просто зря)
\item
  `приобретение' (та же история что и с "покупкой)
\item
  `сыграть' (надеюсь, что не в казино, поэтому проверю и уберу)
\item
  `проведение' (скорее всего это свадьба, но может быть и гадалка)
\item
  `семья' (в проверку)
\item
  `сдача' (в проверку)
\item
  `ремонт' (в проверку) - ПОТРЕБИТЕЛЬСКИЙ
\end{enumerate}

Оставил все существительные и ``сыграть'', т.к. было интересно, может,
на казино люди брали деньги Далее буду объеденять по категориям и
применять

    Комментарий наставника

Все верно, метод lemmatize() использован правильно. Перейдем к
категоризации цели кредитов.

    \hypertarget{ux43aux430ux442ux435ux433ux43eux440ux438ux437ux430ux446ux438ux44f-ux434ux430ux43dux43dux44bux445}{%
\subsubsection{Категоризация
данных}\label{ux43aux430ux442ux435ux433ux43eux440ux438ux437ux430ux446ux438ux44f-ux434ux430ux43dux43dux44bux445}}

    \begin{tcolorbox}[breakable, size=fbox, boxrule=1pt, pad at break*=1mm,colback=cellbackground, colframe=cellborder]
\prompt{In}{incolor}{135}{\hspace{4pt}}
\begin{Verbatim}[commandchars=\\\{\}]
\PY{n}{text} \PY{o}{=} \PY{n}{bank\PYZus{}clients}\PY{p}{[}\PY{l+s+s1}{\PYZsq{}}\PY{l+s+s1}{purpose}\PY{l+s+s1}{\PYZsq{}}\PY{p}{]}\PY{o}{.}\PY{n}{unique}\PY{p}{(}\PY{p}{)}
\PY{n+nb}{print}\PY{p}{(}\PY{n}{text}\PY{p}{)}
\end{Verbatim}
\end{tcolorbox}

    \begin{Verbatim}[commandchars=\\\{\}]
['покупка жилья' 'приобретение автомобиля' 'дополнительное образование'
 'сыграть свадьбу' 'операции с жильем' 'образование'
 'на проведение свадьбы' 'покупка жилья для семьи' 'покупка недвижимости'
 'покупка коммерческой недвижимости' 'покупка жилой недвижимости'
 'строительство собственной недвижимости' 'недвижимость'
 'строительство недвижимости' 'на покупку подержанного автомобиля'
 'на покупку своего автомобиля' 'операции с коммерческой недвижимостью'
 'строительство жилой недвижимости' 'жилье'
 'операции со своей недвижимостью' 'автомобили' 'заняться образованием'
 'сделка с подержанным автомобилем' 'получение образования' 'автомобиль'
 'свадьба' 'получение дополнительного образования' 'покупка своего жилья'
 'операции с недвижимостью' 'получение высшего образования'
 'свой автомобиль' 'сделка с автомобилем' 'профильное образование'
 'высшее образование' 'покупка жилья для сдачи' 'на покупку автомобиля'
 'ремонт жилью' 'заняться высшим образованием']
\end{Verbatim}

    \begin{tcolorbox}[breakable, size=fbox, boxrule=1pt, pad at break*=1mm,colback=cellbackground, colframe=cellborder]
\prompt{In}{incolor}{136}{\hspace{4pt}}
\begin{Verbatim}[commandchars=\\\{\}]
\PY{k}{for} \PY{n}{query} \PY{o+ow}{in} \PY{n}{text}\PY{p}{:}
    \PY{k}{if} \PY{l+s+s1}{\PYZsq{}}\PY{l+s+s1}{ремонт}\PY{l+s+s1}{\PYZsq{}} \PY{o+ow}{in} \PY{n}{query}\PY{p}{:}
        \PY{n+nb}{print}\PY{p}{(}\PY{n}{query}\PY{p}{)}
\end{Verbatim}
\end{tcolorbox}

    \begin{Verbatim}[commandchars=\\\{\}]
ремонт жилью
\end{Verbatim}

    ВЫВОД: ремонт жилью уходит в ``потребительский кредит''

    \begin{tcolorbox}[breakable, size=fbox, boxrule=1pt, pad at break*=1mm,colback=cellbackground, colframe=cellborder]
\prompt{In}{incolor}{137}{\hspace{4pt}}
\begin{Verbatim}[commandchars=\\\{\}]
\PY{k}{for} \PY{n}{query} \PY{o+ow}{in} \PY{n}{text}\PY{p}{:}
    \PY{k}{if} \PY{l+s+s1}{\PYZsq{}}\PY{l+s+s1}{сдач}\PY{l+s+s1}{\PYZsq{}} \PY{o+ow}{in} \PY{n}{query}\PY{p}{:}
        \PY{n+nb}{print}\PY{p}{(}\PY{n}{query}\PY{p}{)}
\end{Verbatim}
\end{tcolorbox}

    \begin{Verbatim}[commandchars=\\\{\}]
покупка жилья для сдачи
\end{Verbatim}

    ВЫВОД: удаляем сдачу - она с жильем

    \begin{tcolorbox}[breakable, size=fbox, boxrule=1pt, pad at break*=1mm,colback=cellbackground, colframe=cellborder]
\prompt{In}{incolor}{138}{\hspace{4pt}}
\begin{Verbatim}[commandchars=\\\{\}]
\PY{k}{for} \PY{n}{query} \PY{o+ow}{in} \PY{n}{text}\PY{p}{:}
    \PY{k}{if} \PY{l+s+s1}{\PYZsq{}}\PY{l+s+s1}{семь}\PY{l+s+s1}{\PYZsq{}} \PY{o+ow}{in} \PY{n}{query}\PY{p}{:}
        \PY{n+nb}{print}\PY{p}{(}\PY{n}{query}\PY{p}{)}
\end{Verbatim}
\end{tcolorbox}

    \begin{Verbatim}[commandchars=\\\{\}]
покупка жилья для семьи
\end{Verbatim}

    ВЫВОД: семью удаляем, она с жильем

    \begin{tcolorbox}[breakable, size=fbox, boxrule=1pt, pad at break*=1mm,colback=cellbackground, colframe=cellborder]
\prompt{In}{incolor}{139}{\hspace{4pt}}
\begin{Verbatim}[commandchars=\\\{\}]
\PY{k}{for} \PY{n}{query} \PY{o+ow}{in} \PY{n}{text}\PY{p}{:}
    \PY{k}{if} \PY{l+s+s1}{\PYZsq{}}\PY{l+s+s1}{провед}\PY{l+s+s1}{\PYZsq{}} \PY{o+ow}{in} \PY{n}{query}\PY{p}{:}
        \PY{n+nb}{print}\PY{p}{(}\PY{n}{query}\PY{p}{)}
\end{Verbatim}
\end{tcolorbox}

    \begin{Verbatim}[commandchars=\\\{\}]
на проведение свадьбы
\end{Verbatim}

    ВЫВОД: проведение со свадьбой. удаляем

    \begin{tcolorbox}[breakable, size=fbox, boxrule=1pt, pad at break*=1mm,colback=cellbackground, colframe=cellborder]
\prompt{In}{incolor}{140}{\hspace{4pt}}
\begin{Verbatim}[commandchars=\\\{\}]
\PY{k}{for} \PY{n}{query} \PY{o+ow}{in} \PY{n}{text}\PY{p}{:}
    \PY{k}{if} \PY{l+s+s1}{\PYZsq{}}\PY{l+s+s1}{сыгр}\PY{l+s+s1}{\PYZsq{}} \PY{o+ow}{in} \PY{n}{query}\PY{p}{:}
        \PY{n+nb}{print}\PY{p}{(}\PY{n}{query}\PY{p}{)}
\end{Verbatim}
\end{tcolorbox}

    \begin{Verbatim}[commandchars=\\\{\}]
сыграть свадьбу
\end{Verbatim}

    ВЫВОД: сыграть тоже на свадьбу. удаляем

    \begin{tcolorbox}[breakable, size=fbox, boxrule=1pt, pad at break*=1mm,colback=cellbackground, colframe=cellborder]
\prompt{In}{incolor}{141}{\hspace{4pt}}
\begin{Verbatim}[commandchars=\\\{\}]
\PY{k}{for} \PY{n}{query} \PY{o+ow}{in} \PY{n}{text}\PY{p}{:}
    \PY{k}{if} \PY{l+s+s1}{\PYZsq{}}\PY{l+s+s1}{приобрет}\PY{l+s+s1}{\PYZsq{}} \PY{o+ow}{in} \PY{n}{query}\PY{p}{:}
        \PY{n+nb}{print}\PY{p}{(}\PY{n}{query}\PY{p}{)}
\end{Verbatim}
\end{tcolorbox}

    \begin{Verbatim}[commandchars=\\\{\}]
приобретение автомобиля
\end{Verbatim}

    ВЫВОД: к автомобилю отнесется. удаляем

    \begin{tcolorbox}[breakable, size=fbox, boxrule=1pt, pad at break*=1mm,colback=cellbackground, colframe=cellborder]
\prompt{In}{incolor}{142}{\hspace{4pt}}
\begin{Verbatim}[commandchars=\\\{\}]
\PY{k}{for} \PY{n}{query} \PY{o+ow}{in} \PY{n}{text}\PY{p}{:}
    \PY{k}{if} \PY{l+s+s1}{\PYZsq{}}\PY{l+s+s1}{свад}\PY{l+s+s1}{\PYZsq{}} \PY{o+ow}{in} \PY{n}{query}\PY{p}{:}
        \PY{n+nb}{print}\PY{p}{(}\PY{n}{query}\PY{p}{)}
\end{Verbatim}
\end{tcolorbox}

    \begin{Verbatim}[commandchars=\\\{\}]
сыграть свадьбу
на проведение свадьбы
свадьба
\end{Verbatim}

    ВЫВОД: вот оно зло. в потребительский свадьбу

    \begin{tcolorbox}[breakable, size=fbox, boxrule=1pt, pad at break*=1mm,colback=cellbackground, colframe=cellborder]
\prompt{In}{incolor}{143}{\hspace{4pt}}
\begin{Verbatim}[commandchars=\\\{\}]
\PY{k}{for} \PY{n}{query} \PY{o+ow}{in} \PY{n}{text}\PY{p}{:}
    \PY{k}{if} \PY{l+s+s1}{\PYZsq{}}\PY{l+s+s1}{операц}\PY{l+s+s1}{\PYZsq{}} \PY{o+ow}{in} \PY{n}{query}\PY{p}{:}
        \PY{n+nb}{print}\PY{p}{(}\PY{n}{query}\PY{p}{)}
\end{Verbatim}
\end{tcolorbox}

    \begin{Verbatim}[commandchars=\\\{\}]
операции с жильем
операции с коммерческой недвижимостью
операции со своей недвижимостью
операции с недвижимостью
\end{Verbatim}

    ВЫВОД: во всех вариантах есть и жилье и недвижимость - удаляем. все
пойдет в недвижимость

    \begin{tcolorbox}[breakable, size=fbox, boxrule=1pt, pad at break*=1mm,colback=cellbackground, colframe=cellborder]
\prompt{In}{incolor}{144}{\hspace{4pt}}
\begin{Verbatim}[commandchars=\\\{\}]
\PY{k}{for} \PY{n}{query} \PY{o+ow}{in} \PY{n}{text}\PY{p}{:}
    \PY{k}{if} \PY{l+s+s1}{\PYZsq{}}\PY{l+s+s1}{образова}\PY{l+s+s1}{\PYZsq{}} \PY{o+ow}{in} \PY{n}{query}\PY{p}{:}
        \PY{n+nb}{print}\PY{p}{(}\PY{n}{query}\PY{p}{)}
\end{Verbatim}
\end{tcolorbox}

    \begin{Verbatim}[commandchars=\\\{\}]
дополнительное образование
образование
заняться образованием
получение образования
получение дополнительного образования
получение высшего образования
профильное образование
высшее образование
заняться высшим образованием
\end{Verbatim}

    ВЫВОД: все по учебе. а могло быть образование семьи. хотя и то и то
уходит в потребительский, или, давайте отдельно поместим. интересно,
много за учебу не отдают?

    \begin{tcolorbox}[breakable, size=fbox, boxrule=1pt, pad at break*=1mm,colback=cellbackground, colframe=cellborder]
\prompt{In}{incolor}{145}{\hspace{4pt}}
\begin{Verbatim}[commandchars=\\\{\}]
\PY{k}{for} \PY{n}{query} \PY{o+ow}{in} \PY{n}{text}\PY{p}{:}
    \PY{k}{if} \PY{l+s+s1}{\PYZsq{}}\PY{l+s+s1}{автом}\PY{l+s+s1}{\PYZsq{}} \PY{o+ow}{in} \PY{n}{query}\PY{p}{:}
        \PY{n+nb}{print}\PY{p}{(}\PY{n}{query}\PY{p}{)}
\end{Verbatim}
\end{tcolorbox}

    \begin{Verbatim}[commandchars=\\\{\}]
приобретение автомобиля
на покупку подержанного автомобиля
на покупку своего автомобиля
автомобили
сделка с подержанным автомобилем
автомобиль
свой автомобиль
сделка с автомобилем
на покупку автомобиля
\end{Verbatim}

    ВЫВОД: формулировок с авто гораздо больше. Запишем в ЛИЗИНГ, хотя это и
не совсем верно

    \begin{tcolorbox}[breakable, size=fbox, boxrule=1pt, pad at break*=1mm,colback=cellbackground, colframe=cellborder]
\prompt{In}{incolor}{146}{\hspace{4pt}}
\begin{Verbatim}[commandchars=\\\{\}]
\PY{k}{for} \PY{n}{query} \PY{o+ow}{in} \PY{n}{text}\PY{p}{:}
    \PY{k}{if} \PY{l+s+s1}{\PYZsq{}}\PY{l+s+s1}{строит}\PY{l+s+s1}{\PYZsq{}} \PY{o+ow}{in} \PY{n}{query}\PY{p}{:}
        \PY{n+nb}{print}\PY{p}{(}\PY{n}{query}\PY{p}{)}        
\end{Verbatim}
\end{tcolorbox}

    \begin{Verbatim}[commandchars=\\\{\}]
строительство собственной недвижимости
строительство недвижимости
строительство жилой недвижимости
\end{Verbatim}

    ВЫВОД: во всех вариантах строительства есть жилье и недвижимость.
удаляем

    \begin{tcolorbox}[breakable, size=fbox, boxrule=1pt, pad at break*=1mm,colback=cellbackground, colframe=cellborder]
\prompt{In}{incolor}{147}{\hspace{4pt}}
\begin{Verbatim}[commandchars=\\\{\}]
\PY{k}{for} \PY{n}{query} \PY{o+ow}{in} \PY{n}{text}\PY{p}{:}
    \PY{k}{if} \PY{l+s+s1}{\PYZsq{}}\PY{l+s+s1}{жил}\PY{l+s+s1}{\PYZsq{}} \PY{o+ow}{in} \PY{n}{query}\PY{p}{:}
        \PY{n+nb}{print}\PY{p}{(}\PY{n}{query}\PY{p}{)}
\end{Verbatim}
\end{tcolorbox}

    \begin{Verbatim}[commandchars=\\\{\}]
покупка жилья
операции с жильем
покупка жилья для семьи
покупка жилой недвижимости
строительство жилой недвижимости
жилье
покупка своего жилья
покупка жилья для сдачи
ремонт жилью
\end{Verbatim}

    ВЫВОД: с жильем все хорошо, кроме ремонта и операций. под операциями,
вероятно имеются в виду сделки. А вот ремнт - совершенно другая
категория. Поэтому при группировке в категории сначала нужно будет
исправить ремонт, а после жилье

    \begin{tcolorbox}[breakable, size=fbox, boxrule=1pt, pad at break*=1mm,colback=cellbackground, colframe=cellborder]
\prompt{In}{incolor}{249}{\hspace{4pt}}
\begin{Verbatim}[commandchars=\\\{\}]
\PY{k}{for} \PY{n}{query} \PY{o+ow}{in} \PY{n}{text}\PY{p}{:}
    \PY{k}{if} \PY{l+s+s1}{\PYZsq{}}\PY{l+s+s1}{недвиж}\PY{l+s+s1}{\PYZsq{}} \PY{o+ow}{in} \PY{n}{query}\PY{p}{:}
        \PY{n+nb}{print}\PY{p}{(}\PY{n}{query}\PY{p}{)}
\end{Verbatim}
\end{tcolorbox}

    \begin{Verbatim}[commandchars=\\\{\}]
покупка недвижимости
покупка коммерческой недвижимости
покупка жилой недвижимости
строительство собственной недвижимости
недвижимость
строительство недвижимости
операции с коммерческой недвижимостью
строительство жилой недвижимости
операции со своей недвижимостью
операции с недвижимостью
\end{Verbatim}

    ВЫВОД: тут вск ОК, недвижимость она и в африке недвижимость

    \begin{tcolorbox}[breakable, size=fbox, boxrule=1pt, pad at break*=1mm,colback=cellbackground, colframe=cellborder]
\prompt{In}{incolor}{250}{\hspace{4pt}}
\begin{Verbatim}[commandchars=\\\{\}]
\PY{k}{for} \PY{n}{query} \PY{o+ow}{in} \PY{n}{text}\PY{p}{:}
    \PY{k}{if} \PY{l+s+s1}{\PYZsq{}}\PY{l+s+s1}{покуп}\PY{l+s+s1}{\PYZsq{}} \PY{o+ow}{in} \PY{n}{query}\PY{p}{:}
        \PY{n+nb}{print}\PY{p}{(}\PY{n}{query}\PY{p}{)}
\end{Verbatim}
\end{tcolorbox}

    \begin{Verbatim}[commandchars=\\\{\}]
покупка жилья
покупка жилья для семьи
покупка недвижимости
покупка коммерческой недвижимости
покупка жилой недвижимости
на покупку подержанного автомобиля
на покупку своего автомобиля
покупка своего жилья
покупка жилья для сдачи
на покупку автомобиля
\end{Verbatim}

    ВЫВОД: в покупке все недвижимость, жилье и автомобили. Все это было.
Можно удалять после нехитрых телодвижений у нас остались 4 категории
(можно было и 3) и 5 ключевых слов 1. НЕДВИЖИМОСТЬ: `недвижимость',
`жилье' 2. ЛИЗИНГ: `автомобиль' 3. ОБРАЗОВАНИЕ: `образование' 4.
ПОТРЕБИТЕЛЬСКИЙ: `ремонт', `свадьба'

сейчас начинаем волшебство

    \begin{tcolorbox}[breakable, size=fbox, boxrule=1pt, pad at break*=1mm,colback=cellbackground, colframe=cellborder]
\prompt{In}{incolor}{251}{\hspace{4pt}}
\begin{Verbatim}[commandchars=\\\{\}]
\PY{k}{def} \PY{n+nf}{name\PYZus{}cat}\PY{p}{(}\PY{n}{x}\PY{p}{)}\PY{p}{:}
    \PY{k}{if} \PY{l+s+s1}{\PYZsq{}}\PY{l+s+s1}{свад}\PY{l+s+s1}{\PYZsq{}} \PY{o+ow}{in} \PY{n}{x}\PY{p}{:}
        \PY{k}{return} \PY{l+s+s1}{\PYZsq{}}\PY{l+s+s1}{потребительский}\PY{l+s+s1}{\PYZsq{}}
    \PY{k}{if} \PY{l+s+s1}{\PYZsq{}}\PY{l+s+s1}{ремо}\PY{l+s+s1}{\PYZsq{}} \PY{o+ow}{in} \PY{n}{x}\PY{p}{:}
        \PY{k}{return} \PY{l+s+s1}{\PYZsq{}}\PY{l+s+s1}{потребительский}\PY{l+s+s1}{\PYZsq{}}
    \PY{k}{if} \PY{l+s+s1}{\PYZsq{}}\PY{l+s+s1}{образов}\PY{l+s+s1}{\PYZsq{}} \PY{o+ow}{in} \PY{n}{x}\PY{p}{:}
        \PY{k}{return} \PY{l+s+s1}{\PYZsq{}}\PY{l+s+s1}{образование}\PY{l+s+s1}{\PYZsq{}}
    \PY{k}{if} \PY{l+s+s1}{\PYZsq{}}\PY{l+s+s1}{авто}\PY{l+s+s1}{\PYZsq{}} \PY{o+ow}{in} \PY{n}{x}\PY{p}{:}
        \PY{k}{return} \PY{l+s+s1}{\PYZsq{}}\PY{l+s+s1}{лизинг}\PY{l+s+s1}{\PYZsq{}}
    \PY{k}{if} \PY{l+s+s1}{\PYZsq{}}\PY{l+s+s1}{недвиж}\PY{l+s+s1}{\PYZsq{}} \PY{o+ow}{in} \PY{n}{x}\PY{p}{:}
        \PY{k}{return} \PY{l+s+s1}{\PYZsq{}}\PY{l+s+s1}{недвижимость}\PY{l+s+s1}{\PYZsq{}}
    \PY{k}{if} \PY{l+s+s1}{\PYZsq{}}\PY{l+s+s1}{жил}\PY{l+s+s1}{\PYZsq{}} \PY{o+ow}{in} \PY{n}{x}\PY{p}{:}
        \PY{k}{return} \PY{l+s+s1}{\PYZsq{}}\PY{l+s+s1}{недвижимость}\PY{l+s+s1}{\PYZsq{}}
    \PY{k}{else}\PY{p}{:}
        \PY{k}{return} \PY{l+s+s1}{\PYZsq{}}\PY{l+s+s1}{хз}\PY{l+s+s1}{\PYZsq{}}

\PY{n}{bank\PYZus{}clients}\PY{p}{[}\PY{l+s+s1}{\PYZsq{}}\PY{l+s+s1}{b\PYZus{}d}\PY{l+s+s1}{\PYZsq{}}\PY{p}{]} \PY{o}{=} \PY{n}{bank\PYZus{}clients}\PY{p}{[}\PY{l+s+s1}{\PYZsq{}}\PY{l+s+s1}{purpose}\PY{l+s+s1}{\PYZsq{}}\PY{p}{]}\PY{o}{.}\PY{n}{apply}\PY{p}{(}\PY{n}{name\PYZus{}cat}\PY{p}{)}

\PY{n+nb}{print}\PY{p}{(}\PY{n}{bank\PYZus{}clients}\PY{o}{.}\PY{n}{groupby}\PY{p}{(}\PY{l+s+s1}{\PYZsq{}}\PY{l+s+s1}{b\PYZus{}d}\PY{l+s+s1}{\PYZsq{}}\PY{p}{)}\PY{p}{[}\PY{l+s+s1}{\PYZsq{}}\PY{l+s+s1}{b\PYZus{}d}\PY{l+s+s1}{\PYZsq{}}\PY{p}{]}\PY{o}{.}\PY{n}{count}\PY{p}{(}\PY{p}{)}\PY{p}{)}
\end{Verbatim}
\end{tcolorbox}

    \begin{Verbatim}[commandchars=\\\{\}]
b\_d
лизинг              4308
недвижимость       10207
образование         4014
потребительский     2942
Name: b\_d, dtype: int64
\end{Verbatim}

    \begin{tcolorbox}[breakable, size=fbox, boxrule=1pt, pad at break*=1mm,colback=cellbackground, colframe=cellborder]
\prompt{In}{incolor}{252}{\hspace{4pt}}
\begin{Verbatim}[commandchars=\\\{\}]
\PY{n}{x} \PY{o}{=} \PY{n}{bank\PYZus{}clients}\PY{o}{.}\PY{n}{duplicated}\PY{p}{(}\PY{p}{)}\PY{o}{.}\PY{n}{sum}\PY{p}{(}\PY{p}{)}
\PY{n+nb}{print}\PY{p}{(}\PY{l+s+s1}{\PYZsq{}}\PY{l+s+s1}{На данный момент дубликатов}\PY{l+s+s1}{\PYZsq{}}\PY{p}{,} \PY{n}{x}\PY{p}{)}
\end{Verbatim}
\end{tcolorbox}

    \begin{Verbatim}[commandchars=\\\{\}]
На данный момент дубликатов 21
\end{Verbatim}

    \hypertarget{ux432ux44bux432ux43eux434}{%
\subsubsection{Вывод}\label{ux432ux44bux432ux43eux434}}

    То, что могло занять 20 минут, случилось за 220! А по таблице - около
половины выдаваемых кредитов с банке на недвижимость. Фантастика. Есть
шанс, что проценты по этому кедиту меньше

    Комментарий наставника

Леммы выделены верно.

Категоризовать также надо и столбец с доходами. Это есть ниже, а должно
быть в данном разделе. Он для категоризации и создан.

    \hypertarget{ux448ux430ux433-3.-ux43eux442ux432ux435ux442ux44cux442ux435-ux43dux430-ux432ux43eux43fux440ux43eux441ux44b}{%
\subsubsection{Шаг 3. Ответьте на
вопросы}\label{ux448ux430ux433-3.-ux43eux442ux432ux435ux442ux44cux442ux435-ux43dux430-ux432ux43eux43fux440ux43eux441ux44b}}

    \begin{itemize}
\tightlist
\item
  Есть ли зависимость между наличием детей и возвратом кредита в срок?
\end{itemize}

    \begin{tcolorbox}[breakable, size=fbox, boxrule=1pt, pad at break*=1mm,colback=cellbackground, colframe=cellborder]
\prompt{In}{incolor}{253}{\hspace{4pt}}
\begin{Verbatim}[commandchars=\\\{\}]
\PY{n}{deti} \PY{o}{=} \PY{n}{bank\PYZus{}clients}\PY{o}{.}\PY{n}{groupby}\PY{p}{(}\PY{l+s+s1}{\PYZsq{}}\PY{l+s+s1}{children}\PY{l+s+s1}{\PYZsq{}}\PY{p}{)}\PY{p}{[}\PY{l+s+s1}{\PYZsq{}}\PY{l+s+s1}{debt}\PY{l+s+s1}{\PYZsq{}}\PY{p}{]}\PY{o}{.}\PY{n}{value\PYZus{}counts}\PY{p}{(}\PY{p}{)}
\PY{n+nb}{print}\PY{p}{(}\PY{n}{deti}\PY{p}{)}
\end{Verbatim}
\end{tcolorbox}

    \begin{Verbatim}[commandchars=\\\{\}]
children  debt
0         0       13044
          1        1063
1         0        4411
          1         445
2         0        1926
          1         202
5         0         349
          1          31
Name: debt, dtype: int64
\end{Verbatim}

    Получилась таблица с интересущими нас данными, но как теперь добраться
до нужных нам ячеек? Пока оставим этот вопрос, потому как 3,4 и 5 детные
люди не по своей численности составляют около 2х процентов. Объединим их
в группу многоденых

    \begin{tcolorbox}[breakable, size=fbox, boxrule=1pt, pad at break*=1mm,colback=cellbackground, colframe=cellborder]
\prompt{In}{incolor}{254}{\hspace{4pt}}
\begin{Verbatim}[commandchars=\\\{\}]
\PY{k}{def} \PY{n+nf}{more\PYZus{}children}\PY{p}{(}\PY{n}{x}\PY{p}{)}\PY{p}{:}
    \PY{k}{if} \PY{n}{x} \PY{o}{\PYZgt{}} \PY{l+m+mi}{2}\PY{p}{:}
        \PY{k}{return} \PY{l+m+mi}{5}
    \PY{k}{return} \PY{n}{x}
\PY{n}{bank\PYZus{}clients}\PY{p}{[}\PY{l+s+s1}{\PYZsq{}}\PY{l+s+s1}{children}\PY{l+s+s1}{\PYZsq{}}\PY{p}{]} \PY{o}{=} \PY{n}{bank\PYZus{}clients}\PY{p}{[}\PY{l+s+s1}{\PYZsq{}}\PY{l+s+s1}{children}\PY{l+s+s1}{\PYZsq{}}\PY{p}{]}\PY{o}{.}\PY{n}{apply}\PY{p}{(}\PY{n}{more\PYZus{}children}\PY{p}{)}
\PY{n}{deti} \PY{o}{=} \PY{n}{bank\PYZus{}clients}\PY{o}{.}\PY{n}{groupby}\PY{p}{(}\PY{l+s+s1}{\PYZsq{}}\PY{l+s+s1}{children}\PY{l+s+s1}{\PYZsq{}}\PY{p}{)}\PY{p}{[}\PY{l+s+s1}{\PYZsq{}}\PY{l+s+s1}{children}\PY{l+s+s1}{\PYZsq{}}\PY{p}{]}\PY{o}{.}\PY{n}{value\PYZus{}counts}\PY{p}{(}\PY{p}{)}
\PY{n+nb}{print}\PY{p}{(}\PY{n}{deti}\PY{p}{)}
\end{Verbatim}
\end{tcolorbox}

    \begin{Verbatim}[commandchars=\\\{\}]
children  children
0         0           14107
1         1            4856
2         2            2128
5         5             380
Name: children, dtype: int64
\end{Verbatim}

    \begin{tcolorbox}[breakable, size=fbox, boxrule=1pt, pad at break*=1mm,colback=cellbackground, colframe=cellborder]
\prompt{In}{incolor}{255}{\hspace{4pt}}
\begin{Verbatim}[commandchars=\\\{\}]
\PY{n}{deti} \PY{o}{=} \PY{n}{bank\PYZus{}clients}\PY{o}{.}\PY{n}{loc}\PY{p}{[}\PY{p}{:}\PY{p}{,}\PY{p}{(}\PY{l+s+s1}{\PYZsq{}}\PY{l+s+s1}{children}\PY{l+s+s1}{\PYZsq{}}\PY{p}{,}\PY{l+s+s1}{\PYZsq{}}\PY{l+s+s1}{debt}\PY{l+s+s1}{\PYZsq{}}\PY{p}{)}\PY{p}{]}
\PY{n+nb}{print}\PY{p}{(}\PY{n}{deti}\PY{p}{)}
\end{Verbatim}
\end{tcolorbox}

    \begin{Verbatim}[commandchars=\\\{\}]
       children  debt
0             1     0
1             1     0
2             0     0
3             5     0
4             0     0
{\ldots}         {\ldots}   {\ldots}
21466         1     0
21467         0     0
21468         1     1
21469         5     1
21470         2     0

[21471 rows x 2 columns]
\end{Verbatim}

    \begin{tcolorbox}[breakable, size=fbox, boxrule=1pt, pad at break*=1mm,colback=cellbackground, colframe=cellborder]
\prompt{In}{incolor}{256}{\hspace{4pt}}
\begin{Verbatim}[commandchars=\\\{\}]
\PY{n}{deti\PYZus{}grouped\PYZus{}sum} \PY{o}{=} \PY{n}{deti}\PY{o}{.}\PY{n}{groupby}\PY{p}{(}\PY{l+s+s1}{\PYZsq{}}\PY{l+s+s1}{children}\PY{l+s+s1}{\PYZsq{}}\PY{p}{)}\PY{o}{.}\PY{n}{sum}\PY{p}{(}\PY{p}{)}
\PY{n}{deti\PYZus{}grouped\PYZus{}count} \PY{o}{=} \PY{n}{deti}\PY{o}{.}\PY{n}{groupby}\PY{p}{(}\PY{l+s+s1}{\PYZsq{}}\PY{l+s+s1}{children}\PY{l+s+s1}{\PYZsq{}}\PY{p}{)}\PY{o}{.}\PY{n}{count}\PY{p}{(}\PY{p}{)}
\PY{n}{deti} \PY{o}{=} \PY{n}{pd}\PY{o}{.}\PY{n}{merge}\PY{p}{(}\PY{n}{deti\PYZus{}grouped\PYZus{}sum}\PY{p}{,} \PY{n}{deti\PYZus{}grouped\PYZus{}count}\PY{p}{,} \PY{n}{on}\PY{o}{=}\PY{p}{(}\PY{l+s+s1}{\PYZsq{}}\PY{l+s+s1}{children}\PY{l+s+s1}{\PYZsq{}}\PY{p}{)}\PY{p}{)}
\end{Verbatim}
\end{tcolorbox}

    посчитали общее кол-во людей и должников и объединили таблицы

    \begin{tcolorbox}[breakable, size=fbox, boxrule=1pt, pad at break*=1mm,colback=cellbackground, colframe=cellborder]
\prompt{In}{incolor}{257}{\hspace{4pt}}
\begin{Verbatim}[commandchars=\\\{\}]
\PY{n}{deti}\PY{p}{[}\PY{l+s+s1}{\PYZsq{}}\PY{l+s+s1}{total}\PY{l+s+s1}{\PYZsq{}}\PY{p}{]} \PY{o}{=} \PY{n}{deti}\PY{p}{[}\PY{l+s+s1}{\PYZsq{}}\PY{l+s+s1}{debt\PYZus{}x}\PY{l+s+s1}{\PYZsq{}}\PY{p}{]} \PY{o}{/} \PY{n}{deti}\PY{p}{[}\PY{l+s+s1}{\PYZsq{}}\PY{l+s+s1}{debt\PYZus{}y}\PY{l+s+s1}{\PYZsq{}}\PY{p}{]}
\end{Verbatim}
\end{tcolorbox}

    \begin{tcolorbox}[breakable, size=fbox, boxrule=1pt, pad at break*=1mm,colback=cellbackground, colframe=cellborder]
\prompt{In}{incolor}{258}{\hspace{4pt}}
\begin{Verbatim}[commandchars=\\\{\}]
\PY{n}{deti}\PY{o}{.}\PY{n}{drop}\PY{p}{(}\PY{p}{[}\PY{l+s+s1}{\PYZsq{}}\PY{l+s+s1}{debt\PYZus{}x}\PY{l+s+s1}{\PYZsq{}}\PY{p}{,}\PY{l+s+s1}{\PYZsq{}}\PY{l+s+s1}{debt\PYZus{}y}\PY{l+s+s1}{\PYZsq{}}\PY{p}{]}\PY{p}{,} \PY{n}{axis}\PY{o}{=}\PY{l+s+s1}{\PYZsq{}}\PY{l+s+s1}{columns}\PY{l+s+s1}{\PYZsq{}}\PY{p}{,} \PY{n}{inplace}\PY{o}{=}\PY{k+kc}{True}\PY{p}{)}
\PY{n+nb}{print}\PY{p}{(}\PY{n}{deti}\PY{p}{)}
\end{Verbatim}
\end{tcolorbox}

    \begin{Verbatim}[commandchars=\\\{\}]
             total
children
0         0.075353
1         0.091639
2         0.094925
5         0.081579
\end{Verbatim}

    Посчитал проценты

    \begin{tcolorbox}[breakable, size=fbox, boxrule=1pt, pad at break*=1mm,colback=cellbackground, colframe=cellborder]
\prompt{In}{incolor}{259}{\hspace{4pt}}
\begin{Verbatim}[commandchars=\\\{\}]
\PY{k}{def} \PY{n+nf}{percent}\PY{p}{(}\PY{n}{x}\PY{p}{)}\PY{p}{:}
    \PY{n}{fin} \PY{o}{=} \PY{l+s+s1}{\PYZsq{}}\PY{l+s+si}{\PYZob{}:.2\PYZpc{}\PYZcb{}}\PY{l+s+s1}{\PYZsq{}}\PY{o}{.}\PY{n}{format}\PY{p}{(}\PY{n}{x}\PY{p}{)}
    \PY{k}{return} \PY{n}{fin}
\PY{n}{deti}\PY{p}{[}\PY{l+s+s1}{\PYZsq{}}\PY{l+s+s1}{total}\PY{l+s+s1}{\PYZsq{}}\PY{p}{]} \PY{o}{=} \PY{n}{deti}\PY{p}{[}\PY{l+s+s1}{\PYZsq{}}\PY{l+s+s1}{total}\PY{l+s+s1}{\PYZsq{}}\PY{p}{]}\PY{o}{.}\PY{n}{apply}\PY{p}{(}\PY{n}{percent}\PY{p}{)}
\end{Verbatim}
\end{tcolorbox}

    \begin{tcolorbox}[breakable, size=fbox, boxrule=1pt, pad at break*=1mm,colback=cellbackground, colframe=cellborder]
\prompt{In}{incolor}{260}{\hspace{4pt}}
\begin{Verbatim}[commandchars=\\\{\}]
\PY{n+nb}{print}\PY{p}{(}\PY{n}{deti}\PY{p}{)}
\end{Verbatim}
\end{tcolorbox}

    \begin{Verbatim}[commandchars=\\\{\}]
          total
children
0         7.54\%
1         9.16\%
2         9.49\%
5         8.16\%
\end{Verbatim}

    \hypertarget{ux432ux44bux432ux43eux434}{%
\subsubsection{Вывод}\label{ux432ux44bux432ux43eux434}}

    Вероятность возврата кредита с появлением детей снижается. Люди с 2
детьми имеют большее количество задолженностей перед банком Вопреки
ожиданиям ``многодетные'' семьи возвращают кредиты охотнее

    Комментарий наставника

Категоризовать данные по числу детей необязательно. Разбиение есть уже в
исходных данных. Все-таки разница в возврате кредита у семей с одним
ребенком или тремя детьми будет.

    DONE

    \begin{itemize}
\tightlist
\item
  Есть ли зависимость между семейным положением и возвратом кредита в
  срок?
\end{itemize}

    По накатанной: группируем и выводим таблицу

    \begin{tcolorbox}[breakable, size=fbox, boxrule=1pt, pad at break*=1mm,colback=cellbackground, colframe=cellborder]
\prompt{In}{incolor}{261}{\hspace{4pt}}
\begin{Verbatim}[commandchars=\\\{\}]
\PY{n}{family} \PY{o}{=} \PY{n}{bank\PYZus{}clients}\PY{o}{.}\PY{n}{loc}\PY{p}{[}\PY{p}{:}\PY{p}{,}\PY{p}{(}\PY{l+s+s1}{\PYZsq{}}\PY{l+s+s1}{family\PYZus{}status}\PY{l+s+s1}{\PYZsq{}}\PY{p}{,}\PY{l+s+s1}{\PYZsq{}}\PY{l+s+s1}{debt}\PY{l+s+s1}{\PYZsq{}}\PY{p}{)}\PY{p}{]}
\PY{n}{family\PYZus{}sum} \PY{o}{=} \PY{n}{family}\PY{o}{.}\PY{n}{groupby}\PY{p}{(}\PY{l+s+s1}{\PYZsq{}}\PY{l+s+s1}{family\PYZus{}status}\PY{l+s+s1}{\PYZsq{}}\PY{p}{)}\PY{o}{.}\PY{n}{sum}\PY{p}{(}\PY{p}{)}
\PY{n}{family\PYZus{}count} \PY{o}{=} \PY{n}{family}\PY{o}{.}\PY{n}{groupby}\PY{p}{(}\PY{l+s+s1}{\PYZsq{}}\PY{l+s+s1}{family\PYZus{}status}\PY{l+s+s1}{\PYZsq{}}\PY{p}{)}\PY{o}{.}\PY{n}{count}\PY{p}{(}\PY{p}{)}
\PY{n}{family} \PY{o}{=} \PY{n}{pd}\PY{o}{.}\PY{n}{merge}\PY{p}{(}\PY{n}{family\PYZus{}sum}\PY{p}{,} \PY{n}{family\PYZus{}count}\PY{p}{,} \PY{n}{on}\PY{o}{=}\PY{p}{(}\PY{l+s+s1}{\PYZsq{}}\PY{l+s+s1}{family\PYZus{}status}\PY{l+s+s1}{\PYZsq{}}\PY{p}{)}\PY{p}{)}
\PY{n}{family}\PY{p}{[}\PY{l+s+s1}{\PYZsq{}}\PY{l+s+s1}{total}\PY{l+s+s1}{\PYZsq{}}\PY{p}{]} \PY{o}{=} \PY{n}{family}\PY{p}{[}\PY{l+s+s1}{\PYZsq{}}\PY{l+s+s1}{debt\PYZus{}x}\PY{l+s+s1}{\PYZsq{}}\PY{p}{]} \PY{o}{/} \PY{n}{family}\PY{p}{[}\PY{l+s+s1}{\PYZsq{}}\PY{l+s+s1}{debt\PYZus{}y}\PY{l+s+s1}{\PYZsq{}}\PY{p}{]}

\PY{k}{def} \PY{n+nf}{percent}\PY{p}{(}\PY{n}{x}\PY{p}{)}\PY{p}{:}
    \PY{n}{fin} \PY{o}{=} \PY{l+s+s1}{\PYZsq{}}\PY{l+s+si}{\PYZob{}:.2\PYZpc{}\PYZcb{}}\PY{l+s+s1}{\PYZsq{}}\PY{o}{.}\PY{n}{format}\PY{p}{(}\PY{n}{x}\PY{p}{)}
    \PY{k}{return} \PY{n}{fin}
\PY{n}{family}\PY{p}{[}\PY{l+s+s1}{\PYZsq{}}\PY{l+s+s1}{total}\PY{l+s+s1}{\PYZsq{}}\PY{p}{]} \PY{o}{=} \PY{n}{family}\PY{p}{[}\PY{l+s+s1}{\PYZsq{}}\PY{l+s+s1}{total}\PY{l+s+s1}{\PYZsq{}}\PY{p}{]}\PY{o}{.}\PY{n}{apply}\PY{p}{(}\PY{n}{percent}\PY{p}{)}
\PY{n}{family}\PY{o}{.}\PY{n}{drop}\PY{p}{(}\PY{p}{[}\PY{l+s+s1}{\PYZsq{}}\PY{l+s+s1}{debt\PYZus{}x}\PY{l+s+s1}{\PYZsq{}}\PY{p}{,}\PY{l+s+s1}{\PYZsq{}}\PY{l+s+s1}{debt\PYZus{}y}\PY{l+s+s1}{\PYZsq{}}\PY{p}{]}\PY{p}{,} \PY{n}{axis}\PY{o}{=}\PY{l+s+s1}{\PYZsq{}}\PY{l+s+s1}{columns}\PY{l+s+s1}{\PYZsq{}}\PY{p}{,} \PY{n}{inplace}\PY{o}{=}\PY{k+kc}{True}\PY{p}{)}

\PY{n}{family}
\end{Verbatim}
\end{tcolorbox}

            \begin{tcolorbox}[breakable, boxrule=.5pt, size=fbox, pad at break*=1mm, opacityfill=0]
\prompt{Out}{outcolor}{261}{\hspace{3.5pt}}
\begin{Verbatim}[commandchars=\\\{\}]
                       total
family\_status
гражданский брак       9.32\%
женат / замужем        7.54\%
не женат / не замужем  8.50\%
\end{Verbatim}
\end{tcolorbox}
        
    \hypertarget{ux432ux44bux432ux43eux434}{%
\subsubsection{Вывод}\label{ux432ux44bux432ux43eux434}}

    Процент долгов у семейных людей ниже. Зависимость есть.

    Комментарий наставника

Вывод не противоречит полученному результату.

    \begin{itemize}
\tightlist
\item
  Есть ли зависимость между уровнем дохода и возвратом кредита в срок?
\end{itemize}

    \begin{tcolorbox}[breakable, size=fbox, boxrule=1pt, pad at break*=1mm,colback=cellbackground, colframe=cellborder]
\prompt{In}{incolor}{262}{\hspace{4pt}}
\begin{Verbatim}[commandchars=\\\{\}]
\PY{n}{incom} \PY{o}{=} \PY{n}{bank\PYZus{}clients}\PY{o}{.}\PY{n}{loc}\PY{p}{[}\PY{p}{:}\PY{p}{,}\PY{p}{(}\PY{l+s+s1}{\PYZsq{}}\PY{l+s+s1}{total\PYZus{}income}\PY{l+s+s1}{\PYZsq{}}\PY{p}{,}\PY{l+s+s1}{\PYZsq{}}\PY{l+s+s1}{debt}\PY{l+s+s1}{\PYZsq{}}\PY{p}{)}\PY{p}{]}

\PY{n}{incom}\PY{p}{[}\PY{l+s+s1}{\PYZsq{}}\PY{l+s+s1}{total\PYZus{}income}\PY{l+s+s1}{\PYZsq{}}\PY{p}{]} \PY{o}{=} \PY{n}{incom}\PY{p}{[}\PY{l+s+s1}{\PYZsq{}}\PY{l+s+s1}{total\PYZus{}income}\PY{l+s+s1}{\PYZsq{}}\PY{p}{]}\PY{o}{.}\PY{n}{astype}\PY{p}{(}\PY{l+s+s1}{\PYZsq{}}\PY{l+s+s1}{int}\PY{l+s+s1}{\PYZsq{}}\PY{p}{)}
\PY{c+c1}{\PYZsh{}уменьшили цифры для читабельности и сгруппируем по зараоткам людей}


\PY{k}{def} \PY{n+nf}{salary\PYZus{}count}\PY{p}{(}\PY{n}{incom}\PY{p}{)}\PY{p}{:}
    \PY{k}{if} \PY{n}{incom} \PY{o}{\PYZlt{}} \PY{l+m+mi}{50000}\PY{p}{:}
        \PY{k}{return} \PY{l+s+s1}{\PYZsq{}}\PY{l+s+s1}{a. до 50}\PY{l+s+s1}{\PYZsq{}}
    \PY{k}{if} \PY{n}{incom} \PY{o}{\PYZlt{}} \PY{l+m+mi}{100000}\PY{p}{:}
        \PY{k}{return} \PY{l+s+s1}{\PYZsq{}}\PY{l+s+s1}{b. 50\PYZhy{}100}\PY{l+s+s1}{\PYZsq{}}
    \PY{k}{if} \PY{n}{incom} \PY{o}{\PYZlt{}} \PY{l+m+mi}{200000}\PY{p}{:}
        \PY{k}{return} \PY{l+s+s1}{\PYZsq{}}\PY{l+s+s1}{c. 100\PYZhy{}200}\PY{l+s+s1}{\PYZsq{}}
    \PY{k}{return} \PY{l+s+s1}{\PYZsq{}}\PY{l+s+s1}{d. 200+}\PY{l+s+s1}{\PYZsq{}}

\PY{n}{incom}\PY{p}{[}\PY{l+s+s1}{\PYZsq{}}\PY{l+s+s1}{less\PYZus{}000}\PY{l+s+s1}{\PYZsq{}}\PY{p}{]} \PY{o}{=} \PY{n}{incom}\PY{p}{[}\PY{l+s+s1}{\PYZsq{}}\PY{l+s+s1}{total\PYZus{}income}\PY{l+s+s1}{\PYZsq{}}\PY{p}{]}\PY{o}{.}\PY{n}{apply}\PY{p}{(}\PY{n}{salary\PYZus{}count}\PY{p}{)}
\PY{c+c1}{\PYZsh{}incom[\PYZsq{}less\PYZus{}000\PYZsq{}] = incom[\PYZsq{}less\PYZus{}000\PYZsq{}].astype(\PYZsq{}int\PYZsq{})}
\PY{n}{incom}\PY{o}{.}\PY{n}{sort\PYZus{}values}\PY{p}{(}\PY{n}{by}\PY{o}{=}\PY{l+s+s1}{\PYZsq{}}\PY{l+s+s1}{less\PYZus{}000}\PY{l+s+s1}{\PYZsq{}}\PY{p}{)}


\PY{n}{incom\PYZus{}sum} \PY{o}{=} \PY{n}{incom}\PY{o}{.}\PY{n}{groupby}\PY{p}{(}\PY{l+s+s1}{\PYZsq{}}\PY{l+s+s1}{less\PYZus{}000}\PY{l+s+s1}{\PYZsq{}}\PY{p}{)}\PY{o}{.}\PY{n}{sum}\PY{p}{(}\PY{p}{)}
\PY{n}{incom\PYZus{}count} \PY{o}{=} \PY{n}{incom}\PY{o}{.}\PY{n}{groupby}\PY{p}{(}\PY{l+s+s1}{\PYZsq{}}\PY{l+s+s1}{less\PYZus{}000}\PY{l+s+s1}{\PYZsq{}}\PY{p}{)}\PY{o}{.}\PY{n}{count}\PY{p}{(}\PY{p}{)}
\PY{n}{incom} \PY{o}{=} \PY{n}{pd}\PY{o}{.}\PY{n}{merge}\PY{p}{(}\PY{n}{incom\PYZus{}sum}\PY{p}{,} \PY{n}{incom\PYZus{}count}\PY{p}{,} \PY{n}{on}\PY{o}{=}\PY{p}{(}\PY{l+s+s1}{\PYZsq{}}\PY{l+s+s1}{less\PYZus{}000}\PY{l+s+s1}{\PYZsq{}}\PY{p}{)}\PY{p}{)}


\PY{n}{incom}\PY{p}{[}\PY{l+s+s1}{\PYZsq{}}\PY{l+s+s1}{total}\PY{l+s+s1}{\PYZsq{}}\PY{p}{]} \PY{o}{=} \PY{n}{incom}\PY{p}{[}\PY{l+s+s1}{\PYZsq{}}\PY{l+s+s1}{debt\PYZus{}x}\PY{l+s+s1}{\PYZsq{}}\PY{p}{]} \PY{o}{/} \PY{n}{incom}\PY{p}{[}\PY{l+s+s1}{\PYZsq{}}\PY{l+s+s1}{debt\PYZus{}y}\PY{l+s+s1}{\PYZsq{}}\PY{p}{]}

\PY{k}{def} \PY{n+nf}{percent}\PY{p}{(}\PY{n}{x}\PY{p}{)}\PY{p}{:}
    \PY{n}{fin} \PY{o}{=} \PY{l+s+s1}{\PYZsq{}}\PY{l+s+si}{\PYZob{}:.2\PYZpc{}\PYZcb{}}\PY{l+s+s1}{\PYZsq{}}\PY{o}{.}\PY{n}{format}\PY{p}{(}\PY{n}{x}\PY{p}{)}
    \PY{k}{return} \PY{n}{fin}
\PY{n}{incom}\PY{p}{[}\PY{l+s+s1}{\PYZsq{}}\PY{l+s+s1}{total}\PY{l+s+s1}{\PYZsq{}}\PY{p}{]} \PY{o}{=} \PY{n}{incom}\PY{p}{[}\PY{l+s+s1}{\PYZsq{}}\PY{l+s+s1}{total}\PY{l+s+s1}{\PYZsq{}}\PY{p}{]}\PY{o}{.}\PY{n}{apply}\PY{p}{(}\PY{n}{percent}\PY{p}{)}
\PY{c+c1}{\PYZsh{}перевели в проценты}


\PY{n}{incom}\PY{o}{.}\PY{n}{drop}\PY{p}{(}\PY{p}{[}\PY{l+s+s1}{\PYZsq{}}\PY{l+s+s1}{debt\PYZus{}x}\PY{l+s+s1}{\PYZsq{}}\PY{p}{,}\PY{l+s+s1}{\PYZsq{}}\PY{l+s+s1}{debt\PYZus{}y}\PY{l+s+s1}{\PYZsq{}}\PY{p}{,}\PY{l+s+s1}{\PYZsq{}}\PY{l+s+s1}{total\PYZus{}income\PYZus{}x}\PY{l+s+s1}{\PYZsq{}}\PY{p}{,}\PY{l+s+s1}{\PYZsq{}}\PY{l+s+s1}{total\PYZus{}income\PYZus{}y}\PY{l+s+s1}{\PYZsq{}}\PY{p}{]}\PY{p}{,} \PY{n}{axis}\PY{o}{=}\PY{l+s+s1}{\PYZsq{}}\PY{l+s+s1}{columns}\PY{l+s+s1}{\PYZsq{}}\PY{p}{,} \PY{n}{inplace}\PY{o}{=}\PY{k+kc}{True}\PY{p}{)}



\PY{n}{incom}
\end{Verbatim}
\end{tcolorbox}

            \begin{tcolorbox}[breakable, boxrule=.5pt, size=fbox, pad at break*=1mm, opacityfill=0]
\prompt{Out}{outcolor}{262}{\hspace{3.5pt}}
\begin{Verbatim}[commandchars=\\\{\}]
            total
less\_000
a. до 50    6.18\%
b. 50-100   8.09\%
c. 100-200  8.62\%
d. 200+     7.07\%
\end{Verbatim}
\end{tcolorbox}
        
    Этот вариант подсчета показаз, что люди с заработком 100-200к рублей
имеют наибольшее количество задолженностей перед банком, Мирнимальные
долги имеют люди, зарабатывающие до 50к Следом за ними 200+ рублей

Далее пересчитаем по варианту,предложенному наставником

    \begin{tcolorbox}[breakable, size=fbox, boxrule=1pt, pad at break*=1mm,colback=cellbackground, colframe=cellborder]
\prompt{In}{incolor}{1}{\hspace{4pt}}
\begin{Verbatim}[commandchars=\\\{\}]
\PY{n}{bank\PYZus{}clients}\PY{p}{[}\PY{l+s+s1}{\PYZsq{}}\PY{l+s+s1}{new}\PY{l+s+s1}{\PYZsq{}}\PY{p}{]} \PY{o}{=} \PY{n}{pd}\PY{o}{.}\PY{n}{qcut}\PY{p}{(}\PY{n}{bank\PYZus{}clients}\PY{p}{[}\PY{l+s+s1}{\PYZsq{}}\PY{l+s+s1}{total\PYZus{}income}\PY{l+s+s1}{\PYZsq{}}\PY{p}{]}\PY{p}{,}\PY{l+m+mi}{4}\PY{p}{)}
\PY{n}{bank\PYZus{}clients}\PY{p}{[}\PY{l+s+s1}{\PYZsq{}}\PY{l+s+s1}{new}\PY{l+s+s1}{\PYZsq{}}\PY{p}{]}\PY{o}{.}\PY{n}{value\PYZus{}counts}\PY{p}{(}\PY{p}{)}
\end{Verbatim}
\end{tcolorbox}

    \begin{Verbatim}[commandchars=\\\{\}]

        ---------------------------------------------------------------------------

        NameError                                 Traceback (most recent call last)

        <ipython-input-1-74c92c456ea7> in <module>
    ----> 1 bank\_clients['new'] = pd.qcut(bank\_clients['total\_income'],4)
          2 bank\_clients['new'].value\_counts()


        NameError: name 'pd' is not defined

    \end{Verbatim}

    Получилось 4 равных категории ``до 107654'', ``до 140116'', ``до
195767'', ``более 195767.5''

    \begin{tcolorbox}[breakable, size=fbox, boxrule=1pt, pad at break*=1mm,colback=cellbackground, colframe=cellborder]
\prompt{In}{incolor}{264}{\hspace{4pt}}
\begin{Verbatim}[commandchars=\\\{\}]
\PY{n}{incom\PYZus{}2} \PY{o}{=} \PY{n}{bank\PYZus{}clients}\PY{o}{.}\PY{n}{loc}\PY{p}{[}\PY{p}{:}\PY{p}{,}\PY{p}{(}\PY{l+s+s1}{\PYZsq{}}\PY{l+s+s1}{total\PYZus{}income}\PY{l+s+s1}{\PYZsq{}}\PY{p}{,}\PY{l+s+s1}{\PYZsq{}}\PY{l+s+s1}{debt}\PY{l+s+s1}{\PYZsq{}}\PY{p}{)}\PY{p}{]}

\PY{n}{incom\PYZus{}2}\PY{p}{[}\PY{l+s+s1}{\PYZsq{}}\PY{l+s+s1}{total\PYZus{}income}\PY{l+s+s1}{\PYZsq{}}\PY{p}{]} \PY{o}{=} \PY{n}{incom\PYZus{}2}\PY{p}{[}\PY{l+s+s1}{\PYZsq{}}\PY{l+s+s1}{total\PYZus{}income}\PY{l+s+s1}{\PYZsq{}}\PY{p}{]}\PY{o}{.}\PY{n}{astype}\PY{p}{(}\PY{l+s+s1}{\PYZsq{}}\PY{l+s+s1}{int}\PY{l+s+s1}{\PYZsq{}}\PY{p}{)}
\PY{k}{def} \PY{n+nf}{salary\PYZus{}count}\PY{p}{(}\PY{n}{incom}\PY{p}{)}\PY{p}{:}
    \PY{k}{if} \PY{n}{incom} \PY{o}{\PYZlt{}} \PY{l+m+mi}{107654}\PY{p}{:}
        \PY{k}{return} \PY{l+s+s1}{\PYZsq{}}\PY{l+s+s1}{a. 0 \PYZhy{} 110}\PY{l+s+s1}{\PYZsq{}}
    \PY{k}{if} \PY{n}{incom} \PY{o}{\PYZlt{}} \PY{l+m+mi}{140116}\PY{p}{:}
        \PY{k}{return} \PY{l+s+s1}{\PYZsq{}}\PY{l+s+s1}{b. 110\PYZhy{}140}\PY{l+s+s1}{\PYZsq{}}
    \PY{k}{if} \PY{n}{incom} \PY{o}{\PYZlt{}} \PY{l+m+mi}{195767}\PY{p}{:}
        \PY{k}{return} \PY{l+s+s1}{\PYZsq{}}\PY{l+s+s1}{c. 140\PYZhy{}195}\PY{l+s+s1}{\PYZsq{}}
    \PY{k}{return} \PY{l+s+s1}{\PYZsq{}}\PY{l+s+s1}{d. 195+}\PY{l+s+s1}{\PYZsq{}}

\PY{n}{incom\PYZus{}2}\PY{p}{[}\PY{l+s+s1}{\PYZsq{}}\PY{l+s+s1}{up\PYZus{}to\PYZus{}000}\PY{l+s+s1}{\PYZsq{}}\PY{p}{]} \PY{o}{=} \PY{n}{incom\PYZus{}2}\PY{p}{[}\PY{l+s+s1}{\PYZsq{}}\PY{l+s+s1}{total\PYZus{}income}\PY{l+s+s1}{\PYZsq{}}\PY{p}{]}\PY{o}{.}\PY{n}{apply}\PY{p}{(}\PY{n}{salary\PYZus{}count}\PY{p}{)}

\PY{n}{incom\PYZus{}new} \PY{o}{=} \PY{n}{incom\PYZus{}2}\PY{o}{.}\PY{n}{pivot\PYZus{}table}\PY{p}{(}\PY{n}{index}\PY{o}{=}\PY{p}{[}\PY{l+s+s1}{\PYZsq{}}\PY{l+s+s1}{up\PYZus{}to\PYZus{}000}\PY{l+s+s1}{\PYZsq{}}\PY{p}{]}\PY{p}{,} \PY{n}{values}\PY{o}{=}\PY{p}{(}\PY{l+s+s1}{\PYZsq{}}\PY{l+s+s1}{debt}\PY{l+s+s1}{\PYZsq{}}\PY{p}{)}\PY{p}{,} \PY{n}{aggfunc}\PY{o}{=}\PY{p}{(}\PY{l+s+s1}{\PYZsq{}}\PY{l+s+s1}{sum}\PY{l+s+s1}{\PYZsq{}}\PY{p}{,}\PY{l+s+s1}{\PYZsq{}}\PY{l+s+s1}{count}\PY{l+s+s1}{\PYZsq{}}\PY{p}{)}\PY{p}{)}



\PY{n}{incom\PYZus{}new}\PY{p}{[}\PY{l+s+s1}{\PYZsq{}}\PY{l+s+s1}{total}\PY{l+s+s1}{\PYZsq{}}\PY{p}{]} \PY{o}{=} \PY{n}{incom\PYZus{}new}\PY{p}{[}\PY{l+s+s1}{\PYZsq{}}\PY{l+s+s1}{sum}\PY{l+s+s1}{\PYZsq{}}\PY{p}{]} \PY{o}{/} \PY{n}{incom\PYZus{}new}\PY{p}{[}\PY{l+s+s1}{\PYZsq{}}\PY{l+s+s1}{count}\PY{l+s+s1}{\PYZsq{}}\PY{p}{]}

\PY{k}{def} \PY{n+nf}{percent}\PY{p}{(}\PY{n}{x}\PY{p}{)}\PY{p}{:}
    \PY{n}{fin} \PY{o}{=} \PY{l+s+s1}{\PYZsq{}}\PY{l+s+si}{\PYZob{}:.2\PYZpc{}\PYZcb{}}\PY{l+s+s1}{\PYZsq{}}\PY{o}{.}\PY{n}{format}\PY{p}{(}\PY{n}{x}\PY{p}{)}
    \PY{k}{return} \PY{n}{fin}
\PY{n}{incom\PYZus{}new}\PY{p}{[}\PY{l+s+s1}{\PYZsq{}}\PY{l+s+s1}{total}\PY{l+s+s1}{\PYZsq{}}\PY{p}{]} \PY{o}{=} \PY{n}{incom\PYZus{}new}\PY{p}{[}\PY{l+s+s1}{\PYZsq{}}\PY{l+s+s1}{total}\PY{l+s+s1}{\PYZsq{}}\PY{p}{]}\PY{o}{.}\PY{n}{apply}\PY{p}{(}\PY{n}{percent}\PY{p}{)}

\PY{n}{incom\PYZus{}new}\PY{o}{.}\PY{n}{drop}\PY{p}{(}\PY{p}{[}\PY{l+s+s1}{\PYZsq{}}\PY{l+s+s1}{sum}\PY{l+s+s1}{\PYZsq{}}\PY{p}{,}\PY{l+s+s1}{\PYZsq{}}\PY{l+s+s1}{count}\PY{l+s+s1}{\PYZsq{}}\PY{p}{]}\PY{p}{,} \PY{n}{axis}\PY{o}{=}\PY{l+s+s1}{\PYZsq{}}\PY{l+s+s1}{columns}\PY{l+s+s1}{\PYZsq{}}\PY{p}{,} \PY{n}{inplace}\PY{o}{=}\PY{k+kc}{True}\PY{p}{)}

\PY{n}{incom\PYZus{}new}
\end{Verbatim}
\end{tcolorbox}

            \begin{tcolorbox}[breakable, boxrule=.5pt, size=fbox, pad at break*=1mm, opacityfill=0]
\prompt{Out}{outcolor}{264}{\hspace{3.5pt}}
\begin{Verbatim}[commandchars=\\\{\}]
            total
up\_to\_000
a. 0 - 110  7.95\%
b. 110-140  8.59\%
c. 140-195  8.76\%
d. 195+     7.13\%
\end{Verbatim}
\end{tcolorbox}
        
    Мы получили практически те же данные, вторая и третья группы склонны
отдавать свои долги реже, нежели 2 оставшиеся группы За исключением
измененных сумм заработков Однако смело можно сделать вывод, что люди,
зарабатывающие от 100 до 200к являются неплательщиками больше, нежели
остальные

    \hypertarget{ux432ux44bux432ux43eux434}{%
\subsubsection{Вывод}\label{ux432ux44bux432ux43eux434}}

    Есть вероятность неверного распределения категорий, однако по данным
следует, что люди зарабатывающие менее 100.000 рублей имеют меньше
неоплаченных кредитов, так же, как и люди, зарабатывающие больше 200.000

Самая ``небезопасная'' категория людей с заработками от 100 до 200 тысяч
рублей Так же в таблице были данные с людьми с нулевыми заработками, и
пропущенным местом работы. Исходя из процентов невозврата можно
предположить, что их заработки находятся в пределах от 200 до 400 тыс
рублей. Я не силен в экономике России, но смею предположить что это
самозанятые люди не трудоустроинные официально.

    Категоризация сделана правильно, с использованием квантилей.

\begin{center}\rule{0.5\linewidth}{0.5pt}\end{center}

    Комментарий наставника

Исходя из чего выбран именно такой способ категоризации столбца с
доходами? Не во всех группах содержится достаточное число клиентов для
построения по ним надежных выводов. Лучше использовать разбиение по
квантилям данных (метод qcut). Тогда группы получатся равного размера.
Или же можно узнать в интернете процентный состав общества по доходам и
затем уже категоризовать столбец с доходами по найденному соотношению с
помощью персентилей.

    DONE

    \begin{itemize}
\tightlist
\item
  Как разные цели кредита влияют на его возврат в срок?
\end{itemize}

    Сказали что нужно иначе

goals = bank\_clients.loc{[}:,(`b\_d',`debt'){]} goals\_sum =
goals.groupby(`b\_d').sum() goals\_count = goals.groupby(`b\_d').count()
goals = pd.merge(goals\_sum, goals\_count, on=(`b\_d'))

goals{[}`total'{]} = goals{[}`debt\_x'{]} / goals{[}`debt\_y'{]}

def percent(x): fin = `\{:.2\%\}'.format(x) return fin
goals{[}`total'{]} = goals{[}`total'{]}.apply(percent) \#перевели в
проценты

\#goals.drop({[}`debt\_x',`debt\_y'{]}, axis=`columns', inplace=True)

goals

    Создадим сгруппированную таблицу по целям кредита

    \begin{tcolorbox}[breakable, size=fbox, boxrule=1pt, pad at break*=1mm,colback=cellbackground, colframe=cellborder]
\prompt{In}{incolor}{265}{\hspace{4pt}}
\begin{Verbatim}[commandchars=\\\{\}]
\PY{n}{data\PYZus{}new} \PY{o}{=} \PY{n}{bank\PYZus{}clients}\PY{o}{.}\PY{n}{pivot\PYZus{}table}\PY{p}{(}\PY{n}{index}\PY{o}{=}\PY{p}{[}\PY{l+s+s1}{\PYZsq{}}\PY{l+s+s1}{b\PYZus{}d}\PY{l+s+s1}{\PYZsq{}}\PY{p}{]}\PY{p}{,} \PY{n}{values}\PY{o}{=}\PY{p}{(}\PY{l+s+s1}{\PYZsq{}}\PY{l+s+s1}{debt}\PY{l+s+s1}{\PYZsq{}}\PY{p}{)}\PY{p}{,} \PY{n}{aggfunc}\PY{o}{=}\PY{p}{(}\PY{l+s+s1}{\PYZsq{}}\PY{l+s+s1}{sum}\PY{l+s+s1}{\PYZsq{}}\PY{p}{,}\PY{l+s+s1}{\PYZsq{}}\PY{l+s+s1}{count}\PY{l+s+s1}{\PYZsq{}}\PY{p}{)}\PY{p}{)}
\PY{n}{data\PYZus{}new}
\end{Verbatim}
\end{tcolorbox}

            \begin{tcolorbox}[breakable, boxrule=.5pt, size=fbox, pad at break*=1mm, opacityfill=0]
\prompt{Out}{outcolor}{265}{\hspace{3.5pt}}
\begin{Verbatim}[commandchars=\\\{\}]
                 count  sum
b\_d
лизинг            4308  403
недвижимость     10207  747
образование       4014  370
потребительский   2942  221
\end{Verbatim}
\end{tcolorbox}
        
    Добавим столбец с процентами

    \begin{tcolorbox}[breakable, size=fbox, boxrule=1pt, pad at break*=1mm,colback=cellbackground, colframe=cellborder]
\prompt{In}{incolor}{266}{\hspace{4pt}}
\begin{Verbatim}[commandchars=\\\{\}]
\PY{n}{data\PYZus{}new}\PY{p}{[}\PY{l+s+s1}{\PYZsq{}}\PY{l+s+s1}{total}\PY{l+s+s1}{\PYZsq{}}\PY{p}{]} \PY{o}{=} \PY{n}{data\PYZus{}new}\PY{p}{[}\PY{l+s+s1}{\PYZsq{}}\PY{l+s+s1}{sum}\PY{l+s+s1}{\PYZsq{}}\PY{p}{]} \PY{o}{/} \PY{n}{data\PYZus{}new}\PY{p}{[}\PY{l+s+s1}{\PYZsq{}}\PY{l+s+s1}{count}\PY{l+s+s1}{\PYZsq{}}\PY{p}{]}
\PY{n}{data\PYZus{}new}
\end{Verbatim}
\end{tcolorbox}

            \begin{tcolorbox}[breakable, boxrule=.5pt, size=fbox, pad at break*=1mm, opacityfill=0]
\prompt{Out}{outcolor}{266}{\hspace{3.5pt}}
\begin{Verbatim}[commandchars=\\\{\}]
                 count  sum     total
b\_d
лизинг            4308  403  0.093547
недвижимость     10207  747  0.073185
образование       4014  370  0.092177
потребительский   2942  221  0.075119
\end{Verbatim}
\end{tcolorbox}
        
    Переведем столбец Тотал в проценты и удалим ненужные столбцы

    \begin{tcolorbox}[breakable, size=fbox, boxrule=1pt, pad at break*=1mm,colback=cellbackground, colframe=cellborder]
\prompt{In}{incolor}{267}{\hspace{4pt}}
\begin{Verbatim}[commandchars=\\\{\}]
\PY{k}{def} \PY{n+nf}{percent}\PY{p}{(}\PY{n}{x}\PY{p}{)}\PY{p}{:}
    \PY{n}{fin} \PY{o}{=} \PY{l+s+s1}{\PYZsq{}}\PY{l+s+si}{\PYZob{}:.2\PYZpc{}\PYZcb{}}\PY{l+s+s1}{\PYZsq{}}\PY{o}{.}\PY{n}{format}\PY{p}{(}\PY{n}{x}\PY{p}{)}
    \PY{k}{return} \PY{n}{fin}
\PY{n}{data\PYZus{}new}\PY{p}{[}\PY{l+s+s1}{\PYZsq{}}\PY{l+s+s1}{total}\PY{l+s+s1}{\PYZsq{}}\PY{p}{]} \PY{o}{=} \PY{n}{data\PYZus{}new}\PY{p}{[}\PY{l+s+s1}{\PYZsq{}}\PY{l+s+s1}{total}\PY{l+s+s1}{\PYZsq{}}\PY{p}{]}\PY{o}{.}\PY{n}{apply}\PY{p}{(}\PY{n}{percent}\PY{p}{)}

\PY{n}{data\PYZus{}new}\PY{o}{.}\PY{n}{drop}\PY{p}{(}\PY{p}{[}\PY{l+s+s1}{\PYZsq{}}\PY{l+s+s1}{count}\PY{l+s+s1}{\PYZsq{}}\PY{p}{,}\PY{l+s+s1}{\PYZsq{}}\PY{l+s+s1}{sum}\PY{l+s+s1}{\PYZsq{}}\PY{p}{]}\PY{p}{,} \PY{n}{axis}\PY{o}{=}\PY{l+s+s1}{\PYZsq{}}\PY{l+s+s1}{columns}\PY{l+s+s1}{\PYZsq{}}\PY{p}{,} \PY{n}{inplace}\PY{o}{=}\PY{k+kc}{True}\PY{p}{)}

\PY{n}{data\PYZus{}new}
\end{Verbatim}
\end{tcolorbox}

            \begin{tcolorbox}[breakable, boxrule=.5pt, size=fbox, pad at break*=1mm, opacityfill=0]
\prompt{Out}{outcolor}{267}{\hspace{3.5pt}}
\begin{Verbatim}[commandchars=\\\{\}]
                 total
b\_d
лизинг           9.35\%
недвижимость     7.32\%
образование      9.22\%
потребительский  7.51\%
\end{Verbatim}
\end{tcolorbox}
        
    \begin{tcolorbox}[breakable, size=fbox, boxrule=1pt, pad at break*=1mm,colback=cellbackground, colframe=cellborder]
\prompt{In}{incolor}{268}{\hspace{4pt}}
\begin{Verbatim}[commandchars=\\\{\}]
\PY{n}{incom\PYZus{}new}
\end{Verbatim}
\end{tcolorbox}

            \begin{tcolorbox}[breakable, boxrule=.5pt, size=fbox, pad at break*=1mm, opacityfill=0]
\prompt{Out}{outcolor}{268}{\hspace{3.5pt}}
\begin{Verbatim}[commandchars=\\\{\}]
            total
up\_to\_000
a. 0 - 110  7.95\%
b. 110-140  8.59\%
c. 140-195  8.76\%
d. 195+     7.13\%
\end{Verbatim}
\end{tcolorbox}
        
    \begin{tcolorbox}[breakable, size=fbox, boxrule=1pt, pad at break*=1mm,colback=cellbackground, colframe=cellborder]
\prompt{In}{incolor}{269}{\hspace{4pt}}
\begin{Verbatim}[commandchars=\\\{\}]
\PY{n}{incom}
\end{Verbatim}
\end{tcolorbox}

            \begin{tcolorbox}[breakable, boxrule=.5pt, size=fbox, pad at break*=1mm, opacityfill=0]
\prompt{Out}{outcolor}{269}{\hspace{3.5pt}}
\begin{Verbatim}[commandchars=\\\{\}]
            total
less\_000
a. до 50    6.18\%
b. 50-100   8.09\%
c. 100-200  8.62\%
d. 200+     7.07\%
\end{Verbatim}
\end{tcolorbox}
        
    \begin{tcolorbox}[breakable, size=fbox, boxrule=1pt, pad at break*=1mm,colback=cellbackground, colframe=cellborder]
\prompt{In}{incolor}{270}{\hspace{4pt}}
\begin{Verbatim}[commandchars=\\\{\}]
\PY{n}{family}
\end{Verbatim}
\end{tcolorbox}

            \begin{tcolorbox}[breakable, boxrule=.5pt, size=fbox, pad at break*=1mm, opacityfill=0]
\prompt{Out}{outcolor}{270}{\hspace{3.5pt}}
\begin{Verbatim}[commandchars=\\\{\}]
                       total
family\_status
гражданский брак       9.32\%
женат / замужем        7.54\%
не женат / не замужем  8.50\%
\end{Verbatim}
\end{tcolorbox}
        
    \hypertarget{ux432ux44bux432ux43eux434}{%
\subsubsection{Вывод}\label{ux432ux44bux432ux43eux434}}

    Разница очевидна. Люди, берущие кредиты на авто и образование возвращают
их реже, чем семейные люди, покупающие жилье и делающие свадьбы. Хотя
это и странно.

    Здорово, метод сводных таблиц освоен. Он пригодится в дальнейших
проектах.

\begin{center}\rule{0.5\linewidth}{0.5pt}\end{center}

    Комментарий наставника

На данном шаге рекомендуется использовать метод сводных таблиц
(pivot\_table()). Однако он является аналогом метода groupby(). Изучи
метод сводных таблиц и примени его хотя бы в одном пункте анализа.

Выводы верные по всем пунктам.

    DONE

    \hypertarget{ux448ux430ux433-4.-ux43eux431ux449ux438ux439-ux432ux44bux432ux43eux434}{%
\subsubsection{Шаг 4. Общий
вывод}\label{ux448ux430ux433-4.-ux43eux431ux449ux438ux439-ux432ux44bux432ux43eux434}}

    Проведенное исследование показало большое количество пропущенных данных,
которые являются важными при выдаче кредита Из этого можно сделать вывод
о возможности улучшить работу людей, заполнявших базу

Итоги исследования показали, что наличие 1 или 2 детей уменьшает
вероятность возвращения кредита, однако многодетные люди охотнее
возвращают, нежели люди с 1 и 2 детьми. Однако отсутствие детей при
выдаче кредита являесть преимуществом

Что касается семейного положения, люди не в отношениях являютя самыми
надежными, за ними следуют семейные люди. Самыми неблагонадежными
являются люди в гражданском браке

По типу кредитования вероятность возврата снижается в следующем порядке:
недвижимость, потребительские, образование, лизинг

    Вывод стал лучше. Приведены ответы на главные вопросы проекта.

\begin{center}\rule{0.5\linewidth}{0.5pt}\end{center}

    Комментарий наставника

Финальный вывод и есть главный результат твоей работы. Стоит писать его
подробно по результатам проведенной работы. В нем можно приводить
полученные в ходе работы значения. Также можно расписать все, что было
сделано в работе.

    DONE

    Помарки исправлены, и теперь работа выполнена хорошо. Успехов в
дальнейших проектах :)

\begin{center}\rule{0.5\linewidth}{0.5pt}\end{center}

    Комментарий наставника

\begin{itemize}
\tightlist
\item
  В начале работы следует описывать данные, с которыми работаешь;
\item
  Используй альтернативный способ обработки пропусков;
\item
  Комментарии по работе делай в отдельных markdown ячейках;
\item
  Удали дубликаты из данных;
\item
  Категоризацию по доходу стоит делать иначе. Смотри данный пункт
  подробнее;
\item
  Изучи метод сводных таблиц и примени его в работе;
\item
  Финальный вывод стоит писать подробно.
\end{itemize}

Ты проделал большую работу, молодец! Исправь отмеченные тут замечания, и
получится очень достойная работа :)

    DONE

    \hypertarget{ux447ux435ux43a-ux43bux438ux441ux442-ux433ux43eux442ux43eux432ux43dux43eux441ux442ux438-ux43fux440ux43eux435ux43aux442ux430}{%
\subsubsection{Чек-лист готовности
проекта}\label{ux447ux435ux43a-ux43bux438ux441ux442-ux433ux43eux442ux43eux432ux43dux43eux441ux442ux438-ux43fux440ux43eux435ux43aux442ux430}}

Поставьте `x' в выполненных пунктах. Далее нажмите Shift+Enter.

    \begin{itemize}
\tightlist
\item[$\boxtimes$]
  открыт файл;
\item[$\boxtimes$]
  файл изучен;
\item[$\boxtimes$]
  определены пропущенные значения;
\item[$\boxtimes$]
  заполнены пропущенные значения;
\item[$\boxtimes$]
  есть пояснение, какие пропущенные значения обнаружены;
\item[$\boxtimes$]
  описаны возможные причины появления пропусков в данных;
\item[$\boxtimes$]
  объяснено, по какому принципу заполнены пропуски;
\item[$\boxtimes$]
  заменен вещественный тип данных на целочисленный;
\item[$\boxtimes$]
  есть пояснение, какой метод используется для изменения типа данных и
  почему;
\item[$\boxtimes$]
  удалены дубликаты;
\item[$\boxtimes$]
  есть пояснение, какой метод используется для поиска и удаления
  дубликатов;
\item[$\boxtimes$]
  описаны возможные причины появления дубликатов в данных;
\item[$\boxtimes$]
  выделены леммы в значениях столбца с целями получения кредита;
\item[$\boxtimes$]
  описан процесс лемматизации;
\item[$\boxtimes$]
  данные категоризированы;
\item[$\boxtimes$]
  есть объяснение принципа категоризации данных;
\item[$\boxtimes$]
  есть ответ на вопрос: ``Есть ли зависимость между наличием детей и
  возвратом кредита в срок?'';
\item[$\boxtimes$]
  есть ответ на вопрос: ``Есть ли зависимость между семейным положением
  и возвратом кредита в срок?'';
\item[$\boxtimes$]
  есть ответ на вопрос: ``Есть ли зависимость между уровнем дохода и
  возвратом кредита в срок?'';
\item[$\boxtimes$]
  есть ответ на вопрос: ``Как разные цели кредита влияют на его возврат
  в срок?'';
\item[$\boxtimes$]
  в каждом этапе есть выводы;
\item[$\boxtimes$]
  есть общий вывод.
\end{itemize}


    % Add a bibliography block to the postdoc
    
    
    
    \end{document}
